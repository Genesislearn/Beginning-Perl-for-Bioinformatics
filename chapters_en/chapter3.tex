\chapter{The Art of Programming}
\label{chap:chapter3}
\minitoc

This chapter provides an overview of how programmers accomplish their jobs. If you already have Perl installed, and you want to get started writing programs for bioinformatics, feel free to skip ahead to \autoref{chap:chapter4}.

Just as visitors to a biology lab tend to have a clueless awe of "all those test tubes," so the newcomer to programming may regard the world of the programmer as a kind of arcane black box full of weird terminology and abstruse skills. So, to make the whole enterprise a little more congenial, let's take a short tour of some important realities that affect all programmers. Two of the most important are practical strategies that good programmers use and where to go to find answers to questions that arise while you are programming. Using a couple of brief narrative case studies, we'll look at how programmers find solutions to problems. \autoref{chap:chapteraa} lists some of the best Perl and bioinformatics resources to help you solve your particular problems. 

\section{Individual Approaches to Programming}
What's the best way to learn programming? The answer depends on what you hope to accomplish. There are several ways to get started. You can:

\begin{itemize}
  \item Take classes of many different kinds
  \item Read a tutorial book like this one
  \item Get the programming manuals and plunge in
  \item Be tutored by a programmer
  \item Identify a program you need
  \item Try any and all of the above until you've managed to write the program
\end{itemize}

The answer also depends on how you choose to learn. Some people prefer classes, because the information is often presented in a well-organized way, and questions can be answered by the teacher. Others learn best with self-paced study.

Some things about learning to program are common to all these approaches. If you've never programmed at all, the information in the following sections is a "heads-up" about what's ahead. 

\section{Edit—Run—Revise (and Save)}
The most important thing about programming is that it's a hands-on learning activity such as dancing, playing music, cooking, or some other family-oriented activity. You can read about it, but you can't actually do it until you actually do it.

While learning to program in Perl, you need to read about how Perl works, as you will in the chapters that follow. You also need to look at plenty of examples of programs. But you especially need to attempt to write your own programs, as you are asked to do in the exercises at the end of the later chapters. Only this kind of direct experience will make you a programmer.

So I want to give you an overview of the most important tasks involved in writing programs, to help you approach your first programs with a clearer idea of what's really involved.

What exactly will you be doing at the computer? The bulk of a programmer's work involves the steps of writing or revising a program in an editor, then running the program and watching how it behaves, and on the basis of that behavior going back and revising the program again. A typical programmer spends more than half of his or her time editing the program.

\subsection{Saves and Backups}
Once you have even a few lines of code written, it's important to save it. In fact, you should always remember to save a version of your program at regular intervals during editing, so if you make a bunch of edits and the computer crashes, you don't lose hours of work. Also, make sure you back up your work on another disk. Hard disks fail, and when yours does, the information on it will be lost. Therefore it's essential to make regular (daily) backups of your work onto some other medium—tape, floppy disk, Zip disk, another hard disk, writable CD—whatever, just so you won't lose all your work if a disk failure occurs.

In addition to backups of your disks, it's also a good idea to save a dated version of your program at regular intervals. This will allow you to go back to an earlier version of your program should that prove necessary.

It's also a good idea to make sure the backups you're making actually work. So, for instance, if you're backing up to a tape drive, try restoring the files from your tape drive every once in a while, just to make sure that the software and the tapes themselves are all working. You may also want to print out ("make a hardcopy") of your programs at regular intervals for extra insurance against system failures. Finally, it's good policy to keep the backups somewhere away from the computer, so in case of fire or other disaster, the backups will be safe.

\subsection{Error Messages}
Fixing errors is an essential step in writing programs. After you've written and edited a program, the next step is to run it to see if it works. Very often, you'll find that you've made some typographical error, like forgetting to put in a semicolon. As a result, your program isn't valid, and you'll get various error messages from the system. You then have to read the error messages and reedit your program to repair the offending code.

These error messages are sometimes rather cryptic. In the event of an error, the Perl interpreter may have some trouble knowing exactly where you went wrong. It may only recognize that there is something wrong. So it guesses where the problem is, and in the process, it may give you some extraneous information.

The most important thing about using error messages is to look at the first one or two error messages and ignore the rest; fix the top problems, and try running the program again. Error messages are often verbose and can run on for several pages. Just ignore everything but the first errors reported. Another important point is that the line numbers reported in those first error messages are usually right. Sometimes they're off by a line, and they're rarely way off. Later on, we'll practice generating and reading error messages.

\subsection{Debugging}
Perhaps your edits created a valid program, and the Perl interpreter reads in your program and runs it. You find, however, that the program isn't doing what you want it to do. Now you have to go back, look at the program, and try to figure out what's wrong.

Perhaps you made a simple mistake, such as adding instead of subtracting. You may have misread the documentation, and you're using the language the wrong way (reread the documentation). You may simply have an inadequate plan for accomplishing your goal (rethink your strategy and reprogram that part of the code). Sometimes you can't see what's wrong, and you have to look elsewhere (try searching newsgroup archives or FAQs or asking colleagues for help).

For errors that are difficult to find, there are programs called debuggers that allow you to run the program step by step, looking at everything that's happening in the program. (\autoref{chap:chapter6} takes an in-depth look at Perl's debugger.)

There are other tools and techniques you can use. For instance, you can examine your program by adding \verb|print| statements that print out intermediate values or results. There are also special helper programs that can observe your program while it's running and then report about it, telling you, for instance, about where the program is spending most of its time. These tools, and others like them, are essential to programming, and you need to learn how to use them. 

\section{An Environment of Programs}
Programming is an exercise in problem solving. It's an iterative, gradual process. Although it can be done by one person alone, it's often a social activity (this surprises many newcomers). It requires developing specific problem-solving skills and learning a few tools. Programming is sometimes tricky and can be frustrating. On the other hand, for those with an aptitude, there's a great sense of satisfaction that comes from building a working program.

Computer programs can be many things, from barely useful, to aesthetically and intellectually stimulating, to important generators of new knowledge. They can be beautiful. (They can also be destructive, stupid, silly, or vicious; they are human creations, after all.) Because writing a program is an iterative, building, gradual process, there can be real satisfaction in seeing the work unfold from simple beginnings to complete structures. For the beginning student, this gradual unfolding of a new program mirrors the gradual mastery of the language.

As our culture began writing and accumulating programs in the middle of the 20th century, a programming environment began to develop. Gradually, we've been accumulating a substantial body of procedural knowledge.  Programs often reflect the fact that they swim in waters populated by many other programs, and beginning programmers can expect to learn a lot from this environment.

\subsection{Open Source Programs}
As programming has become important in the world, it has also become economically valuable. As a result, the source code for many programs is kept hidden to protect commercial assets and stymie the competition.

However, the source code for many of the best and most used programs are freely available for anyone to examine. Freely available source code is called open source. (There are various kinds of copyrights that may attach to open source program code, but they all allow anyone to examine the source code.) The open source movement treats program source code in a similar manner to the way scientists publish their results: publicly and open to unfettered examination and discussion.

The source code for these programs can be a wonderful place for the beginning programmer to learn how professional programmers write. The programs available in open source include the Perl interpreter and a large amount of Perl code, the Linux operating system, the Apache web server, the Netscape web browser, the sendmail mail transfer agent, and much more. 

\section{Programming Strategies}
In order to give you, the beginning programmer, an idea of how programming is done, let's see how an experienced programmer goes about solving problems by giving a couple of instructive case studies.

Imagine that you want to count all the regulatory elements\footnote{A regulatory element is a stretch of DNA used by the cell in the control of a coding region, helping to determine if and when it's used to create a protein.} in a large chunk of DNA that you just got from the sequencing lab. You're a professional bioinformatics programmer. What do you do? There are two possible solutions: find a program or write one yourself.

It's likely there is already a perfectly good, working, and maybe even free program that does exactly what you need. Very often, you can find exactly what you need on the Web and avoid the cost and expense of reinventing the wheel. This is programming at its best—minimal work for maximal effect. It's the classic case of the experimentalist's adage: a day in the library can save you six months in the lab.

An important part of the art of programming is to keep aware of collections of programs that are available. Then you can simply use the code if it does exactly what you need, or you can take an existing program and alter it to suit your own needs. Of course, copyright laws must be observed, but much is available at no cost, especially to educational and nonprofit organizations. Most Perl module code has a copyright, but you are allowed to use it and modify it given certain restrictions. Details are available at the Perl web site and with the particular modules.

How do you find this wonderful, free, and already existing program? The Perl community has an organized collection of such programming code at the Comprehensive Perl Archive Network (CPAN) web site, \href{http://www.CPAN.org}{http://www.CPAN.org}. Try exploring: you'll find it's organized by topic, so it's possible to quickly find, for example, web, statistics, or graphics programs. In our case, you will find the BioPerl module, which includes several useful bioinformatics functions. A module is a collection of Perl code that can be easily loaded and used by your Perl programs.

The most useful kinds of code are convenient libraries or modules that package a suite of functions. These packages offer a great deal of flexibility in creating new programs. Although you still have to program, the job may be only a small fraction of the work of writing the whole program from scratch. For instance, to continue our example of looking for regulatory elements, your search may turn up a convenient module that lists the regulatory elements plus code that takes a list of elements and searches for them in a DNA library. Then all you have to do is combine the existing code, provide the DNA library, and with a little bit of programming, you're done.

There are lots of other places to look for already existing code. You can search the Internet with your favorite search engines. You can browse collections of links for bioinformatics, looking for programs. You can also search the other sources we've already covered, such as newsgroups, relevant experts, etc.

If you haven't hit paydirt yet, and you know that the program will take a significant amount of time to write yourself, you may want to search the literature in the library, and perhaps enlist the aid of a librarian. You can search MEDLINE for articles about regulatory elements, since often an article will advertise code (an actual program in a language like Perl) that the authors will forward. You can consult conference proceedings, books, and journals. Conferences and trade shows are also great places to look around, meet people, and ask questions.

In many cases you succeed, and despite the effort involved, you saved yourself and your laboratory days, weeks, or months of effort.

However, one big warning about modifying existing code: depending on how much alteration is required, it can sometimes be more difficult to modify existing code than to write a whole program from scratch. Why? Well, depending on who wrote the program, it may be difficult just to see what the different parts of the code do. You can't make modifications if you can't understand what methods the program uses in the first place. (We'll talk more about writing readable code, and the importance of comments in code, later.) This factor alone accounts for a large part of the expense of programming; many programs can't be easily read, or understood, so they can't be maintained. Also, testing the program may be difficult for various reasons, and it may take a lot of time and effort to assure yourself that your modifications are working correctly.

Okay, let's say that you spent three days looking for an existing program, and there really wasn't anything available. (Well, there was one program, but it cost \$30,000 which is way outside your budget, and your local programming expert was too busy to write one for you.) So you absolutely have to write the program yourself.

How do you start from scratch and come up with a program that counts the regulatory elements in some DNA? Read on. 

\section{The Programming Process}
You've been assigned to write a program that counts the regulatory elements in DNA. If you've never programmed you probably have no idea of how to start. Let's talk about what you need to know to write the program.

Here's a summary of the steps we'll cover:

\begin{enumerate}
  \item Identify the required inputs, such as data or information given by the user.
  \item Make an overall design for the program, including the general method—the algorithm—by which the program computes the output.
  \item Decide how the outputs will print; for example, to files or displayed graphically.
  \item Refine the overall design by specifying more detail.  
  \item Write the Perl program code.
\end{enumerate}

These steps may be different for shorter or longer programs, but this is the general approach you will take for most of your programming.

\subsection{The Design Phase}
First, you need to conceive a plan for how the program is going to work. This is the overall design of the program and an important step that's usually done before the actual writing of the program begins. Programs are often compared to kitchen recipes, in that they are specific instructions on how to accomplish some task. For instance, you need an idea of what inputs and outputs the program will have. In our example, the input would be the new DNA. You then need a strategy for how the program will do the necessary computing to calculate the desired output from the input.

In our example, the program first needs to collect information from the user: namely, where is the DNA? (This information can be the name of a file that contains the computer representation of the DNA sequence.) The program needs to allow the user to type in the name of a datafile, maybe from the computer screen or from a web page. Then the program has to check if the file exists (and complain if not, as might happen, for instance, if the user misspelled the name) and finally open the file and read in the DNA before continuing.

This simple step deserves some comment. You can put the DNA directly into the program code and avoid having to write this whole part of the program. But by designing the program to read in the DNA, it's more useful, because you won't have to rewrite the program every time you get some new DNA. It's a simple, even obvious idea, but very powerful.

The data your program uses to compute is called the input. Input can come from files, from other programs, from users running the program, from forms filled out on web sites, from email messages, and so forth.  Most programs read in some form of input; some programs don't.

Let's add the list of regulatory elements to the actual program code. You can ask for a file that contains this list, as we did with the DNA, and have the program be capable of searching different lists of regulatory elements. However, in this case, the list you will use isn't going to change, so why bother the user with inputting the name of another file?

Now that we have the DNA and the list of regulatory elements you have to decide in general terms how the program is actually going to search for each regulatory element in the DNA. This step is obviously the critical one, so make sure you get it right. For instance, you want the program to run quickly enough, if the speed of the program is an important consideration.

This is the problem of choosing the correct algorithm for the job. An algorithm is a design for computing a problem (I'll say more about it in a minute). For instance, you may decide to take each regulatory element in turn and search through the DNA from beginning to end for that element before going on to the next one. Or perhaps you may decide to go through the DNA only once, and at each position check each of the regulatory elements to see if it is present. Is there be any advantage to one way or the other? Can you sort the list of regulatory elements so your search can proceed more quickly? For now, let's just say that your choice of algorithm is important.

The final part of the design is to provide some form of output for the results. Perhaps you want the results displayed on a web page, as a simple list on the computer screen, in a printable file, or perhaps all of the above. At this stage, you may need to ask the user for a filename to save the output.

This brings up the problem of how to display results. This question is actually a critically important one. The ideal solution is to display the results in a way that shows the user at a glance the salient features of the computation. You can use graphics, color, maps, little bouncing balls over the unexpected result: there are many options. A program that outputs results that are hard to read is clearly not doing a good job. In fact, output that makes the salient results hard to find or understand can completely negate all the effort you put into writing an elegant program. Enough said for now.

There are several strategies employed by programmers to help create good overall designs. Usually, any program but the smallest is written in several small but interconnecting parts. (We'll see lots of this as we proceed in later chapters.) What will the parts be, and how will they interconnect? The field of software engineering addresses these kinds of issues. At this point I only want to point out that they are very important and mention some of the ways programmers address the need for design.

There are many design methodologies; each have their dedicated adherents. The best approach is to learn what is available and use the best methodology for the job at hand. For instance, in this book I'm teaching a style of programming called imperative programming, relying on dividing a problem into interacting procedures or subroutines (see \autoref{chap:chapter6}), known as structured design. Another popular style is called object-oriented programming, which is also supported by Perl.  

If you're working in a large group of programmers on a big project, the design phase can be very formal and may even be done by different people than the programmers themselves. On the other end of the scale, you will find solitary programmers who just start writing, developing a plan as they write the code. There is no one best way that works for everyone. But no matter how you approach it, as a beginner you still need to have some sort of design in mind before you start writing code. 

\subsection{Algorithms}
An algorithm is the design, or plan, for the computation done by a computer program. (It's actually a tricky term to define, outside of a formal mathematical system, but this is a reasonable definition.) An algorithm is implemented by coding it in a specific computer language, but the algorithm is the idea of the computation. It's often well represented in pseudocode, which gives the idea of a program without actually being a real computer program.

Most programs do simple things. They get filenames from users, open the files, and read in the data. They perform simple calculations and display the results. These are the types of algorithms you'll learn here.

However, the science of algorithms is a deep and fruitful one, with many important implications for bioinformatics. Algorithms can be designed to find new ways of analyzing biological data and of discovering new scientific results. There are certainly many problems in biology whose solutions could be, and will be, substantially advanced by inventing new algorithms.

The science of algorithms includes many clever techniques. As a beginning programmer, you needn't worry about them just yet. At this stage, an introductory chapter in a beginning tutorial on programming, it's not reasonable to go into details about algorithmic methods. Your first task is just to learn how to write in some programming language.  But if you keep at it, you'll start to learn the techniques. A decent textbook to keep around as a reference is a good investment for a serious programmer (see \autoref{chap:chapteraa}).

In the current example that counts regulatory elements in DNA, I suggest a way of proceeding. Take each regulatory element in turn, and search through the DNA for it, before proceeding to the next regulatory element. Other algorithms are also possible; in fact, this is one example from the general problem called string matching, which is one of the most important for bioinformatics, and the study of which has resulted in a variety of clever algorithms.

Algorithms are usually grouped by such problems or by technique, and there is a wealth of material available. For the practical programmer, some of the most valuable materials are collections of algorithms written in specific languages, that can be incorporated into your programs. Use \autoref{chap:chapteraa} as a starting place. Using the collections of code and books given there, it's possible to incorporate many algorithmic techniques in your Perl code with relative ease. 

\subsection{Pseudocode and Code}
Now you have an overall design, including input, algorithm, and output. How do you actually turn this general idea into a design for a program?

A common implementation strategy is to begin by writing what is called pseudo-code. Pseudocode is an informal program, in which there are no details, and formal syntax isn't followed.\footnote{Syntax refers to the rules of grammar. English syntax decrees, "Go to school" not "School go to." Programming languages also have syntax rules.} It doesn't actually run as a program; its purpose is to flesh out an idea of the overall design of a program in a quick and informal way.

For example, in an actual Perl program you might write a bit of code called a subroutine (see \autoref{chap:chapter6}), in this case, a subroutine that gets an answer from a user typing at the keyboard. Such a subroutine may look like this: 

\begin{lstlisting}
sub getanswer {
  print "Type in your answer here :";
  my $answer  = <STDIN>;
  chomp $answer;
  return $answer;
}
\end{lstlisting}

But in pseudocode, you might just say:

\begin{lstlisting}
getanswer
\end{lstlisting}

and worry about the details later.

Here's an example of pseudocode for the program I've been discussing:

\begin{lstlisting}
get the name of DNAfile from the user

read in the DNA from the DNAfile

for each regulatory element
  if element is in DNA, then
    add one to the count

print count
\end{lstlisting}

\subsection{Comments}
Comments are parts of Perl source code that are used as an aid to understanding what the program does. Anything from a \# sign to the end of a line is considered a comment and is ignored by the Perl interpreter. (The exception is the first line of many Perl programs, which looks something like this: \verb|#!/usr/bin/perl|; see \autoref{sec:section4.2.3} in \autoref{chap:chapter4}.)

Comments are of considerable importance in keeping code useful. They typically include a discussion of the overall purpose and design of the program, examples of how to use the program, and detailed notes interspersed throughout the code explaining why that code is there and what it does. In general, a good programmer writes good comments as an integral part of the program. You'll see comments in all the programming examples in this book.

This is important: your code has to be readable by humans as well as computers.

Comments can also be useful when debugging misbehaving programs. If you're having trouble figuring out where a program is going wrong, you can try to selectively comment out different parts of the code. If you find a section that, when commented out, removes the problem, you can then narrow down the part you've commented out until you have a fairly short section of code in which you know where the problem is. This is often a useful debugging approach.

Comments can be used when you turn pseudocode into Perl source code. Pseudocode is not Perl code, so the Perl interpreter will complain about any pseudocode that is not commented out. You can comment out the pseudocode by placing \# signs at the beginning of all pseudocode lines: 

\begin{lstlisting}
#get the name of DNAfile from the user

#read in the DNA from the DNAfile

#for each regulatory element
#  if element is in DNA, then
#    add one to the count

#print count
\end{lstlisting}

As you expand your pseudocode design into Perl code, you can uncomment the Perl code by removing the \# signs. In this way you may have a mixture of Perl and pseudocode, but you can run and test the Perl parts; the Perl interpreter simply ignores commented-out lines.

You can even leave the complete pseudocode design, commented out, intact in the program. This leaves an outline of the program's design that may come in handy when you or someone else tries to read or modify the code.

We've now reached the point where we're ready for actual Perl programming. In \autoref{chap:chapter4} you will learn Perl syntax and begin programming in Perl. As you do, remember the initial phase of designing your program, followed by the cycle you will spend most of your time in: editing the program, running the program, and revising the program. 

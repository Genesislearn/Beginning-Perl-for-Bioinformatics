\chapter{Appendix A. Resources}
\label{chap:chapteraa}
\minitoc

There is a wide array of resource material for Perl and for bioinformatics programming. This list is not at all exhaustive, but it includes those resources, both online and in print, that I think you may find interesting and useful as you expand your Perl programming repertoire.

\section{Perl}
The documentation for Perl is extensive. It includes lists of FAQs (Frequently Asked Questions, with answers), tutorials, precise definitions in the form of Unix-style manpages, and discussions of specific areas. There are various web sites, a well-organized storehouse of useful Perl programs called CPAN, newsgroups that have searchable archives, conferences, and many good books. It's also worth your while to find and cultivate your own local Perl community. Don't be afraid to engage your colleagues, though as your programming skills grow, they're liable to start asking you questions!

As I've mentioned before, Perl is free. It's part of the wider open source movement, which includes such developments as Linux, the Apache web server, and so on. Since Perl is free, it relies on a community of interested parties to develop code and to write documentation. Because of this, you may notice that a lot of the documentation is a bit fragmented (or, in some cases, very fragmented). Still, the level of support for all these projects equals that available for the best of the commercial software packages.

\subsection{Web Site}
\href{http://www.perl.com}{http://www.perl.com}

\begin{adjustwidth}{1cm}{}
This is the starting point for all things Perl. By all means, explore it. From here, you'll find many more sites dedicated to various aspects of Perl programming. Among several, you might find \href{http://www.perl.org}{http://www.perl.org} especially useful.
\end{adjustwidth}

\subsection{CPAN: Comprehensive Perl Archive Network}
\href{http://www.cpan.org/}{http://www.cpan.org/}

\begin{adjustwidth}{1cm}{}
The Comprehensive Perl Archive Network is an important resource and is \textit{the} place to look for Perl modules. It's also a repository for other software, documentation, and web links. Before taking the time to write a program yourself, look here first to see if it has already been written.
\end{adjustwidth}

\subsection{FAQs: Frequently Asked Questions}
\href{http://www.perl.com/pub/v/faqs}{http://www.perl.com/pub/v/faqs}

\begin{adjustwidth}{1cm}{}
FAQs are a compendium of the most common questions newcomers ask, along with answers, that are usually quite helpful. As a beginning programmer, it is a good idea to take the time to read the Perl FAQs—skimming as necessary—in order to get the lay of the land.
\end{adjustwidth}

You should spend at least enough time reading them to get an idea of what sorts of questions are archived in the FAQs. Be sure to check the FAQs before asking for help from a local expert or posting to a newsgroup. Repeatedly asking questions that have already been exhaustively answered in the FAQs, especially on the Perl newsgroups, might be considered irritating.

You'll find that the Perl FAQs are divided into several parts. When consulting FAQs, look for the date when they were last updated; this isn't a big problem with Perl, but in general, you can find lots of out-of-date information on the Web.

\subsubsection{Beginners}
There are several documents aimed at beginners in the FAQs and in the documentation. There are some other beginning books besides this one, mentioned elsewhere in this appendix. There are also some online tutorials and beginners' articles about Perl at \href{http://learn.perl.org}{http://learn.perl.org} (this is new as I write but looks very promising). There are also a number of mailing lists you can subscribe to, including a mailing list called \href{mailto:beginners@perl.org}{beginners@perl.org}, which you can subscribe to by visiting \href{http://lists.perl.org}{http://lists.perl.org}.

\subsection{Online Manuals}
\href{http://www.perl.com/pub/v/documentation}{http://www.perl.com/pub/v/documentation}

\begin{adjustwidth}{1cm}{}
The Perl manual is available online at the Perl web site mentioned earlier. It should also be installed on your computer. You can access it by typing \verb|perldoc perl|. On Unix/Linux systems, you can also type \verb|man perl| to get the beginning manpage. As that explains, the manual is split into several pages. For instance, to find the manual for Perl's built-in functions, type \verb|man perlfunc| or \verb|perldoc|. HTML versions of the manual exist, and they can be installed on your local computer. This is my preferred method of accessing the documentation: it gives you links that make navigating easier, and if it's installed locally, you can use it even when you're not connected to the Internet.
\end{adjustwidth}

\subsection{Books}
There are lots of Perl books. Many of them are excellent; some are not. Here's a short list of the Perl books I've found most useful in my own work.

\textit{Programming Perl, Third Edition}; by Larry Wall, Tom Christiansen, and Jon Orwant; O'Reilly \& Associates. This is the standard book on Perl by the creator of the language. It explains pretty much everything, although it can lag behind the latest version of Perl. So the absolute authority for your installation should be the online manuals. \textit{Programming Perl} covers a lot of ground; it's good as a reference, a tutorial, and as a ripping yarn if you're into that sort of thing. It presents some of the philosophy behind the language, so it's a good way to absorb some of the computer-science mindset. Earlier editions, if you happen to have them, will also serve; I'm particularly fond of the first edition.

\textit{Perl Cookbook}, by Tom Christiansen and Nathan Torkington, O'Reilly \& Associates. This is billed as the companion volume to \textit{Programming Perl}, and so it is. Here, you will find examples that use Perl for different tasks. It's a great help in many situations, and if you will be doing much Perl programming, it's worth taking at least a few hours to peruse it.

\textit{Mastering Algorithms with Perl}; by Jon Orwant, Jarkho Hietaniemi, and John Macdonald; O'Reilly \& Associates. I've mentioned the importance of studying algorithms and this fine book presents many important algorithms in the context of Perl. It explains concepts and gives code; it doesn't, however, teach the mathematics of analyzing and measuring algorithms. Really serious algorithms students will find that information in texts such as \textit{Introduction to Algorithms} by Corman, Leiserson, and Rivest. Even if you're a novice programmer, this is still a valuable book, and you'll find lots of code you'll be able to use.

\textit{Mastering Regular Expressions}; by Jeffrey R. Friedl, O'Reilly \& Associates. A good book on an important topic with excellent coverage of Perl.

\textit{Elements of Programming in Perl}, by Andrew L. Johnson, Manning Publications. This is another book intended for beginners. It's very good, and I recommend it as a supplement to this text.

\textit{Learning Perl, Third Edition}; by Randall L. Schwartz and Tom Christiansen; O'Reilly \& Associates. This is the classic tutorial book on Perl. It's well-written and well-organized. If you've gotten through \textit{Beginning Perl for Bioinformatics}, you should have no trouble with Learning Perl.

\textit{Object-Oriented Perl}, by Damian Conway, Manning Publications. A superb book on the topic suitable for the beginning or advanced programmer.

\subsection{Conference}
\textit{O'Reilly Open Source Convention}. This convention now includes the yearly Perl Conference. It's a chance to attend classes and lectures and meet Perl practitioners of all sorts. There are also regular YAPC (yet another Perl conference) meetings; you'll find the details at the main Perl web site.

\subsection{Newsgroups}
Perl newsgroups are an important resource for programmers. If you've never seen them, they're accessible over the Web (among other ways). They give you the ability to write a message to a large group of people with interests in any of hundreds of specific topics. If you have a question that you haven't been able to answer in the Perl documentation or the FAQs, searching the newsgroups for the topic of your question can often result in an answer. You can also post a question to a newsgroup if you can't find an answer already provided: but this is not often necessary.

I want to emphasize how useful this resource is. The drawback is that there tends to be a "low signal-to-noise ratio": in other words, there's often a lot of uninformative material in newsgroups. But it can be worth wading through; even negative responses (no known solution to the problem) can save you time and effort.

There are a number of newsgroups related to Perl in the comp.lang.perl hierarchy. The search engine deja.com (recently sold to google.com but still available) lets you search the archives of these newsgroups. More information is available in the Perl FAQs about specific newsgroups; for instance, many specific Perl modules have their own newsgroups, mailing lists, or web sites. The CPAN web site is another place to find searchable newsgroup archives.

\section{Computer Science}
Even though you're programming for biological applications, you'll often find yourself venturing into the realm of traditional computer science. Here are some published resources to help you find your way.

\subsection{Algorithms}
\textit{Mastering Algorithms with Perl}; by Jon Orwant, Jarkho Hietaniemi, and John Macdonald; O'Reilly \& Associates. The best book for noncomputer scientists who program in Perl.

\textit{Introduction to Algorithms}; by Thomas H. Cormen, Charles E. Leiserson, and Ronald L. Rivest; MIT Press and McGraw-Hill. This is a really good book on algorithms—in many ways, the best. It's one of the standard university texts (arguably the standard text) at both the graduate and undergraduate levels. It works well as a textbook and as a reference. Its target audience is computer-science students, so there is a fair amount of math included, but even nonmathematical programmers will find this book very helpful.

\textit{Fundamentals of Algorithmics}, by Gilles Brassard and Paul Bratley, Prentice Hall. An easy overview of algorithmic techniques.

\textit{Algorithms on Strings, Trees, and Sequences: Computer Science and Computational Biology}; by Dan Gusfield; Cambridge University Press. This book specializes in algorithms for strings, including such topics as sequence alignment. It's very detailed, but even so, not complete: this is a big field! The best single source on string algorithms, with lots of information about biological sequence similarity.

The following books are for advanced study.

\textit{The Design and Analysis of Computer Algorithms}; by Alfred V. Aho, John E. Hopcroft, and Jeffrey D. Ullman; Addison-Wesley. This is the classic book on the science of algorithms.

\textit{Introduction to Parallel Algorithms and Architectures: Arrays, Trees, Hypercubes}; by Frank Thomson Leighton; Morgan Kaufmann. A comprehensive and rigorous text and reference.

\textit{Randomized Algorithms}, by Rajeev Motwani and Prabhakar Raghavan, Cambridge University Press. A clear, rigorous book.

\subsection{Software Engineering}
\textit{Software Engineering, Second Edition}; by Ian Sommerville; Addison-Wesley. A good, general book that covers lots of important topics and generally avoids taking sides for or against competing styles.

\subsection{Theory of Computer Science}
\textit{Introduction to Automata Theory, Languages, and Computation, Second Edition}; by John E. Hopcroft, Rajeev Motwani, and Jeffrey D. Ullman; Addison-Wesley. The classic text on computer-science theory.

\textit{Computers and Intractability: A Guide to the Theory of Np-Completeness}, by Michael R. Garey and David S. Johnson, W.H. Freeman \& Co. The classic, and superb, book on the topic.

\subsection{General Programming}
\textit{The Unix Programmers Manual}, Steven V. Earhart, ed., Harcourt, Brace and Jovanovich School. This manual for Unix (whatever version of Unix) is a crash course in computer science with an emphasis on programming. The design of the interacting programs, and the concepts of pipes, redirection, processes, and so on, has been one of the great success stories of programming. This manual summarizes the system: Part I documents user programs; Parts II and III document the programming interface. The programmable shell, and the programs grep, awk, and sed were some of the primary inspirations for Perl.

\textit{The C Programming Language}, by Brian W. Kernighan and Dennis M. Ritchie, Prentice Hall PTR. C and C++ are important languages in bioinformatics, and this classic book teaches C. If you work through the book, attempting all the programming exercises, you'll have some excellent programming training.

\textit{Structure and Interpretation of Computer Programs}; by Harold Abelson, Gerald Jay Sussman, and Juke Sussman; MIT Press. A really interesting book that looks deeply at programming in the context of learning a dialect of Lisp.

\textit{The Unix Programming Environment}, by Brian W. Kernighan and Robert Pike, Prentice Hall. This book is fun, and it talks about good software design.

\section{Linux}
If you have a Linux system, you have all the source code for the entire system available (this is also true for some Unix systems). (If it's not installed, you can get it from the distribution CDs, from the web site \href{http://www.linux.org}{http://www.linux.org}, or from the web site of the company that produced your version of Linux.) This is a great resource. You can take a look at how any program is actually written, even the operating system. Now you're really getting into programming!

\section{Bioinformatics}
Bioinformatics is a relatively new discipline that's attracting a lot of attention, so the available resources are multiplying fairly quickly. Here are a few books and other resources to help get you started.

\subsection{Books}
\textit{Developing Bioinformatics Computer Skills}, by Cynthia Gibas and Per Jambeck, O'Reilly \& Associates. This is a really good book for beginners. It covers setting up a Linux workstation and the installation and use of many of the best, and least expensive, bioinformatics programs. It teaches how to use bioinformatics programs, not how to program. It's the most practical bioinformatics book available.

\textit{Introduction to Computational Biology: Maps, Sequences and Genomes}; by Michael S. Waterman; CRC Press. This is a classic book with a predominantly statistical outlook.

\textit{Bioinformatics: A Practical Guide to the Analysis of Genes and Proteins, Second Edition}; edited by Andreas D. Baxecvanis and B.F. Francis Ouellette; John Wiley \& Sons. Includes chapters on a wide range of topics by several authors.

\subsection{Governmental Organizations}
Absolutely essential. The following web sites are for the most important government-sponsored bioinformatics organizations:

\href{http://www.ncbi.nlm.nih.gov/}{http://www.ncbi.nlm.nih.gov/}:

\begin{adjustwidth}{1cm}{}
the National Center for Biotechnology Information (NCBI). The U.S. government center.
\end{adjustwidth}

\href{http://www.embl.org/}{http://www.embl.org/}:

\begin{adjustwidth}{1cm}{}
the European Molecular Biology Laboratory (EMBL). The European Union laboratory.
\end{adjustwidth}

\href{http://www.ebi.ac.uk/}{http://www.ebi.ac.uk/}:

\begin{adjustwidth}{1cm}{}
the European Bioinformatics Institute (EBI) of EMBL.
\end{adjustwidth}

\subsection{Conferences}
Bioinformatics has long been a part of various biology conferences, for instance the Cold Spring Harbor conferences on sequencing. Now there are many conferences with such coverage, often under the heading of "genomics." Here are a few interesting conferences:

\begin{itemize}
  \item \textit{ISMB: Intelligent Systems for Molecular Biology}, now in its ninth year
  \item \textit{Bioinformatics Open Source Conference}, \href{http://www.bioinformatics.org/}{http://www.bioinformatics.org/}
  \item \textit{RECOMB}: Conference on Computational Molecular Biology
\end{itemize}

\section{Molecular Biology}
\textit{Recombinant DNA}, by James Watson, et al., W.H. Freeman \& Co. This book, though getting old for such a fast-moving field, is a gem for programmers and computer scientists entering the bioinformatics field.  Many standard techniques are clearly and briefly explained with excellent illustrations. Look for the second edition, if you can find it.

\textit{Molecular Biology of the Gene, Fourth Edition}, by James Watson, et al., Addison-Wesley. The classic book in molecular biology. It's very detailed; at this level of coverage, it's definitely out of date, but it's—well—a classic. Makes a good reference for the basics.

\textit{Molecular Cell Biology, Fourth Edition}, by Harvey Lodish, et al., W.H.  Freeman \& Co. An excellent and extensive introductory review of cell biology.

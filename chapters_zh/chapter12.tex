\chapter{BLAST}
\label{chap:chapter12}
\minitoc

在生物学研究中,查找序列的相似性是非常重要的。比如,一个研究人员发现了一个可能非常重要的DNA或者蛋白质序列,他想知道这条序列是否已经被别的研究人员发现并且研究表征过了。如果没有的话,这个研究人员想知道这条序列是否和某个物种中的某条序列比较类似。这些信息对于物种中这条序列的功能会提供非常重要的线索。

BLAST(Basic Local Alignment Search Tool)是在生物学研究中最流行的软件工具之一。对于一条查询序列,它会在一个已知序列库中进行测试,来查找相似性。BLAST实际上是一个程序集,根据查询-数据库对的不同有不同的版本,比如核酸-核酸、蛋白质-核酸、蛋白质-蛋白质、核酸-蛋白质等等。

本章将解析这个程序的核酸-核酸版本的输出,也就是\textit{BLASTN}的输出。简便起见,此处我会简单地用BLAST来指代它。本章的主要目标就是来演示如何编写代码使用正则表达式来解析一个BLAST输出文件。代码非常简单而是很基本,但是它确实可以完成这个工作。一旦你理解了基本知识,你就可以向你的解析器中添加更多的特性,或者从网上找一个更加精致的BLAST输出解析器。不管哪种情况,你都需要对输出解析器有足够的了解,才能去使用或者扩展它们。

本章还会对Bioperl进行简要的介绍,它是一个用于生物信息学的Perl模块集。Bioperl项目是一个开源项目的实例,作为使用Perl的生物信息学程序员,你可以好好地利用它。Perl编程语言本身就是一个开源项目。程序以及它的源代码都可以随意使用和修改,只有几个非常合理的限制,并且是免费的奥。

\section{获取BLAST}
有许多不同的BLAST实现,最流行的可能是NCBI(National Center for Biotechnology Information)免费提供的版本:\href{http://www.ncbi.nlm.nih.gov/BLAST/}{http://www.ncbi.nlm.nih.gov/BLAST/}。NCBI网站提供了一个可以公开使用的BALST服务器、全面的数据库集和组织良好的文档和指南文集,还有可供下载的BLAST软件。

另外一个比较流行的实现是华盛顿大学的WU-BLAST。包含其他WU-BLAST服务器列表的主网站可以在\href{http://blast.wustl.edu}{http://blast.wustl.edu}找到\footnote{译者注:最新版本是2009-10-30发布的AB-BLAST(\href{http://blast.advbiocomp.com/}{http://blast.advbiocomp.com/})。}。旧版本的WU-BLAST可以免费获取。如果你是一个研究人员或者非盈利的组织,并且同意开发和维护这个程序的华盛顿大学的许可协议,那么新版的WU-BALST也可以免费获得。如果你在一个大型的研究机构工作,你可能已经有了一个WU-BLAST程序的站点许可证了。如果你是一个营利性的公司,那么就需要为新版本的WU-BLAST程序支付一笔非常昂贵的费用了(如果你想在自己的计算机上运行BLAST,旧版本的程序仍然是免费的)。宾夕法尼亚州立大学也开发了一些BLAST程序,可以在\href{http://bio.cse.psu.edu/}{http://bio.cse.psu.edu/}找到\footnote{译者注:最新版本是2010-01-12发布的LASTZ(\href{http://www.bx.psu.edu/miller\_lab}{http://www.bx.psu.edu/miller\_lab})。}。除了NCBI和WU-BLAST,还有许多其他可以使用的BLAST服务器网站。在谷歌(Google,\href{http://www.google.com}{http://www.google.com})中搜索“BLAST server”就能得到很多结果。

当研究人员使用BLAST的时候,它们面临的一个大问题就是是在公用的BLAST服务器上运行呢,还是在本地运行呢。使用公用的服务器,有一些显著的优势,最大的优势在于BLAST服务器使用的数据库(比如GenBank)总是最新的。要让你自己的这些数据库时刻保持最新,需要大量的硬盘空间,以及拥有高端处理器、大量内存和高速网络连接的计算机,还需要花费大量的时间来设置并监视更新数据库的软件。另一方面,可能你有自己的序列库,想用它来进行BLAST搜索,你需要经常搜索或者进行大量的搜索,又或者你有一些原因不得不使用内部自己的BLAST引擎。这种情况下,对硬件进行投资、在本地运行它就是比较明智的了。

BLAST的在线文档非常详尽,包括了程序用来计算相似度的统计方法的细节内容。在接下来的小节中,我会对这些进行简单的介绍,但是你应该到BLAST的主页上以及NCBI的网站上找到这些优秀的材料,从头到尾把整个内容通读一下,详读需要查阅的内容。此处我们的兴趣并不在于理论知识,而是解析程序的输出。

\section{字符串匹配和同源}
\textit{字符串匹配}是计算机科学中的术语,指在另一个字符串中找到某个嵌入在内的字符串的算法。它有一个悠久且成果卓著的历史,使用不同的技术,开发出了许多字符串匹配算法,用于各种不同的情况。(参看\autoref{chap:chapteraa}中Gusfield的书对其进行了精彩的讲解,且侧重点在于生物学领域。)我们已经进行了不少的字符串匹配,使用绑定操作符用正则表达式来查找基序和其他的文本。

BLAST是一个基本的字符串匹配程序。字符串匹配算法的细节,以及BLAST中使用的算法,已经超出了本书的范畴。但是首先我想来定义几个常常被混淆或者混用的术语。此外,我还会对BLAST涉及的统计进行一个简要的介绍。

生物学的字符串匹配用来查找相似,它是同源的一个指标。查询序列和数据库中序列的\textit{相似度(similarity)}可以用\textit{百分同一性(percent identity)}来衡量,或者是查询序列和数据库中一个序列对应区域完全匹配的碱基数目。它也可以用\textit{保守(conservation)}的程度来衡量,它会找到等价(冗余)密码子或者不改变蛋白质功能的具有相似属性的氨基酸残基之间的匹配。(参看\autoref{chap:chapter8})。序列之间\textit{同源(homology)}表示序列在进化上是相关的。两个序列要么同源,要么不同源,没有同源度这种说法。\footnote{译者注:相似不一定同源。}

冒着把一个复杂主题过度简化的风险,我要对BLAST的统计中的一些方面进行总结。(完整的细节可以参看BLAST的文档。)BLAST搜索的输出中会报告它找到的匹配的一些值和统计信息,主要基于原始值S、打分算法的参数,以及查询和数据库的属性。\textit{原始值S(raw score S)}是对相似性和匹配大小的一种度量。BLAST的输出罗列了按照E值排序的击中(hit)。粗略来说,一个匹配的\textit{E值/期望值(E value,expect value)}衡量的是在一个随机生成的具有同样大小和组成的数据库中字符串匹配(允许空位)的几率。E值越接近0,这个匹配越不可能随机发生。换言之,E值越小,匹配越好。作为BLASTN的一个经验准则,E值小于1的可能是一个比较可靠的击中,E值小于10的可能还需要看看,但这并不是一个硬性规定。(当然,对于蛋白质来说,即使只有很小的百分同一性,也可能是同源的;它的相似度百分比通常比同源DNA要高。)

现在,既然你已经找我的基础知识,就让我们编写代码来解析BLAST的输出吧。首先,你需要把击中分隔开,然后提取出序列,最后找到注释把E值这个统计量显示出来。

\section{BLAST输出文件}
  下面是BLAST输出文件的一部分。通过把\autoref{chap:chapter8}中\textit{sample.dna}这个文件的几行输入到NCBI网站的BLAST程序中,在使用默认参数的情况下得到了这个文件。然后我把输出以文本形式保存到\textit{blast.txt}文件中,在本书的网站上可以找到它。在贯穿本章的解析工作中,我会重复使用它。因为输出有数页之长,此处我把它截断了,只显示文件的开头、中间和结尾部分。

\begin{lstlisting}
BLASTN 2.1.3 [Apr-11-2001]

Reference: Altschul, Stephen F., Thomas L. Madden, Alejandro A. Schaffer,
Jinghui Zhang, Zheng Zhang, Webb Miller, and David J. Lipman (1997),
"Gapped BLAST and PSI-BLAST: a new generation of protein database search
programs",  Nucleic Acids Res. 25:3389-3402.
RID: 991533563-27495-9092
Query=
         (400 letters)

Database: nt
           868,831 sequences; 3,298,558,333 total letters

                                                                   Score     E
Sequences producing significant alignments:                        (bits)  Value

dbj|AB031069.1|AB031069 Homo sapiens PCCX1 mRNA for protein cont...   793  0.0
ref|NM_014593.1| Homo sapiens CpG binding protein (CGBP), mRNA        779  0.0
gb|AF149758.1|AF149758 Homo sapiens CpG binding protein (CGBP) m...   779  0.0
ref|XM_008699.3| Homo sapiens CpG binding protein (CGBP), mRNA        765  0.0
emb|AL136862.1|HSM801830 Homo sapiens mRNA; cDNA DKFZp434F174 (f...   450  e-124
emb|AJ132339.1|HSA132339 Homo sapiens CpG island sequence, subcl...   446  e-123
emb|AJ236590.1|HSA236590 Homo sapiens chromosome 18 CpG island D...   406  e-111
dbj|AK010337.1|AK010337 Mus musculus ES cells cDNA, RIKEN full-l...   234  3e-59
dbj|AK017941.1|AK017941 Mus musculus adult male thymus cDNA, RIK...   210  5e-52
gb|AC009750.7|AC009750 Drosophila melanogaster, chromosome 2L, r...    46  0.017
gb|AE003580.2|AE003580 Drosophila melanogaster genomic scaffold ...    46  0.017
ref|NC_001905.1| Leishmania major chromosome 1, complete sequence      40  1.0
gb|AE001274.1|AE001274 Leishmania major chromosome 1, complete s...    40  1.0
gb|AC008299.5|AC008299 Drosophila melanogaster, chromosome 3R, r...    38  4.1
gb|AC018662.3|AC018662 Human Chromosome 7 clone RP11-339C9, comp...    38  4.1
gb|AE003774.2|AE003774 Drosophila melanogaster genomic scaffold ...    38  4.1
gb|AC008039.1|AC008039 Homo sapiens clone SCb-391H5 from 7q31, c...    38  4.1
gb|AC005315.2|AC005315 Arabidopsis thaliana chromosome II sectio...    38  4.1
emb|AL353748.13|AL353748 Human DNA sequence from clone RP11-317B...    38  4.1

ALIGNMENTS
>dbj|AB031069.1|AB031069 Homo sapiens PCCX1 mRNA for protein containing CXXC
domain 1,
           complete cds
          Length = 2487

 Score =  793 bits (400), Expect = 0.0
 Identities = 400/400 (100%)
 Strand = Plus / Plus

Query: 1   agatggcggcgctgaggggtcttgggggctctaggccggccacctactggtttgcagcgg 60
           ||||||||||||||||||||||||||||||||||||||||||||||||||||||||||||
Sbjct: 1   agatggcggcgctgaggggtcttgggggctctaggccggccacctactggtttgcagcgg 60

Query: 61  agacgacgcatggggcctgcgcaataggagtacgctgcctgggaggcgtgactagaagcg 120
           ||||||||||||||||||||||||||||||||||||||||||||||||||||||||||||
Sbjct: 61  agacgacgcatggggcctgcgcaataggagtacgctgcctgggaggcgtgactagaagcg 120

Query: 121 gaagtagttgtgggcgcctttgcaaccgcctgggacgccgccgagtggtctgtgcaggtt 180
           ||||||||||||||||||||||||||||||||||||||||||||||||||||||||||||
Sbjct: 121 gaagtagttgtgggcgcctttgcaaccgcctgggacgccgccgagtggtctgtgcaggtt 180

Query: 181 cgcgggtcgctggcgggggtcgtgagggagtgcgccgggagcggagatatggagggagat 240
           ||||||||||||||||||||||||||||||||||||||||||||||||||||||||||||
Sbjct: 181 cgcgggtcgctggcgggggtcgtgagggagtgcgccgggagcggagatatggagggagat 240

Query: 241 ggttcagacccagagcctccagatgccggggaggacagcaagtccgagaatggggagaat 300
           ||||||||||||||||||||||||||||||||||||||||||||||||||||||||||||
Sbjct: 241 ggttcagacccagagcctccagatgccggggaggacagcaagtccgagaatggggagaat 300

Query: 301 gcgcccatctactgcatctgccgcaaaccggacatcaactgcttcatgatcgggtgtgac 360
           ||||||||||||||||||||||||||||||||||||||||||||||||||||||||||||
Sbjct: 301 gcgcccatctactgcatctgccgcaaaccggacatcaactgcttcatgatcgggtgtgac 360

Query: 361 aactgcaatgagtggttccatggggactgcatccggatca 400
           ||||||||||||||||||||||||||||||||||||||||
Sbjct: 361 aactgcaatgagtggttccatggggactgcatccggatca 400

>ref|NM_014593.1| Homo sapiens CpG binding protein (CGBP), mRNA


 ... (file truncated here)


>dbj|AK010337.1|AK010337 Mus musculus ES cells cDNA, RIKEN full-length
enriched library,
           clone:2410002I16, full insert sequence
          Length = 2538

 Score =  234 bits (118), Expect = 3e-59
 Identities = 166/182 (91%)
 Strand = Plus / Plus

Query: 219 gagcggagatatggagggagatggttcagacccagagcctccagatgccggggaggacag 278
           ||||||||||||||| |||||||| |||||||  || ||||| ||||||||||| |||||
Sbjct: 260 gagcggagatatggaaggagatggctcagacctggaacctccggatgccggggacgacag 319

Query: 279 caagtccgagaatggggagaatgcgcccatctactgcatctgccgcaaaccggacatcaa 338
           |||||| |||||||||||||| || ||||||||||||||||| |||||||||||||||||
Sbjct: 320 caagtctgagaatggggagaacgctcccatctactgcatctgtcgcaaaccggacatcaa 379

Query: 339 ctgcttcatgatcgggtgtgacaactgcaatgagtggttccatggggactgcatccggat 398
            ||||||||||| || |||||||||||||| |||||||||||||| ||||||||||||||
Sbjct: 380 ttgcttcatgattggatgtgacaactgcaacgagtggttccatggagactgcatccggat 439

Query: 399 ca 400
           ||
Sbjct: 440 ca 441
 Score = 44.1 bits (22), Expect = 0.066
 Identities = 25/26 (96%)
 Strand = Plus / Plus

Query: 118 gcggaagtagttgtgggcgcctttgc 143
           ||||||||||||| ||||||||||||
Sbjct: 147 gcggaagtagttgcgggcgcctttgc 172

>dbj|AK017941.1|AK017941 Mus musculus adult male thymus cDNA, RIKEN
full-length enriched library, clone:5830420C16, full insert sequence
          Length = 1461

 Score =  210 bits (106), Expect = 5e-52
 Identities = 151/166 (90%)
 Strand = Plus / Plus

Query: 235  ggagatggttcagacccagagcctccagatgccggggaggacagcaagtccgagaatggg 294
            |||||||| |||||||  || ||||| ||||||||||| ||||||||||| |||||||||
Sbjct: 1048 ggagatggctcagacctggaacctccggatgccggggacgacagcaagtctgagaatggg 1107

Query: 295  gagaatgcgcccatctactgcatctgccgcaaaccggacatcaactgcttcatgatcggg 354
            ||||| || ||||||||||||||||| ||||||||||||||||| ||||||||||| ||
Sbjct: 1108 gagaacgctcccatctactgcatctgtcgcaaaccggacatcaattgcttcatgattgga 1167

Query: 355  tgtgacaactgcaatgagtggttccatggggactgcatccggatca 400
            |||||||||||||| |||||||||||||| ||||||||||||||||
Sbjct: 1168 tgtgacaactgcaacgagtggttccatggagactgcatccggatca 1213

 Score = 44.1 bits (22), Expect = 0.066
 Identities = 25/26 (96%)
 Strand = Plus / Plus

Query: 118 gcggaagtagttgtgggcgcctttgc 143
           ||||||||||||| ||||||||||||
Sbjct: 235 gcggaagtagttgcgggcgcctttgc 260

>gb|AC009750.7|AC009750 Drosophila melanogaster, chromosome 2L, region 23F-24A,
BAC clone


 ... (file truncated here)


>emb|AL353748.13|AL353748 Human DNA sequence from clone RP11-317B17 on
chromosome 9, complete
             sequence [Homo sapiens]
          Length = 179155

 Score = 38.2 bits (19), Expect = 4.1
 Identities = 22/23 (95%)
 Strand = Plus / Plus

Query: 192   ggcgggggtcgtgagggagtgcg 214
             |||| ||||||||||||||||||
Sbjct: 48258 ggcgtgggtcgtgagggagtgcg 48280

  Database: nt
    Posted date:  May 30, 2001  3:54 AM
  Number of letters in database: -996,408,959
  Number of sequences in database:  868,831

Lambda     K      H
    1.37    0.711     1.31

Gapped
Lambda     K      H
    1.37    0.711     1.31

Matrix: blastn matrix:1 -3
Gap Penalties: Existence: 5, Extension: 2
Number of Hits to DB: 436021
Number of Sequences: 868831
Number of extensions: 436021
Number of successful extensions: 7536
Number of sequences better than 10.0: 19
length of query: 400
length of database: 3,298,558,333
effective HSP length: 20
effective length of query: 380
effective length of database: 3,281,181,713
effective search space: 1246849050940
effective search space used: 1246849050940
T: 0
A: 30
X1: 6 (11.9 bits)
X2: 15 (29.7 bits)
S1: 12 (24.3 bits)
S2: 19 (38.2 bits)
\end{lstlisting}

如你所见,文件包含了三大部分:在开头的是一些头信息,之后是对比对的总结和详细的比对信息,末尾是附加的一些参数和统计信息总结。

\section{解析BLAST输出}
那么为什么要解析BLAST的输出呢?一个原因是看看你的DNA在持续增长的数据库中是否有新的匹配。你可以编写一个程序来自动执行,每天进行一次BLAST查找,然后通过解析击中的总结列表来和前一天的总结列表进行比较,对比它的结果和前一天的结果。如果有新的结果出现,你也可以让程序发送邮件给你。

\subsection{提取注释和比对}
\autoref{exam:example12.1}由一个主程序和两个新的子程序组成。\textit{parse\_blast}和\textit{parse\_blast\_alignment}这两个子程序使用正则表达式来从标量字符串中提取大量的数据位。我之所以选取这种方法,是因为数据本身,数据虽然是结构化的,但是每一行并没有明确指定的功能。(参看\autoref{chap:chapter10}和\autoref{chap:chapter11}中的讨论。)

\textbf{例:12-1:从BLAST输出文件中提取注释和比对}
\lstinputlisting[label=exam:example12.1]{./scripts/example12-1.pl}

The main program does little more than call the parsing subroutine and print the results. The arguments, initialized as empty, are passed by reference (see \autoref{chap:chapter6}).  

The subroutine \textit{parse\_blast} does the top-level parsing job of separating the three sections of a BLAST output file: the annotation at the beginning, the alignments in the middle, and the annotation at the end.  It then calls the \textit{parse\_blast\_alignment} subroutine to extract the individual alignments from that middle alignment section. The data is first read in from the named file with our old friend the \textit{get\_file\_data} subroutine from \autoref{chap:chapter8}. Use the \textit{join} function to store the array of file data into a scalar string.

The three sections of the BLAST output file are separated by the following statement: 

\begin{lstlisting}
($$beginning_annotation, $alignment_section, $$ending_annotation)

  = ($blast_output_file =~ /(.*^ALIGNMENTS\n)(.*)(^  Database:.*)/ms);
\end{lstlisting}

The pattern match contains three parenthesized expressions: 

\begin{lstlisting}
(.*^ALIGNMENTS\n) 
\end{lstlisting}

which is returned into \verb|$$beginning_annotation;|

\begin{lstlisting}
(.*) 
\end{lstlisting}

which is saved in \verb|$alignment_section;| and:

\begin{lstlisting}
(^  Database:.*) 
\end{lstlisting}

which is saved in \verb|$$ending_annotation|.

The use of \verb|$$| instead of \verb|$| at the beginning of two of these variables indicates that they are references to scalar variables. Recall that they were passed in as arguments to the subroutine, where they were preceded by a backslash, like so: 

\begin{lstlisting}
parse_blast(\$beginning_annotation, \$ending_annotation, \%alignments, $filename);
\end{lstlisting}

You've seen references to variables before, starting in \autoref{chap:chapter6}. Let's review them briefly. Within the \textit{parse\_blast} subroutine, those variables with only one \verb|$| are references to the scalar variables. They need an additional \verb|$| to represent actual scalar variables. This is how references work; they need an additional special character to indicate what kinds of variables they are references to. So a reference to a scalar variable needs to start with \verb|$$|, a reference to an array variable needs to start with \verb|@$|, and a reference to a hash variable needs to start with \verb|%$|. 

The regular expression in the previous code snippet matches everything up to the word \verb|ALIGNMENTS| at the end of a line \verb|(.*^ALIGNMENTS\n)|; then everything for a while \verb|(.*)|; then a line that begins with two spaces and the word \verb|Database|: followed by the rest of the file \verb|(^ Database:.*)|. These three expressions in parentheses correspond to the three desired parts of the BLAST output file; the beginning annotation, the alignment section, and the ending annotation.

The alignments saved in \verb|$alignment_section| are separated out by the subroutine \textit{parse\_blast\_alignment}. This subroutine has one important loop: 

\begin{lstlisting}
while($alignment_section =~ /^>.*\n(^(?!>).*\n)+/gm) {
  my($value) = $&;
  my($key) = (split(/\|/, $value)) [1];
  $alignment_hash{$key} = $value;
}
\end{lstlisting}

You're probably thinking that this regular expression looks downright evil. At first glance, regular expressions do sometimes seem incomprehensible, so let's take a closer look. There are a few new things to examine.

The five lines comprise a \verb|while| loop, which (due to the global \verb|/g| modifier on the pattern match in the \verb|while| loop) keeps matching the pattern as many times as it appears in the string. Each time the program cycles through the loop, the pattern match finds the value (the entire alignment), then determines the key. The key and values are saved in the hash \verb|%alignment_hash|.

Here's an example of one of the matches that's found by this \verb|while| loop when parsing the BLAST output shown in \autoref{sect:section12.3}:

\begin{lstlisting}
>emb|AL353748.13|AL353748 Human DNA sequence from clone RP11-317B17 on
chromosome 9, complete
             sequence [Homo sapiens]
          Length = 179155

 Score = 38.2 bits (19), Expect = 4.1
 Identities = 22/23 (95%)
 Strand = Plus / Plus

Query: 192   ggcgggggtcgtgagggagtgcg 214
             |||| ||||||||||||||||||
Sbjct: 48258 ggcgtgggtcgtgagggagtgcg 48280
\end{lstlisting}

This text starts with a line beginning with a \verb|>| character. In the complete BLAST output, sections like these follow one another. What you want to do is start matching from a line beginning with \verb|>| and include all following adjacent lines that don't start with a \verb|>| character. You also want to extract the identifier, which appears between the first and second vertical bar \verb=|= characters on the first line (e.g., \verb|AL353748.13| in this alignment). 

Let's dissect the regular expression:

\begin{lstlisting}
$alignment_section =~ /^>.*\n(^(?!>).*\n)+/gm
\end{lstlisting}

This pattern match, which appears in a \verb|while| loop within the code, has the modifier \verb|m| for multiline. The \verb|m| modifier allows \verb|^| to match any beginning-of-line inside the multiline string, and \verb|$| to match any end-of-line.

The regular expression breaks down as follows. The first part is:

\begin{lstlisting}
^>.*\n
\end{lstlisting}

It looks for \verb|>| at the beginning of the BLAST output line, followed by \verb|.*|, which matches any quantity of anything (except newlines), up to the first newline. In other words, it matches the first line of the alignment.

Here's the rest of the regular expression:

\begin{lstlisting}
(^(?!>).*\n)+
\end{lstlisting}

After the \verb|^| which matches the beginning of the line, you'll see a \textit{negative lookahead assertion}, \verb|(?!>)|, which ensures that a \verb|>| doesn't follow. Next, the \verb|.*| matches all non-newline characters, up to the final \verb|\n| at the end of the line. All of that is wrapped in parentheses with a surrounding \verb|+|, so that you match all the available lines.

Now that you've matched the entire alignment, you want to extract the key and populate the hash with your key and value. Within the \verb|while| loop, the alignment that you just matched is automatically set by Perl as the value of the special variable \verb|$&| and saved in the variable \verb|$value|. Now you need to extract your key from the alignment. It can be found on the first line of the alignment stored in \verb|$value|, between the first and second \verb=|= symbols.

Extracting this identifying key is done using the \verb|split| function, which
breaks the string into an array. The call to \verb|split|: 

\begin{lstlisting}
split(/\|/, $value)
\end{lstlisting}

splits \verb|$value| into pieces delimited by \verb=|= characters. That is, the \verb=|= symbol is used to determine where one list element ends and the next one begins. (Remember that the vertical bar \verb=|= is a metacharacter and must be escaped as \verb=\|=.) By surrounding the call to split with parentheses and adding an array offset (\verb|[1]|), you can isolate the key and save it into \verb|$key|.

Let's step back now and look at \autoref{exam:example12.1} in its entirety. Notice that it's very short—barely more than two pages, including comments.  Although it's not an easy program, due to the complexity of the regular expressions involved, you can make sense of it if you put a little effort into examining the BLAST output files and the regular expressions that parse it.

Regular expressions have lots of complex features, but as a result, they can do lots of useful things. As a Perl programmer, the effort you put into learning them is well worth it and can have significant payoffs down the road. 

\subsection{Parsing BLAST Alignments}
Let's take the parsing of the BLAST output file a little further. Notice that some of the alignments include more than one aligned string—for instance, the alignment for ID AK017941.1, shown again here: 

\begin{lstlisting}
>dbj|AK017941.1|AK017941 Mus musculus adult male thymus cDNA, RIKEN
full-length enriched
            library, clone:5830420C16, full insert sequence
          Length = 1461

 Score =  210 bits (106), Expect = 5e-52
 Identities = 151/166 (90%)
 Strand = Plus / Plus

Query: 235  ggagatggttcagacccagagcctccagatgccggggaggacagcaagtccgagaatggg 294
            |||||||| |||||||  || ||||| ||||||||||| ||||||||||| |||||||||
Sbjct: 1048 ggagatggctcagacctggaacctccggatgccggggacgacagcaagtctgagaatggg 1107

Query: 295  gagaatgcgcccatctactgcatctgccgcaaaccggacatcaactgcttcatgatcggg 354
            ||||| || ||||||||||||||||| ||||||||||||||||| ||||||||||| ||
Sbjct: 1108 gagaacgctcccatctactgcatctgtcgcaaaccggacatcaattgcttcatgattgga 1167

Query: 355  tgtgacaactgcaatgagtggttccatggggactgcatccggatca 400
            |||||||||||||| |||||||||||||| ||||||||||||||||
Sbjct: 1168 tgtgacaactgcaacgagtggttccatggagactgcatccggatca 1213

 Score = 44.1 bits (22), Expect = 0.066
 Identities = 25/26 (96%)
 Strand = Plus / Plus

Query: 118 gcggaagtagttgtgggcgcctttgc 143
           ||||||||||||| ||||||||||||
Sbjct: 235 gcggaagtagttgcgggcgcctttgc 260
\end{lstlisting}

To parse these alignments, we have to parse out each of the matched strings, which in BLAST terminology are called \textit{high-scoring pairs} (HSPs).

Each HSP also contains some annotation, and then the HSP itself. Let's parse each HSP into annotation, query string, and subject string, together with the starting and ending positions of the strings. More parsing is possible; you can extract specific features of the annotation, as well as the locations of identical and nonidentical bases in the HSP, for instance.

\autoref{exam:example12.2} includes a pair of subroutines; one to parse the alignments into their HSPs, and the second to extract the sequences and their end positions. The main program extends \autoref{exam:example12.1} using these new subroutines. 

\textbf{Example 12-2. Parse alignments from BLAST output file}
\lstinputlisting[label=exam:example12.2]{./scripts/example12-2.pl}

\autoref{exam:example12.2} gives the following output:

\begin{lstlisting}
-> Expect value:   5e-52

-> Query string:
ggagatggttcagacccagagcctccagatgccggggaggacagcaagtccgagaatggg
gagaatgcgcccatctactgcatctgccgcaaaccggacatcaactgcttcatgatcgggtgtgacaactgcaatgagt
ggttccatggggactgcatccggatca

-> Query range:    235..400

-> Subject String:
ctggagatggctcagacctggaacctccggatgccggggacgacagcaagtctgagaatg
ggctgagaacgctcccatctactgcatctgtcgcaaaccggacatcaattgcttcatgattggacttgtgacaactgca
acgagtggttccatggagactgcatccggatca

-> Subject range:  1048..1213
\end{lstlisting}

Let's discuss the new features of \autoref{exam:example12.2} and its subroutines.  First notice that the two new subroutines from \autoref{exam:example12.1} have been placed into the \textit{BeginPerlBioinfo.pm} module, so they aren't printed again here.

The main program, \autoref{exam:example12.2}, starts the same as \autoref{exam:example12.1}; it calls the \textit{parse\_blast} subroutine to separate the annotation from the alignments in the BLAST output file.

The next line fetches one of the alignments from the \verb|%alignments| hash, which is then used as the argument to the \textit{parse\_blast\_alignment\_HSP} subroutine, which then returns an array of annotation (as the first element) and HSPs in \verb|@HSPs|. Here you see that not only can a subroutine return an array on a scalar value; it can also return a hash.

Finally, \autoref{exam:example12.2} does the lower-level parsing of an individual HSP by calling the \textit{extract\_HSP\_information} subroutine, and the extracted parts of one of the HSPs are printed.

\autoref{exam:example12.2} shows a certain inconsistency in our design. Some subroutines call their arguments by reference; others call them by value (see \autoref{chap:chapter6}). You may ask: is this a bad thing?

The answer is: not necessarily. The subroutine \textit{parse\_blast} mixes several arguments, and one of them is not a scalar type. Recall that this is a potentially good place to use call-by-reference in Perl. The other subroutines don't mix argument types this way. However, they can be designed to call their arguments by reference.

Continuing with the code, let's examine the subroutine \textit{parse\_blast\_alignment\_HSP}. This takes one of the alignments from the BLAST output and separates out the individual HSP string matches. The technique used is, once again, regular expressions operating on a single string that contains all the lines of the alignment given as the input argument.

The first regular expression parses out the annotation and the section containing the HSPs: 

\begin{lstlisting}
($beginning_annotation, $HSP_section )

= ($alignment =~ /(.*?)(^ Score =.*)/ms);
\end{lstlisting}

The first parentheses in the regular expression is \verb|(.*?)| This is the nongreedy or minimal matching mentioned in \autoref{chap:chapter9}. Here it gobbles up everything before the first line that begins \verb|Score =| (without the \verb|?| after the \verb|*|, it would gobble everything until the final line that begins \verb|Score =|). This is the exact dividing line between the beginning annotation and the HSP string matches.

The next loop and regular expression separates the individual HSP string matches: 

\begin{lstlisting}
while($HSP_section =~ /(^ Score =.*\n)(^(?! Score =).*\n)+/gm) {
  
  push(@HSPs, $&);

}
\end{lstlisting}

This is the same kind of global string match in a \verb|while| loop you've seen
before; it keeps iterating as long as the match can be found. The other
modifier \verb|/m| is the multiline modifier, which enables the metacharacters
\verb|$| and \verb|^| to match before and after embedded newlines.

The expression within the first pair of parentheses—\verb|(^ Score =.*\n)|—matches a line that begins \verb|Score =|, which is the kind of line that begins an HSP string match section.

The code within the second pair of parentheses—\verb|(^(?! Score =).*\n)+|—matches one or more (the \verb|+| following the other parentheses) lines that do not begin with \verb|Score =|. The \verb|?!| at the beginning of the embedded parentheses is the negative lookahead assertion you encountered in \autoref{exam:example12.2}. So, in total, the regular expression captures a line beginning with \verb|Score =| and all succeeding adjacent lines that don't begin with \verb|Score =|. 

\section{Presenting Data}
Up to now, we've relied on the \textit{print} statement to format output. In this section, I introduce three additional Perl features for writing output:

\begin{itemize}
  \item \textit{printf} function
  \item \textit{here} documents
  \item \textit{format} and \verb|write| functions
\end{itemize}

The entire story about these Perl output features is beyond the scope of this book, but I'll tell you just enough to give you an idea of how they can be used.  

\subsection{The printf Function}
The \textit{printf} function is like the \textit{print} function but with extra features that allow you to specify how certain data is printed out. Perl's \textit{printf} function is taken from the C language function of the same name. Here's an example of a \textit{printf} statement: 

\begin{lstlisting}
my $first  = '3.14159265';
my $second  = 76;
my $third = "Hello world!";

printf STDOUT "A float: %6.4f An integer: %-5d and a string: %s\n", 
     $first, $second,  $third;
\end{lstlisting}

This code snippet prints the following:

\begin{lstlisting}
A float:  3.1416 An integer: 76    and a string: Hello world!
\end{lstlisting}

The arguments to the \textit{printf} function consist of a format string, followed by a list of values that are printed as specified by the format string. The format string may also contain any text along with the directives to print the list of values. (You may also specify an optional filehandle in the same manner you would a \verb|print| function.)

The directives consist of a percent sign followed by a required conversion specifier, which in the example includes \verb|f| for floating point, \verb|d| for integer, and \verb|s| for string. The conversion specifier indicates what kind of data is in the variable to be printed. Between the \verb|%| and the conversion specifier, there may be 0 or more flags, an optional minimum field width, an optional precision, and an optional length modifier. The list of values following the format string must contain data that matches the types of directives, in order.

There are many possible options for these flags and specifiers (some are listed in \autoref{chap:chapterab}). Here's what is in \autoref{exam:example12.3}. First, the directive \verb|%6.4f| specifies to print a floating point (that is, a decimal) number, with a minimum width of six characters overall (padded with spaces if necessary), and at most four positions for the decimal part.  You see in the output that, although the \verb|$f| floating-point number gives the value of pi to eight decimal places, the example specifies a precision of four decimal places, which are all that is printed out.  

The \verb|%-5d| directive specifies an integer to be printed in a field of width 5; the - flag causes the number to be left-justified in the field.  Finally, the \verb|%s| directive prints a string.  

\subsection{here Documents}
Now we'll briefly examine \verb|here| documents. These are convenient ways to specify multiline text for output with perhaps some variables to be interpolated, in a way that looks pretty much the same in your code as it will in the output—that is, without a lot of \verb|print| statements or embedded newline \verb|\n| characters. We'll follow \autoref{exam:example12.3} and its output with a discussion. 

\textbf{Example 12-3. Example of here document}
\lstinputlisting[label=exam:example12.3]{./scripts/example12-3.pl}

Here's the output from \autoref{exam:example12.3}:

\begin{lstlisting}
On iteration 0 of the loop!
AAACCCCCCGGGGGGGGTTTTTT

On iteration 1 of the loop!
AAACCCCCCGGGGGGGGTTTTTT
\end{lstlisting}

In \autoref{exam:example12.3}, a \verb|here| document was put in a \textit{for} loop, so that you can see the \verb|$i| variable changing in the printout. The variables are interpolated into a \verb|here| document in the same way they are interpolated into a double-quoted string. Every time they go through the loop, the contents of the \verb|here| document are subject to variable interpolation and are printed out. The terminating string used in this example, HEREDOC, can be any string you specify. (There are several options for dealing with things like indentation and so forth; I won't discuss them here and refer you to the Perl documentation.) Here documents are handy for some tasks, such as when you have a long, multiline document with just a few changes applied each time you print it. A business form letter, with only the addressee changed, is a typical example. Using a \verb|here| document preserves the look of the final output in the code, while allowing variable interpolation.

\subsection{format and write}
Finally, let's take a look at the \textit{format} and \textit{write} functions. \verb|format| is designed to generate reports and can handle page numbers, headers, and various layout options such as centering and left and right justification. It's modelled on the FORTRAN programming-language conventions for formatting and so is particularly handy for producing reports based on that style, such as the PDB file format, in which fields are specified as occupying certain columns on the line.

\autoref{exam:example12.4} is a short example of a format that creates a FASTA-style output. 

\textbf{Example 12-4. Example of format function to produce FASTA output}
\lstinputlisting[label=exam:example12.4]{./scripts/example12-4.pl}

Here's the output of \autoref{exam:example12.4}:

\begin{lstlisting}
>A0000      Highly unlikely DNA.  This DNA is so\ldots
AAAAAACCCCCCCCCCCCCCGGGGGGGGGGGGGGGGGGGGGGTTTTTTTTT
TTTTTTTTTTTT
\end{lstlisting}

After declaring and initializing the variables that fill in the form, the form is defined with: 

\begin{lstlisting}
format STDOUT =
\end{lstlisting}

and the format continues until it reaches the line with a period at the beginning.

The format is composed of three kinds of lines:

\begin{itemize}
  \item A comment beginning with the pound sign \verb|#|
  \item A picture line that specifies the layout of text
  \item An argument line that names the variables that fill in the preceding picture line
\end{itemize}

The picture line and the argument line must be adjacent; they can't be separated by a comment line, for instance.

The first picture line/argument line combo is for the header information: 

\begin{lstlisting}
>@<<<<<<<<< @<<<<<<<<<<<<<<<<<<<<<<<<<<<<<<<<<<<...
$id,        $description
\end{lstlisting}

The picture line has two picture fields in it, associated with the variables \verb|$id| and \verb|$description|, respectively. The picture line begins with a greater-than sign, \verb|>|, which is just text that begins each FASTA file header line, by definition. Then comes the first picture field, which is an \verb|@| sign followed by nine \verb|<| signs. The \verb|@| sign declares a field that has the associated variable interpolated into it. The use of the nine less-than signs specifies that the value should be left-justified, for a total of 10 columns. If the value is bigger than 10 columns, it is truncated. A less-than sign left-justifies, a greater-than sign right-justifies, and a vertical bar \verb=|= centers the data in the field.

The second picture field is almost identical. It is longer and ends with three dots (an ellipsis) which prints if the contents of the variable \verb|$description| can't fit into the length of the picture field (which, in this case, is true.)

The next pair of picture/argument lines is:

\begin{lstlisting}
^<<<<<<<<<<<<<<<<<<<<<<<<<<<<<<<<<<<<<<<<<<<<<<<<<~~
$DNA
\end{lstlisting}

The picture field starts with a caret, which declares a picture field that will handle variable-length records. The line also contains 49 less-than signs, for a total of 50 columns, left-justified. At the end are two tilde \verb|~| signs, which indicate there should be additional lines for the data if it doesn't fit one on one line.

The \verb|write| command simply prints the previously defined format. By default, the output goes to STDOUT, as is done in the example, but you can supply a filehandle to the \verb|format| and \verb|write| statements if you desire.

The upcoming release of Perl 6 will move formats out of the core of the language and make them into a module. Details are not available as of this writing, but this change will probably entail adding a statement such as \verb|use Formats;| near the top of your code in order to load the module for using formats.

\section{Bioperl}
The \textit{Bioperl} project is an important collection of Perl code for bioinformatics that has been in development since 1998. Although Bioperl uses the more advanced object-oriented style of Perl program design, it's possible to take an introductory look here at how it's organized and used.

The main focus of Bioperl modules is to perform sequence manipulation, provide access to various biology databases (both local and web-based), and parse the output of various programs.

Bioperl is available at \href{http://www.bioperl.org/}{http://www.bioperl.org/}. Some of its features rely on having additional Perl modules—available from CPAN (\href{http://www.cpan.org/}{http://www.cpan.org/})—installed. This situation is quite common, and as you do more Perl programming, you'll become familiar with installing modules from CPAN. The Bioperl tutorials include information on installing Bioperl and additional modules for the three major operating systems: Unix or Linux, Mac, and Windows.

Bioperl doesn't provide complete programs. Rather, it provides a fairly large—and growing—set of modules for accomplishing common tasks, including some tasks you've seen in this book. You're responsible for writing the code that holds the modules together. By providing these ready and (usually) easy-to-use modules, Bioperl makes developing bioinformatics applications in Perl faster and easier. There are example programs for most of the modules, which can be examined and modified to get started.

Like many open source projects, Bioperl has suffered from fragmentation and uneven documentation, due to the strictly volunteer and geographically dispersed group of contributors. But recent work on the project leading up to Release 0.7 in March 2001 has significantly improved the project. In particular, there is now enough tutorial information on using the modules to enable you to make good use of the code.

Some difficulties still remain. Most of the code has been developed on Unix or Linux systems. Not all of it works on Macs or Windows operating systems, but most will. There are some documents available at the Bioperl web site that discuss using Bioperl on non-Unix computers, but the bottom line is that you might find that some things don't work.

If you're going to give Bioperl a try (and I strongly recommend you do), you should make sure you have a fairly recent version of Perl installed.  You'll need at least Version 5.004; it would be much better to install the latest stable release from the Perl web site \href{http://www.perl.com}{http://www.perl.com}. 

\subsection{Sample Modules}
To give you an idea of what tasks Bioperl can make easier for you, \autoref{tab:table12.1} displays a representative sample of some of the most useful modules available. 

\begin{table}[!htbp]
  \begin{center}
  \caption{Bioperl modules}
  \label{tab:table12.1}
  \begin{tabu}{X[2,l]X[3,l]}
  \toprule
  Module & Description\\
  \midrule
  Bio::Seq & Sequence object, with features\\
  Bio::SimpleAlign & Multiple alignments held as a set of sequences\\
  Bio::Species & Generic species object\\
  Bio::DB::Ace & Database object interface to ACeDB servers\\
  Bio::DB::GDB & Database object interface to GDB HTTP query\\
  Bio::DB::GenBank & Database object interface to GenBank\\
  Bio::DB::GenPept & Database object interface to GenPept\\
  Bio::DB::NCBIHelper & A collection of routines useful for queries to NCBI databases\\
  Bio::DB::SwissProt & Database object interface to SWISS-PROT retrieval\\
  Bio::Index::Fasta & Interface for indexing FASTA files\\
  Bio::Index::GenBank & Interface for indexing GenBank seq files, that is, flat files in GenBank format\\
  Bio::Location::Simple & Implementation of a simple location on a sequence\\
  Bio::Location::Split & Implementation of a location on a sequence that has multiple locations\\
  Bio::SeqFeature::FeaturePair & Holds pair feature information, e.g., BLAST hits\\
  Bio::SeqFeature::Generic & Generic SeqFeature\\
  Bio::SeqFeature::Similarity & Sequence feature based on similarity\\
  Bio::SeqFeature::SimilarityPair & Sequence feature based on the similarity of two sequences\\
  Bio::SeqFeature::Gene::Exon & Feature representing an exon\\
  Bio::SeqFeature::Gene::GeneStructure & Feature representing an arbitrarily complex structure of a gene\\
  Bio::SeqFeature::Gene::Transcript & Feature representing a transcript\\
  Bio::SeqFeature::Gene::TranscriptI & Interface for a feature representing a transcript of exons, promoter, UTR, and a poly-adenylation site\\
  Bio::Tools::Blast & Bioperl BLAST sequence analysis object\\
  Bio::Tools::BPbl2seq & Lightweight BLAST parser for pair-wise sequence alignment using the BLAST algorithm\\
  Bio::Tools::BPlite &  Lightweight BLAST parser\\
  Bio::Tools::BPpsilite & Lightweight BLAST parser for PSIBLAST reports\\
  Bio::Tools::CodonTable & Bioperl codon table object\\
  Bio::Tools::Fasta & Bioperl FASTA utility object\\
  Bio::Tools::IUPAC & Generates unique seq objects from an ambiguous seq object\\
  Bio::Tools::RestrictionEnzyme & Bioperl object for a restriction endonuclease object\\
  Bio::Tools::SeqPattern & Bioperl object for a sequence pattern or motif\\
  Bio::Tools::SeqStats & Object holding statistics for one particular sequence\\
  Bio::Tools::SeqWords & Object holding n-mer statistics for one sequence\\
  Bio::Tools::Blast::HSP & Bioperl BLAST high-scoring segment pair object\\
  Bio::Tools::Blast::HTML & Bioperl utility module for HTML-formatting BLAST reports\\
  Bio::Tools::Blast::Sbjct & Bioperl BLAST "hit" object\\
  Bio::Tools::Blast::Run::LocalBlast & Bioperl module for running BLAST analyses locally\\
  Bio::Tools::Blast::Run::Webblast & Bioperl module for running BLAST analyses using an HTTP interface\\
  Bio::Tools::Prediction::Exon & Predicted exon feature\\
  Bio::Tools::Prediction::Gene & Predicted gene structure feature\\
  Bio::Variation::AAChange & Sequence change class for polypeptides\\
  Bio::Variation::AAReverseMutate & Point mutation and codon information from single amino acid changes\\
  Bio::Variation::Allele & Sequence object with allele-specific attributes\\
  Bio::Variation::DNAMutation & DNA-level mutation class\\
  Bio::Variation::IO & Handler for sequence variation I/O formats\\
  \bottomrule
  \end{tabu}
  \end{center}
\end{table}

\subsection{Bioperl Tutorial Script}
Bioperl has a tutorial script to help you try out various parts of the package. In this section, I'll show how to start up and run some example computations.

I've mentioned already that you should learn how to download code from CPAN in order to add modules such as Bioperl. A great deal of the usefulness of the Perl programming environment now resides in these modules available on CPAN. This was a design decision: by concentrating on the core Perl language, the Perl designers can focus on making the language as good as they can. The Perl module developers can then concentrate on their many modules. By all means, take a look around the CPAN web site for an idea of the wealth of Perl modules available to you.

I won't give the details of how to install Bioperl here: as mentioned, they are available at the Bioperl web site, or you can visit the CPAN web site for information.  

So, let's assume you've installed the Bioperl module and looked over the tutorial at the Bioperl web site. Now, let's see how to try out some Bioperl programs.

Go to the directory where the Bioperl software has been built on your system. For instance, on my Linux computer, I put the download file \textit{bioperl-0.7.0.tar.gz} into the directory \textit{/usr/local/src}, and then unpacked it with the command: 

\begin{lstlisting}[language=bash]
tar xvzf bioperl-0.7.0.tar.gz
\end{lstlisting}

which creates the source directory \textit{/usr/local/src/bioperl-0.7.0}. After installing the module (check the documentation), you're ready to run the tutorial script.

Change to the source directory and type \verb|perl bptutorial.pl|. Here's the result (I've shown the head of the tutorial to give the author and copyright information): 

\begin{lstlisting}
% head bptutorial.pl 
# $Id: ch12,v 1.44 2001/10/10 20:37:42 troutman Exp mam $

=head1  BioPerl Tutorial

  Cared for by Peter Schattner <schattner@alum.mit.edu>

  Copyright Peter Schattner

   This tutorial includes "snippets" of code and text from various
   Bioperl documents including module documentation, example scripts
% perl bptutorial.pl 

The following numeric arguments can be passed to run the corresponding demo-script.
1 => access_remote_db ,
2 => index_local_db ,
3 => fetch_local_db ,               (# NOTE: needs to be run with demo 2)
4 => sequence_manipulations ,
5 => seqstats_and_seqwords ,
6 => restriction_and_sigcleave ,
7 => other_seq_utilities ,
8 => run_standaloneblast ,
9 => blast_parser ,
10 => bplite_parsing ,
11 => hmmer_parsing ,
12 => run_clustalw_tcoffee ,
13 => run_psw_bl2seq ,
14 => simplealign_univaln ,
15 => gene_prediction_parsing ,
16 => sequence_annotation ,
17 => largeseqs ,
18 => liveseqs ,
19 => demo_variations ,
20 => demo_xml ,

In addition the argument "100" followed by the name of a single
bioperl object will display a list of all the public methods
available from that object and from what object they are inherited.

Using the parameter "0" will run all tests.
Using any other argument (or no argument) will run this display.

So typical command lines might be:
To run all demo scripts:
 > perl -w  bptutorial.pl 0
or to just run the local indexing demos:
 > perl -w  bptutorial.pl 2 3
or to list all the methods available for object Bio::Tools::SeqStats -
 > perl -w  bptutorial.pl 100 Bio::Tools::SeqStats

%
\end{lstlisting}

Now let's try option 9, the BLAST parser, and option 1, \verb|access_remote_db|. So here goes, starting with the BLAST parser: 

\begin{lstlisting}
% perl bptutorial.pl 9

Beginning blast.pm parser example... 

QUERY NAME     : gi|1401126
QUERY DESC     : UNKNOWN
LENGTH         : 504
FILE           : t/blast.report
DATE           : Thu, 16 Apr 1998 18:56:18 -0400
PROGRAM        : TBLASTN
VERSION        : 2.0.4 [Feb-24-1998]</b>
DB-NAME        : Non-redundant GenBank+EMBL+DDBJ+PDB sequences
DB-RELEASE     : Apr 16, 1998  9:38 AM
DB-LETTERS     : 677679054
DB-SEQUENCES   : 336723
GAPPED         : YES
TOTAL HITS     : 100
CHECKED ALL    : YES
FILT FUNC      : NO
SIGNIF HITS    : 4
SIGNIF CUTOFF  : 1.0e-05 (EXPECT-VALUE)
LOWEST EXPECT  : 0.0
HIGHEST EXPECT : 1e-05
HIGHEST EXPECT : 7.6 (OVERALL)
MATRIX         : BLOSUM62
FILTER         : NONE
EXPECT         : 10
LAMBDA, K, H   : 0.270, 0.0470, 0.230 (SHARED STATS)
WORD SIZE      : 13
S              : 42, 74 (SHARED STATS)
GAP CREATION   : 11
GAP EXTENSION  : 1

Number of hits is 4 
Fraction identical for hit 1 is 0.25 
Sequence identities for hsp of hit 1 are 66-68 70 73 76 79 80 87-89 114 117
119 131 144 146 149 150 152 156 162 165 168 170 171 176 178-182 184 187 190
191 205-207 211 214 217 222 226 241 244 245 249 256 266-268 270 278 284 291
296 304 306 309 311 316 319 324 
%
\end{lstlisting}

This is an interesting way to parse BLAST output! Now let's look at the access of the remote DB: 

\begin{lstlisting}
% perl bptutorial.pl 1
Beginning remote database access example... 
seq1 display id is MUSIGHBA1 
seq2 display id is AF303112 
Display id of first sequence in stream is AF041456
% 
\end{lstlisting}

Well, that was less informative as an output, but it seems you can infer that the remote DB access was successful. (By the way, if you're unsuccessful with this, it may be that you're behind a firewall which is denying access—a not uncommon occurrence in universities or large companies.)

The documentation suggests running the \textit{bptutorial.pl} script under the Perl debugger to watch what happens step by step. I concur with that suggestion but won't include the output here. Try it yourself!

Since that last example wasn't much fun, let's try one more: here's the sequence manipulation tutorial: 

\begin{lstlisting}
% perl bptutorial.pl 4

Beginning sequence_manipulations and SeqIO example... 
First sequence in fasta format... 
>Test1
AGCTTTTCATTCTGACTGCAACGGGCAATATGTCTCTGTGTGGATTAAAAAAAGAGTGTC
TGATAGCAGCTTCTGAACTGGTTACCTGCCGTGAGTAAATTAAAATTTTATTGACTTAGG
TCACTAAATACTTTAACCAATATAGGCATAGCGCACAGACAGATAAAAATTACAGAGTAC
ACAACATCCATGAAACGCATTAGCACCACC
Seq object display id is Test1
Sequence is AGCTTTTCATTCTGACTGCAACGGGCAATATGTCTCTGTGTGGATTAAAAAAAGAGTGTCTGATAG
CAGCTTCTGAACTGGTTACCTGCCGTGAGTAAATTAAAATTTTATTGACTTAGGTCACTAAATACTTTAACCAATATA
GGCATAGCGCACAGACAGATAAAAATTACAGAGTACACAACATCCATGAAACGCATTAGCACCACC 
Sequence from 5 to 10 is TTTCAT 
Acc num is unknown 
Moltype is dna 
Primary id is Test1 
Truncated Seq object sequence is TTTCAT 
Reverse complemented sequence 5 to 10  is GTGCTA  
Translated sequence 6 to 15 is LQRAICLCVD 

Beginning 3-frame and alternate codon translation example... 
ctgagaaaataa translated using method defaults   : LRK*
ctgagaaaataa translated as a coding region (CDS): MRK

Translating in all six frames:
 frame: 0 forward: LRK*
 frame: 0 reverse-complement: LFSQ
 frame: 1 forward: *ENX
 frame: 1 reverse-complement: YFLX
 frame: 2 forward: EKI
 frame: 2 reverse-complement: IFS
Translating with all codon tables using method defaults:
1 : LRK*
2 : L*K*
3 : TRK*
4 : LRK*
5 : LSK*
6 : LRKQ
9 : LSN*
10 : LRK*
11 : LRK*
12 : SRK*
13 : LGK*
14 : LSNY
15 : LRK*
16 : LRK*
21 : LSN*
% 
\end{lstlisting}

That was more fun, because this part of Bioperl is doing several things we've done in this book.

I hope this brief look at Bioperl has whetted your appetite for more.  It's a good idea to explore this set of modules. A Perl module for parsing BLAST output called \textit{BPLite.pm} may also be of interest: it's now part of the Bioperl project. 

\section{Exercises}
\textcolor{red}{\textit{Exercise 12.1}}
\begin{adjustwidth}{1cm}{}
  \textit{Basic string matching}. Write a program that looks for a query string in a target string. For instance, if the query string is "gone", it finds a match at position 22 of the target string "goof through the way-gone-osphere." Don't use regular expressions or any of Perl's built-in string-matching abilities; instead, examine individual positions in the strings, compare characters, and invent your own algorithm. 
\end{adjustwidth}

\textcolor{red}{\textit{Exercise 12.2}}
\begin{adjustwidth}{1cm}{}
  Explore the NCBI BLAST web pages at \href{http://www.ncbi.nlm.nih.gov/BLAST}{http://www.ncbi.nlm.nih.gov/BLAST}. Familiarize yourself with the purpose and use of the various component programs and read the tutorial information on the meaning of the statistics. 
\end{adjustwidth}

\textcolor{red}{\textit{Exercise 12.3}}
\begin{adjustwidth}{1cm}{}
  Explore the Bioperl web pages at \href{http://www.bioperl.org}{http://www.bioperl.org}. Download the code and install it on your computer. 
\end{adjustwidth}

\textcolor{red}{\textit{Exercise 12.4}}
\begin{adjustwidth}{1cm}{}
Perform BLAST searches at the NCBI web site. Search with DNA against DNA databases; then search with the same DNA against protein databases, and compare the output.
\end{adjustwidth}

\textcolor{red}{\textit{Exercise 12.5}}
\begin{adjustwidth}{1cm}{}
Perform two BLAST searches with related sequences. Parse the BLAST output of the searches and extract the top 10 hits in the header annotation of each search. Write a program that reports on the differences and similarities between the two searches.
\end{adjustwidth}

\textcolor{red}{\textit{Exercise 12.6}}
\begin{adjustwidth}{1cm}{}
Write a program that uses Bioperl to perform a BLAST search at the NCBI web site, then use Bioperl to parse the BLAST output. 
\end{adjustwidth}

\textcolor{red}{\textit{Exercise 12.7}}
\begin{adjustwidth}{1cm}{}
Using Bioperl modules mixed with your own code, write a program that runs BLAST on a set of DNA sequences and saves the IDs of the list of hits of each BLAST run sorted in arrays. Allow the user to view each list, to view hits in common between multiple lists and hits unique to one of multiple lists. For each hit, enable the user to fetch its entire GenBank record. 
\end{adjustwidth}

\textcolor{red}{\textit{Exercise 12.8}}
\begin{adjustwidth}{1cm}{}
Write an explanation of the code for the subroutine \textit{extract\_HSP\_information}. Be sure to refer to the format of the data the code uses as input. 
\end{adjustwidth}


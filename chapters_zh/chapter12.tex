\chapter{BLAST}
\label{chap:chapter12}
\minitoc

在生物学研究中,查找序列的相似性是非常重要的。比如,一个研究人员发现了一个可能非常重要的DNA或者蛋白质序列,他想知道这条序列是否已经被别的研究人员发现并且研究表征过了。如果没有的话,这个研究人员想知道这条序列是否和某个物种中的某条序列比较类似。这些信息对于物种中这条序列的功能会提供非常重要的线索。

BLAST(Basic Local Alignment Search Tool)是在生物学研究中最流行的软件工具之一。对于一条查询序列,它会在一个已知序列库中进行测试,来查找相似性。BLAST实际上是一个程序集,根据查询-数据库对的不同有不同的版本,比如核酸-核酸、蛋白质-核酸、蛋白质-蛋白质、核酸-蛋白质等等。

本章将解析这个程序的核酸-核酸版本的输出,也就是\textit{BLASTN}的输出。简便起见,此处我会简单地用BLAST来指代它。本章的主要目标就是来演示如何编写代码使用正则表达式来解析一个BLAST输出文件。代码非常简单而是很基本,但是它确实可以完成这个工作。一旦你理解了基本知识,你就可以向你的解析器中添加更多的特性,或者从网上找一个更加精致的BLAST输出解析器。不管哪种情况,你都需要对输出解析器有足够的了解,才能去使用或者扩展它们。

本章还会对Bioperl进行简要的介绍,它是一个用于生物信息学的Perl模块集。Bioperl项目是一个开源项目的实例,作为使用Perl的生物信息学程序员,你可以好好地利用它。Perl编程语言本身就是一个开源项目。程序以及它的源代码都可以随意使用和修改,只有几个非常合理的限制,并且是免费的奥。

\section{获取BLAST}
有许多不同的BLAST实现,最流行的可能是NCBI(National Center for Biotechnology Information)免费提供的版本:\href{http://www.ncbi.nlm.nih.gov/BLAST/}{http://www.ncbi.nlm.nih.gov/BLAST/}。NCBI网站提供了一个可以公开使用的BALST服务器、全面的数据库集和组织良好的文档和指南文集,还有可供下载的BLAST软件。

另外一个比较流行的实现是华盛顿大学的WU-BLAST。包含其他WU-BLAST服务器列表的主网站可以在 \href{http://blast.wustl.edu}{http://blast.wustl.edu} 找到\footnote{译者注:最新版本是2009-10-30发布的AB-BLAST(\href{http://blast.advbiocomp.com/}{http://blast.advbiocomp.com/})。}。旧版本的WU-BLAST可以免费获取。如果你是一个研究人员或者非盈利的组织,并且同意开发和维护这个程序的华盛顿大学的许可协议,那么新版的WU-BALST也可以免费获得。如果你在一个大型的研究机构工作,你可能已经有了一个WU-BLAST程序的站点许可证了。如果你是一个营利性的公司,那么就需要为新版本的WU-BLAST程序支付一笔非常昂贵的费用了(如果你想在自己的计算机上运行BLAST,旧版本的程序仍然是免费的)。宾夕法尼亚州立大学也开发了一些BLAST程序,可以在 \href{http://bio.cse.psu.edu/}{http://bio.cse.psu.edu/} 找到\footnote{译者注:最新版本是2010-01-12发布的LASTZ(\href{http://www.bx.psu.edu/miller\_lab}{http://www.bx.psu.edu/miller\_lab})。}。除了NCBI和WU-BLAST,还有许多其他可以使用的BLAST服务器网站。在谷歌(Google,\href{http://www.google.com}{http://www.google.com})中搜索“BLAST server”就能得到很多结果。

当研究人员使用BLAST的时候,它们面临的一个大问题就是是在公用的BLAST服务器上运行呢,还是在本地运行呢。使用公用的服务器,有一些显著的优势,最大的优势在于BLAST服务器使用的数据库(比如GenBank)总是最新的。要让你自己的这些数据库时刻保持最新,需要大量的硬盘空间,以及拥有高端处理器、大量内存和高速网络连接的计算机,还需要花费大量的时间来设置并监视更新数据库的软件。另一方面,可能你有自己的序列库,想用它来进行BLAST搜索,你需要经常搜索或者进行大量的搜索,又或者你有一些原因不得不使用内部自己的BLAST引擎。这种情况下,对硬件进行投资、在本地运行它就是比较明智的了。

BLAST的在线文档非常详尽,包括了程序用来计算相似度的统计方法的细节内容。在接下来的小节中,我会对这些进行简单的介绍,但是你应该到BLAST的主页上以及NCBI的网站上找到这些优秀的材料,从头到尾把整个内容通读一下,详读需要查阅的内容。此处我们的兴趣并不在于理论知识,而是解析程序的输出。

\section{字符串匹配和同源}
\textit{字符串匹配}是计算机科学中的术语,指在另一个字符串中找到某个嵌入在内的字符串的算法。它有一个悠久且成果卓著的历史,使用不同的技术,开发出了许多字符串匹配算法,用于各种不同的情况。(参看\autoref{chap:chapteraa}中Gusfield的书对其进行了精彩的讲解,且侧重点在于生物学领域。)我们已经进行了不少的字符串匹配,使用绑定操作符用正则表达式来查找基序和其他的文本。

BLAST是一个基本的字符串匹配程序。字符串匹配算法的细节,以及BLAST中使用的算法,已经超出了本书的范畴。但是首先我想来定义几个常常被混淆或者混用的术语。此外,我还会对BLAST涉及的统计进行一个简要的介绍。

生物学的字符串匹配用来查找相似,它是同源的一个指标。查询序列和数据库中序列的\textit{相似度(similarity)}可以用\textit{百分同一性(percent identity)}来衡量,或者是查询序列和数据库中一个序列对应区域完全匹配的碱基数目。它也可以用\textit{保守(conservation)}的程度来衡量,它会找到等价(冗余)密码子或者不改变蛋白质功能的具有相似属性的氨基酸残基之间的匹配。(参看\autoref{chap:chapter8})。序列之间\textit{同源(homology)}表示序列在进化上是相关的。两个序列要么同源,要么不同源,没有同源度这种说法。\footnote{译者注:相似不一定同源。}

冒着把一个复杂主题过度简化的风险,我要对BLAST的统计中的一些方面进行总结。(完整的细节可以参看BLAST的文档。)BLAST搜索的输出中会报告它找到的匹配的一些值和统计信息,主要基于原始值S、打分算法的参数,以及查询和数据库的属性。\textit{原始值S(raw score S)}是对相似性和匹配大小的一种度量。BLAST的输出罗列了按照E值排序的击中(hit)。粗略来说,一个匹配的\textit{E值/期望值(E value,expect value)}衡量的是在一个随机生成的具有同样大小和组成的数据库中字符串匹配(允许空位)的几率。E值越接近0,这个匹配越不可能随机发生。换言之,E值越小,匹配越好。作为BLASTN的一个经验准则,E值小于1的可能是一个比较可靠的击中,E值小于10的可能还需要看看,但这并不是一个硬性规定。(当然,对于蛋白质来说,即使只有很小的百分同一性,也可能是同源的;它的相似度百分比通常比同源DNA要高。)

现在,既然你已经找我的基础知识,就让我们编写代码来解析BLAST的输出吧。首先,你需要把击中分隔开,然后提取出序列,最后找到注释把E值这个统计量显示出来。

\section{BLAST输出文件}
\label{sect:section12.3}
下面是BLAST输出文件的一部分。通过把\autoref{chap:chapter8}中\textit{sample.dna}这个文件的几行输入到NCBI网站的BLAST程序中,在使用默认参数的情况下得到了这个文件。然后我把输出以文本形式保存到\textit{blast.txt}文件中,在本书的网站上可以找到它。在贯穿本章的解析工作中,我会重复使用它。因为输出有数页之长,此处我把它截断了,只显示文件的开头、中间和结尾部分。

\begin{lstlisting}
BLASTN 2.1.3 [Apr-11-2001]

Reference: Altschul, Stephen F., Thomas L. Madden, Alejandro A. Schaffer,
Jinghui Zhang, Zheng Zhang, Webb Miller, and David J. Lipman (1997),
"Gapped BLAST and PSI-BLAST: a new generation of protein database search
programs",  Nucleic Acids Res. 25:3389-3402.
RID: 991533563-27495-9092
Query=
         (400 letters)

Database: nt
           868,831 sequences; 3,298,558,333 total letters

                                                                   Score     E
Sequences producing significant alignments:                        (bits)  Value

dbj|AB031069.1|AB031069 Homo sapiens PCCX1 mRNA for protein cont...   793  0.0
ref|NM_014593.1| Homo sapiens CpG binding protein (CGBP), mRNA        779  0.0
gb|AF149758.1|AF149758 Homo sapiens CpG binding protein (CGBP) m...   779  0.0
ref|XM_008699.3| Homo sapiens CpG binding protein (CGBP), mRNA        765  0.0
emb|AL136862.1|HSM801830 Homo sapiens mRNA; cDNA DKFZp434F174 (f...   450  e-124
emb|AJ132339.1|HSA132339 Homo sapiens CpG island sequence, subcl...   446  e-123
emb|AJ236590.1|HSA236590 Homo sapiens chromosome 18 CpG island D...   406  e-111
dbj|AK010337.1|AK010337 Mus musculus ES cells cDNA, RIKEN full-l...   234  3e-59
dbj|AK017941.1|AK017941 Mus musculus adult male thymus cDNA, RIK...   210  5e-52
gb|AC009750.7|AC009750 Drosophila melanogaster, chromosome 2L, r...    46  0.017
gb|AE003580.2|AE003580 Drosophila melanogaster genomic scaffold ...    46  0.017
ref|NC_001905.1| Leishmania major chromosome 1, complete sequence      40  1.0
gb|AE001274.1|AE001274 Leishmania major chromosome 1, complete s...    40  1.0
gb|AC008299.5|AC008299 Drosophila melanogaster, chromosome 3R, r...    38  4.1
gb|AC018662.3|AC018662 Human Chromosome 7 clone RP11-339C9, comp...    38  4.1
gb|AE003774.2|AE003774 Drosophila melanogaster genomic scaffold ...    38  4.1
gb|AC008039.1|AC008039 Homo sapiens clone SCb-391H5 from 7q31, c...    38  4.1
gb|AC005315.2|AC005315 Arabidopsis thaliana chromosome II sectio...    38  4.1
emb|AL353748.13|AL353748 Human DNA sequence from clone RP11-317B...    38  4.1

ALIGNMENTS
>dbj|AB031069.1|AB031069 Homo sapiens PCCX1 mRNA for protein containing CXXC
domain 1,
           complete cds
          Length = 2487

 Score =  793 bits (400), Expect = 0.0
 Identities = 400/400 (100%)
 Strand = Plus / Plus

Query: 1   agatggcggcgctgaggggtcttgggggctctaggccggccacctactggtttgcagcgg 60
           ||||||||||||||||||||||||||||||||||||||||||||||||||||||||||||
Sbjct: 1   agatggcggcgctgaggggtcttgggggctctaggccggccacctactggtttgcagcgg 60

Query: 61  agacgacgcatggggcctgcgcaataggagtacgctgcctgggaggcgtgactagaagcg 120
           ||||||||||||||||||||||||||||||||||||||||||||||||||||||||||||
Sbjct: 61  agacgacgcatggggcctgcgcaataggagtacgctgcctgggaggcgtgactagaagcg 120

Query: 121 gaagtagttgtgggcgcctttgcaaccgcctgggacgccgccgagtggtctgtgcaggtt 180
           ||||||||||||||||||||||||||||||||||||||||||||||||||||||||||||
Sbjct: 121 gaagtagttgtgggcgcctttgcaaccgcctgggacgccgccgagtggtctgtgcaggtt 180

Query: 181 cgcgggtcgctggcgggggtcgtgagggagtgcgccgggagcggagatatggagggagat 240
           ||||||||||||||||||||||||||||||||||||||||||||||||||||||||||||
Sbjct: 181 cgcgggtcgctggcgggggtcgtgagggagtgcgccgggagcggagatatggagggagat 240

Query: 241 ggttcagacccagagcctccagatgccggggaggacagcaagtccgagaatggggagaat 300
           ||||||||||||||||||||||||||||||||||||||||||||||||||||||||||||
Sbjct: 241 ggttcagacccagagcctccagatgccggggaggacagcaagtccgagaatggggagaat 300

Query: 301 gcgcccatctactgcatctgccgcaaaccggacatcaactgcttcatgatcgggtgtgac 360
           ||||||||||||||||||||||||||||||||||||||||||||||||||||||||||||
Sbjct: 301 gcgcccatctactgcatctgccgcaaaccggacatcaactgcttcatgatcgggtgtgac 360

Query: 361 aactgcaatgagtggttccatggggactgcatccggatca 400
           ||||||||||||||||||||||||||||||||||||||||
Sbjct: 361 aactgcaatgagtggttccatggggactgcatccggatca 400

>ref|NM_014593.1| Homo sapiens CpG binding protein (CGBP), mRNA


 ... (file truncated here)


>dbj|AK010337.1|AK010337 Mus musculus ES cells cDNA, RIKEN full-length
enriched library,
           clone:2410002I16, full insert sequence
          Length = 2538

 Score =  234 bits (118), Expect = 3e-59
 Identities = 166/182 (91%)
 Strand = Plus / Plus

Query: 219 gagcggagatatggagggagatggttcagacccagagcctccagatgccggggaggacag 278
           ||||||||||||||| |||||||| |||||||  || ||||| ||||||||||| |||||
Sbjct: 260 gagcggagatatggaaggagatggctcagacctggaacctccggatgccggggacgacag 319

Query: 279 caagtccgagaatggggagaatgcgcccatctactgcatctgccgcaaaccggacatcaa 338
           |||||| |||||||||||||| || ||||||||||||||||| |||||||||||||||||
Sbjct: 320 caagtctgagaatggggagaacgctcccatctactgcatctgtcgcaaaccggacatcaa 379

Query: 339 ctgcttcatgatcgggtgtgacaactgcaatgagtggttccatggggactgcatccggat 398
            ||||||||||| || |||||||||||||| |||||||||||||| ||||||||||||||
Sbjct: 380 ttgcttcatgattggatgtgacaactgcaacgagtggttccatggagactgcatccggat 439

Query: 399 ca 400
           ||
Sbjct: 440 ca 441
 Score = 44.1 bits (22), Expect = 0.066
 Identities = 25/26 (96%)
 Strand = Plus / Plus

Query: 118 gcggaagtagttgtgggcgcctttgc 143
           ||||||||||||| ||||||||||||
Sbjct: 147 gcggaagtagttgcgggcgcctttgc 172

>dbj|AK017941.1|AK017941 Mus musculus adult male thymus cDNA, RIKEN
full-length enriched library, clone:5830420C16, full insert sequence
          Length = 1461

 Score =  210 bits (106), Expect = 5e-52
 Identities = 151/166 (90%)
 Strand = Plus / Plus

Query: 235  ggagatggttcagacccagagcctccagatgccggggaggacagcaagtccgagaatggg 294
            |||||||| |||||||  || ||||| ||||||||||| ||||||||||| |||||||||
Sbjct: 1048 ggagatggctcagacctggaacctccggatgccggggacgacagcaagtctgagaatggg 1107

Query: 295  gagaatgcgcccatctactgcatctgccgcaaaccggacatcaactgcttcatgatcggg 354
            ||||| || ||||||||||||||||| ||||||||||||||||| ||||||||||| ||
Sbjct: 1108 gagaacgctcccatctactgcatctgtcgcaaaccggacatcaattgcttcatgattgga 1167

Query: 355  tgtgacaactgcaatgagtggttccatggggactgcatccggatca 400
            |||||||||||||| |||||||||||||| ||||||||||||||||
Sbjct: 1168 tgtgacaactgcaacgagtggttccatggagactgcatccggatca 1213

 Score = 44.1 bits (22), Expect = 0.066
 Identities = 25/26 (96%)
 Strand = Plus / Plus

Query: 118 gcggaagtagttgtgggcgcctttgc 143
           ||||||||||||| ||||||||||||
Sbjct: 235 gcggaagtagttgcgggcgcctttgc 260

>gb|AC009750.7|AC009750 Drosophila melanogaster, chromosome 2L, region 23F-24A,
BAC clone


 ... (file truncated here)


>emb|AL353748.13|AL353748 Human DNA sequence from clone RP11-317B17 on
chromosome 9, complete
             sequence [Homo sapiens]
          Length = 179155

 Score = 38.2 bits (19), Expect = 4.1
 Identities = 22/23 (95%)
 Strand = Plus / Plus

Query: 192   ggcgggggtcgtgagggagtgcg 214
             |||| ||||||||||||||||||
Sbjct: 48258 ggcgtgggtcgtgagggagtgcg 48280

  Database: nt
    Posted date:  May 30, 2001  3:54 AM
  Number of letters in database: -996,408,959
  Number of sequences in database:  868,831

Lambda     K      H
    1.37    0.711     1.31

Gapped
Lambda     K      H
    1.37    0.711     1.31

Matrix: blastn matrix:1 -3
Gap Penalties: Existence: 5, Extension: 2
Number of Hits to DB: 436021
Number of Sequences: 868831
Number of extensions: 436021
Number of successful extensions: 7536
Number of sequences better than 10.0: 19
length of query: 400
length of database: 3,298,558,333
effective HSP length: 20
effective length of query: 380
effective length of database: 3,281,181,713
effective search space: 1246849050940
effective search space used: 1246849050940
T: 0
A: 30
X1: 6 (11.9 bits)
X2: 15 (29.7 bits)
S1: 12 (24.3 bits)
S2: 19 (38.2 bits)
\end{lstlisting}

如你所见,文件包含了三大部分:在开头的是一些头信息,之后是对比对的总结和详细的比对信息,末尾是附加的一些参数和统计信息总结。

\section{解析BLAST输出}
那么为什么要解析BLAST的输出呢?一个原因是看看你的DNA在持续增长的数据库中是否有新的匹配。你可以编写一个程序来自动执行,每天进行一次BLAST查找,然后通过解析击中的总结列表来和前一天的总结列表进行比较,对比它的结果和前一天的结果。如果有新的结果出现,你也可以让程序发送邮件给你。

\subsection{提取注释和比对}
\autoref{exam:example12.1}由一个主程序和两个新的子程序组成。\textit{parse\_blast}和\textit{parse\_blast\_alignment}这两个子程序使用正则表达式来从标量字符串中提取大量的数据位。我之所以选取这种方法,是因为数据本身,数据虽然是结构化的,但是每一行并没有明确指定的功能。(参看\autoref{chap:chapter10}和\autoref{chap:chapter11}中的讨论。)

%\textbf{例12-1:从BLAST输出文件中提取注释和比对}
\lstinputlisting[label=exam:example12.1,caption={例12-1:从BLAST输出文件中提取注释和比对}]{./scripts/example12-1.pl}

主程序仅仅是调用了用来解析的子程序,并且输出结果而已。初始值为空的参数,是通过指针进行传递的。(参看\autoref{chap:chapter6})。

子程序\textit{parse\_blast}进行的是顶层解析的工作,它把BLAST输出文件的三部分分割了开来:开头的注释部分,中间的比对部分,和末尾的注释部分。然后,它调用了\textit{parse\_blast\_alignment}子程序来从中间的比对部分中提取出单个的比对。我们首先使用我们的老朋友——\autoref{chap:chapter8}中的\textit{get\_file\_data}子程序从指定的文件中读取数据。用\textit{join}函数把数组中的文件数据存储到一个标量字符串中。

BLAST输出文件中的三大部分是通过下面的语句分割开的:

\begin{lstlisting}
($$beginning_annotation, $alignment_section, $$ending_annotation)
  = ($blast_output_file =~ /(.*^ALIGNMENTS\n)(.*)(^  Database:.*)/ms);
\end{lstlisting}

模式匹配包含了三个用小括号括起来的表达式:

\begin{lstlisting}
(.*^ALIGNMENTS\n) 
\end{lstlisting}

会返回到\verb|$$beginning_annotation;|,

\begin{lstlisting}
(.*) 
\end{lstlisting}

会保存到\verb|$alignment_section;|,最后:

\begin{lstlisting}
(^  Database:.*) 
\end{lstlisting}

会保存到\verb|$$ending_annotation|。

在三个标量中有两个变量的开头使用的是\verb|$$|而非\verb|$|,这表明它们是标量变量的指针。回忆一下当它们作为参数传递给子程序的时候,在它们的前面都有一个斜杠,就像这样:

\begin{lstlisting}
parse_blast(\$beginning_annotation, \$ending_annotation, \%alignments, $filename);
\end{lstlisting}

从\autoref{chap:chapter6}开始,我们就已经见过变量的指针了。让我们简单复习一下。在\textit{parse\_blast}子程序中,只有一个\verb|$|的这些变量都是标量变量的指针。如果要表示真正的标量变量,还需要额外的一个\verb|$|。指针就是这么使用的,它们需要额外的一个特殊字符来表明它们指代的到底是哪种类型的变量。所以一个标量变量的指针需要用\verb|$$|来起始,一个数组变量的指针需要用\verb|@$|来起始,一个散列变量的指针需要用\verb|%$|来起始。

前面代码片段中的正则表达式中,\verb|(.*^ALIGNMENTS\n)|会匹配所有内容一直到行尾的\verb|ALIGNMENTS|单词;然后,\verb|(.*)|匹配所有内容;之后\verb|(^  Database:.*)|会匹配以两个空格和\verb|Database|单词起始的一行,以及文件之后的所有剩余内容。小括号中的这三个表达式正好对应BLAST输出文件中的那三大部分:开头的注释,比对部分,以及结尾的注释。

保存在\verb|$alignment_section|变量中的比对部分用子程序\textit{parse\_blast\_alignment}来进行分割。子程序中有一个重要的循环:

\begin{lstlisting}
while($alignment_section =~ /^>.*\n(^(?!>).*\n)+/gm) {
  my($value) = $&;
  my($key) = (split(/\|/, $value)) [1];
  $alignment_hash{$key} = $value;
}
\end{lstlisting}

你可能会认为这个正则表达式实在是糟糕透顶了。第一眼看上去,正则表达式所做的事情是在让人费解,所以让我们仔细看一下。有几个新的东西需要学习一下。

这五行代码组成了一个\verb|while|循环,它会持续尽可能多地去匹配字符串中出现的模式(这要归因于\verb|while|循环中模式匹配的\verb|/g|全局修饰符)。程序每次循环的时候,模式匹配都会找到它的值(也就是整个比对),进而确定键。键和值都保存到散列\verb|%alignment_hash|中。

当解析\autoref{sect:section12.3}中的BLAST输出时,下面是这个\verb|while|循环找到的一个匹配的例子:

\begin{lstlisting}
>emb|AL353748.13|AL353748 Human DNA sequence from clone RP11-317B17 on
chromosome 9, complete
             sequence [Homo sapiens]
          Length = 179155

 Score = 38.2 bits (19), Expect = 4.1
 Identities = 22/23 (95%)
 Strand = Plus / Plus

Query: 192   ggcgggggtcgtgagggagtgcg 214
             |||| ||||||||||||||||||
Sbjct: 48258 ggcgtgggtcgtgagggagtgcg 48280
\end{lstlisting}

这段文字起始于以一个\verb|>|字符开始的行。在完整的BLAST输出中,像这样的部分一个接一个。你想做的是从以\verb|>|起始的一行开始匹配,把随后相邻的所有行都包含在内,但是不包括以\verb|>|字符起始的行。你还想提取出识别符,它出现在第一行(比如,在这个比对中就是\verb|AL353748.13|)的前两个竖线\verb=|=字符之间。

让我们来剖析一下这个正则表达式:

\begin{lstlisting}
$alignment_section =~ /^>.*\n(^(?!>).*\n)+/gm
\end{lstlisting}

出现在代码的\verb|while|循环中的这个模式匹配,有用于多行匹配的\verb|m|修饰符。\verb|m|修饰符允许\verb|^|去匹配多行字符串中任意的行开头,也允许\verb|$|去匹配任意的行结尾。

正则表达式可以如下进行分解。第一部分是:

\begin{lstlisting}
^>.*\n
\end{lstlisting}

它会寻找BLAST输出中以\verb|>|起始的行,后面跟着\verb|.*|,它会匹配任意数量的任意字符(换行符除外),一直到第一个换行符为止。换言之,它匹配比对的第一行。

下面是正则表达式的剩余部分:

\begin{lstlisting}
(^(?!>).*\n)+
\end{lstlisting}

在你见过\verb|^|会匹配行首之后,你将看到\textit{否定性前瞻断言/前向否定断言/前向否定匹配(negative lookahead assertion)},\verb|(?!>)|,它会确保不会紧跟着一个\verb|>|。接下来,\verb|.*|匹配非换行符的所有字符,直到行尾的最后\verb|\n|。所有的这些都被包裹在小括号中,并且使用了\verb|+|,所有它会匹配所有符合要求的行。

现在,既然你已经匹配了整个的比对,你想把键提出来,用你的键和值来填充散列。在\verb|while|循环中,你刚刚匹配的比对会自动被Perl设置成特殊变量\verb|$&|的值,把它保存到\verb|$value|变量中。现在,你需要从比对中提取出你的键来。可以在保存到\verb|$value|的比对的第一行中找到它,位于第一个和第二个\verb=|=符号之间。

使用\verb|split|函数就可以提取出整个用于识别的键,它会把字符串打断成一个数组。\verb|split|的调用:

\begin{lstlisting}
split(/\|/, $value)
\end{lstlisting}

把\verb|$value|按照\verb=|=字符打断成了多个片段。也就是说,\verb=|=符号用来决定一个列表元素终止和下一个元素起始的位置。(记住,竖线\verb=|=是一个元字符,必须像\verb=\|=这样对它进行转义。)通过用小括号把对\textit{split}的调用包裹起来,然后添加一个数组偏移量(\verb|[1]|),你可以把键分离出来,保存到\verb|$key|中。

现在让我们后退一步,来整体看一下\autoref{exam:example12.1}。注意,它非常简短——只有两页多一点,这还把注释也算在内了。尽管这并不是一个简单的程序,因为它里面使用了复杂的正则表达式,但如果你能够在BLAST输出文件和解析它的正则表达式上花点功夫,你还是能够理解它的。

正则表达式有许多复杂的特性,也正因为如此,它们可以做大量有用的事情。作为一个Perl程序员,你花在学习它们上的努力是非常值得的,在以后的编程之路上会给你巨额的回报。

\subsection{解析BLAST比对}
让我们把对BLAST输出文件的解析更进一步。注意,有些比对包含不知一个比对字符串——比如,ID为AK017941.1的比对,再次展示如下:

\begin{lstlisting}
>dbj|AK017941.1|AK017941 Mus musculus adult male thymus cDNA, RIKEN
full-length enriched
            library, clone:5830420C16, full insert sequence
          Length = 1461

 Score =  210 bits (106), Expect = 5e-52
 Identities = 151/166 (90%)
 Strand = Plus / Plus

Query: 235  ggagatggttcagacccagagcctccagatgccggggaggacagcaagtccgagaatggg 294
            |||||||| |||||||  || ||||| ||||||||||| ||||||||||| |||||||||
Sbjct: 1048 ggagatggctcagacctggaacctccggatgccggggacgacagcaagtctgagaatggg 1107

Query: 295  gagaatgcgcccatctactgcatctgccgcaaaccggacatcaactgcttcatgatcggg 354
            ||||| || ||||||||||||||||| ||||||||||||||||| ||||||||||| ||
Sbjct: 1108 gagaacgctcccatctactgcatctgtcgcaaaccggacatcaattgcttcatgattgga 1167

Query: 355  tgtgacaactgcaatgagtggttccatggggactgcatccggatca 400
            |||||||||||||| |||||||||||||| ||||||||||||||||
Sbjct: 1168 tgtgacaactgcaacgagtggttccatggagactgcatccggatca 1213

 Score = 44.1 bits (22), Expect = 0.066
 Identities = 25/26 (96%)
 Strand = Plus / Plus

Query: 118 gcggaagtagttgtgggcgcctttgc 143
           ||||||||||||| ||||||||||||
Sbjct: 235 gcggaagtagttgcgggcgcctttgc 260
\end{lstlisting}

要解析这样的比对,我们必须把每一个匹配的字符串都解析出来,在BLAST中它的专用术语叫做\textit{高分值片段对(high-scoring pairs,HSPs)}。

每一个HSP也都包含一些注释,然后才是HSP本身。让我们把每一个HSP解析成注释、查询字符串和主题字符串,以及字符串起始和终止的位置。更多的解析也是有可能的,比如,你可以提取出注释中的特定特征,以及HSP中相同和不同碱基的位置。

\autoref{exam:example12.2}包含了一对子程序:第一个把比对解析成它们的HSPs,第二个提取出序列和它们终止的位置。主程序对\autoref{exam:example12.1}进行了扩展,使用了这两个新的子程序。

%\textbf{例12-2:从BLAST输出文件中解析比对}
\lstinputlisting[label=exam:example12.2,caption={例12-2:从BLAST输出文件中解析比对}]{./scripts/example12-2.pl}

\autoref{exam:example12.2}给出如下的输出:

\begin{lstlisting}
-> Expect value:   5e-52

-> Query string:   ggagatggttcagacccagagcctccagatgccggggaggacagcaagtccgagaatggg
gagaatgcgcccatctactgcatctgccgcaaaccggacatcaactgcttcatgatcgggtgtgacaactgcaatgagt
ggttccatggggactgcatccggatca

-> Query range:    235..400

-> Subject String: ggagatggctcagacctggaacctccggatgccggggacgacagcaagtctgagaatggg
gagaacgctcccatctactgcatctgtcgcaaaccggacatcaattgcttcatgattggatgtgacaactgcaacgagt
ggttccatggagactgcatccggatca

-> Subject range:  1048..1213
\end{lstlisting}

让我们讨论一下\autoref{exam:example12.2}和它子程序的新特性。首先注意,\autoref{exam:example12.1}中的两个新子程序已经放到了\textit{BeginPerlBioinfo.pm}模块里面,所以就不需要再次在此处把它们打印出来了。

\autoref{exam:example12.2}的主程序,开头部分和\autoref{exam:example12.1}完全一样。它调用\textit{parse\_blast}子程序来把BLAST输出文件中的注释和比对分割开来。

接下来的一行从\verb|%alignment|散列中提取出一个比对,随后它就被用作了\textit{parse\_blast\_alignment\_HSP}子程序的参数,这个子程序会返回由注释(第一个元素)和HSPs组成的数组\verb|@HSPs|。

最后,\autoref{exam:example12.2}通过调用\textit{extract\_HSP\_information}子程序对单个的HSP进行了更低层次的解析,并且把从一个HSP中提取的各部分打印了出来。

\autoref{exam:example12.2}演示的和我们的设计互相矛盾。有些子程序通过指针调用它们的参数,而其他的则通过值调用它们(参看\autoref{chap:chapter6})。你可能会问:这样是不是不太好呢?

答案是:不一定。子程序\textit{parse\_blast}混合了好几个参数,而且其中一个并不是标量类型。回忆一下,在Perl中,这可能是使用指针进行调用的一个好地方。其他的子程序并没像它这样混合参数的类型。然而,也可以设计来通过指针调用它们的参数。

继续讨论我们的代码,来看一下子程序\textit{parse\_blast\_alignment\_HSP}。它处理BLAST输出中的一个比对,把单个的HSP字符串匹配分割开来。此处使用的技术还是正则表达式,在一个单个的包含所有比对行的字符串上使用正则表达式,这个比对是作为输入参数指定的。

第一个正则表达式解析出注释以及包含HSPs的部分:

\begin{lstlisting}
($beginning_annotation, $HSP_section )

= ($alignment =~ /(.*?)(^ Score =.*)/ms);
\end{lstlisting}

正则表达式中的第一个小括号是\verb|(.*?)|。这是在\autoref{chap:chapter9}中提到的非贪婪匹配或者最小化匹配,它会匹配尽可能短的字符串。默认\verb|*|是贪婪的,会匹配尽可能长的字符串。此处,它会匹配第一行以\verb|Score =|开头的行之前的所有内容(如果没有\verb|*|后面的\verb|?|的话,它会匹配最后一行以\verb|Score =|开头的行之前的所有内容)。这正好是开头的注释和HSP字符串匹配之间的分隔行。

下一个循环和正则表达式把单个的HSP字符串匹配分割开来:

\begin{lstlisting}
while($HSP_section =~ /(^ Score =.*\n)(^(?! Score =).*\n)+/gm) {
  
  push(@HSPs, $&);

}
\end{lstlisting}

这和你前面看到的\verb|while|循环类型是一样的,都是全局字符串匹配,只要能找到匹配,它就会持续进行循环。另外的修饰符\verb|\m|是多行修饰符,它让元字符\verb|$|和\verb|^|可以匹配嵌入在内部的换行符之前和之后的位置。

第一对小括号中的表达式——\verb|(^ Score =.*\n)|——匹配以\verb|Score =|起始的一行,这正是引起HSP字符串匹配部分的行。

第二对小括号中的代码——\verb|(^(?! Score =).*\n)+|——匹配不以\verb|Score =|起始的一行或多行(因为在小括号外面紧跟着\verb|+|)。小括号括起来的部分的开头是\verb|?!|,它就是你在\autoref{exam:example12.2}中遇到的否定性前瞻断言。所以,总体来看,正则表达式会捕获以\verb|Score =|起始的行,以及随后的不以\verb|Score =|起始的相邻行。

\section{呈现数据}
直到现在为止,我们都还依赖于\textit{print}语句来格式化输出。在本节中,我将介绍打印输出的其他三个Perl特性:

\begin{itemize}
  \item \textit{printf}函数
  \item \textit{here}文档
  \item \textit{format}和\verb|write|函数
\end{itemize}

关于这些Perl输出特性的完整内容,已经超出了本书的范畴,但我还是会告诉你足够多的知识,让你对它们的用法有一个基本的了解。

\subsection{printf函数}
\textit{printf}函数和\textit{print}函数非常类似,但是有一些额外的特性,可以让你指定如何输出特定的数据。Perl的\textit{printf}函数是从C语言中同名的函数借鉴而来的。下面是\textit{printf}语句的一个例子:

\begin{lstlisting}
my $first  = '3.14159265';
my $second  = 76;
my $third = "Hello world!";

printf STDOUT "A float: %6.4f An integer: %-5d and a string: %s\n", 
     $first, $second,  $third;
\end{lstlisting}

这个代码片段会输出如下内容:

\begin{lstlisting}
A float:  3.1416 An integer: 76    and a string: Hello world!
\end{lstlisting}

\textit{printf}函数的参数包括一个格式字符串,后面紧跟着一个值列表,这些值会按照格式字符串指定的格式输出出来。格式字符串除了指定值列表输出格式的指令外,也可能会包括任意的文本。(你也可能会指定一个可选的文件句柄,它的使用方式和在\verb|print|函数中的使用方式是完全一样的。)

指令由一个百分号以及紧随其后的必需的转换指定符组成,刚才例子中的转换指定符有用于浮点数的\verb|f|、用于整数的\verb|d|和用于字符串的\verb|s|。转换指定符指明了变量中的数据该以何种类型输出出来。在\verb|%|和转换指定符之间,可能还会有0个或者多个标识,一个可选的最小字段宽度,一个可选的精度,以及一个可选的长度修饰符。格式字符串后面的值列表中的数据必须和指令中的类型一一对应。

对于这些标识和指定符(有一些在\autoref{chap:chapterab}中罗列了出来)来说,有许多可能的选项。下面是对\autoref{exam:example12.3}的解释。首先,指令\verb|%6.4f|指定要输出一个浮点数(也就是小数),总的最小宽度是六个字符(如果有必要就用空白填充),并且小数部分最多只能有四位。你看一下输出,尽管\verb|$f|这个浮点数给出的$\pi$的值已经到了小数点后八位,但是例子中指定的精度是小数点四位,结果中确实是按要求输出的。

指令\verb|%-5d|指定输出一个整数,其字段宽度为5;\verb|-|标识会让数字在字段中居左对齐。最后,指令\verb|%s|输出一个字符串。

\subsection{here文档}
现在,我们将简单来看一下\verb|here|文档。这是指定多行输出文本的比较方便的一种方法,在其中还可以嵌入一些变量用于变量内插,用这种方式的话,输出的格式就和你在代码中看到的格式几乎是完全一样的——也就是说,不需要大量的\verb|print|语句以及嵌入的换行符\verb|\n|字符。我们将用\autoref{exam:example12.3}及其输出来进行讨论。

%\textbf{例12-3:here文档示例}
\lstinputlisting[label=exam:example12.3,caption={例12-3:here文档示例}]{./scripts/example12-3.pl}

下面是\autoref{exam:example12.3}的输出:

\begin{lstlisting}
On iteration 0 of the loop!
AAACCCCCCGGGGGGGGTTTTTT

On iteration 1 of the loop!
AAACCCCCCGGGGGGGGTTTTTT
\end{lstlisting}

在\autoref{exam:example12.3}中,一个\verb|here|文档放在了\textit{for}循环中,这样你就可以在输出中看到\verb|$i|变量的变化了。就像在双引号字符串中对变量进行内插一样,在\verb|here|文档中也可以以同样的方式进行变量内插。每次进行循环的时候,\verb|here|文档的内容都会在变量内插后输出出来。终止字符串可以是你指定的任意字符串,在这个例子中是\verb|HEREDOC|。(处理缩进这样的事情有很多选项,在此处我不会进行讨论,你可以查阅Perl的文档。)\verb|here|文档对许多任务都非常顺手,比如,你有一个很长的多行文档,每次输出它的时候都只有很小的一点改动。典型的例子就是商业信函,其中只有地址需要改变。使用\verb|here|文档在最终的输出中保留了它在代码中看起来的样子,同时还允许变量内插。

\subsection{format和write}
最后,让我们来看一下\textit{format}和\textit{write}函数。\verb|format|被用来设计生成报表,它可以处理页码、页眉,以及居中、居左和居右对齐等各种排版选项。在格式化方面,它参考的是FORTRAN编程语言的规范,所以非常适合用来基于样式来生成报表,比如PDB文件格式,它行中的字段都被指定占用特定的列。

\autoref{exam:example12.4}是一个简短的例子,生成FASTA样式的输出格式。

%\textbf{例12-4:生出FASTA输出的format函数示例}
\lstinputlisting[label=exam:example12.4,caption={例12-4:生出FASTA输出的format函数示例}]{./scripts/example12-4.pl}

下面是\autoref{exam:example12.4}的输出:

\begin{lstlisting}
>A0000      Highly unlikely DNA.  This DNA is so...
AAAAAACCCCCCCCCCCCCCGGGGGGGGGGGGGGGGGGGGGGTTTTTTTTT
TTTTTTTTTTTT
\end{lstlisting}

在声明并初始化用于样式中的变量之后,样式是这样被定义的:

\begin{lstlisting}
format STDOUT =
\end{lstlisting}

格式会一直持续,直到以一个点号起始的行为止。

格式由三种类型的行组成:

\begin{itemize}
  \item 以英镑符号\verb|#|起始的注释
  \item 指定文本布局的图像行(picture line)
  \item 指明用于上一行图像行的变量的参数行
\end{itemize}

图像行和参数行必须紧邻在一起,比如,它们不能被一个注释行分隔开。

第一个图像行/参数行组合是用于头信息的:

\begin{lstlisting}
>@<<<<<<<<< @<<<<<<<<<<<<<<<<<<<<<<<<<<<<<<<<<<<...
$id,        $description
\end{lstlisting}

图像行中有两个图像字段,分别与\verb|$id|和\verb|$description|这两个变量关联对应。图像行以一个大于号\verb|>|起始,根据定义,这就是FASTA文件每个头信息行的起始字符。之后是第一个图像字段,也就是一个\verb|@|符号,后面跟着九个\verb|<|符号。\verb|@|符号声明一个字段,有一个与之关联对应的变量会按照其要求进行变量内插。使用九个小于号指定变量值必须居左对齐,其总长度为10列。如果值多于10列,它就会被截断。小于号表示居左对齐,大于号表示居右对齐,而竖线\verb=|=则会让数据在字段中居中对齐。

第二个图像字段几乎是一样的。它更长一些,并且以三个点号(一个省略号)结束,当变量\verb|$description|的内容超出图像字段指定的长度时,三个点号就会被输出出来(在这个例子中就是这样。)

接下来的图像/参数对是:

\begin{lstlisting}
^<<<<<<<<<<<<<<<<<<<<<<<<<<<<<<<<<<<<<<<<<<<<<<<<<~~
$DNA
\end{lstlisting}

图像字段以一个脱字符起始,它声明图像字段将要处理的是一个长度不定的记录。这一行也包含49个小于号,总共有50列,居左对齐。在末尾是两个波浪号\verb|~|,它表示如果无法把数据放进一行里,就是用额外的行。

\verb|write|命令只是简单地把前面定义的格式打印出来。默认情况下,会输出打印到STDOUT,在这个例子中就是这样,但是你你需要的话,可以为\verb|format|和\verb|write|语句指定一个文件句柄。

即将到来的Perl 6会把格式化的内容移出语言的核心代码,把它们放进一个模块中。在撰写本书的时候,还没有具体的细节,但这种变化可能意味着你要在接近代码顶端的地方加上一句\verb|use Formats;|这样的语句,来把模块载入进来用于格式处理。

\section{Bioperl}
\textit{Bioperl}是一个重要的Perl代码集合,专门用于生物信息学,这个项目从1998年开始一直在发展。尽管Bioperl使用了Perl语言设计中更加高级的面向对象的风格,此处还是可以对它进行一个简单的了解,知道它是如何组织的、如何去使用它。

Bioperl模块的主要关注点是进行序列处理,提供对各种生物学数据库(包括本地和基于网络的数据库)的访问,以及解析各种程序的输出。

在 \href{http://www.bioperl.org/}{http://www.bioperl.org/} 上可以找到Bioperl。它的一些特性依赖于已经安装的额外的Perl模块——可以从CPAN(\href{http://www.cpan.org/}{http://www.cpan.org/})上进行获取。这种情形非常常见,随着你进行更多的Perl编程,你会对从CPAN上获取安装模块越来越熟悉的。Bioperl指南包括在三大主流操作系统——Unix或Linux、Mac和Windows——上安装Bioperl和其他模块的介绍。

Bioperl并不提供完整的程序,而是提供一个巨大——还在持续增长——的模块集,可以用来完成常见的任务,包括你在本书中已经看到过的一些任务。你需要自己编写代码,来让这些模块进行协作。通过提供这些准备妥当并且(通常)易用的模块,Bioperl使得用Perl来开发生物信息学应用更加快捷、简便。对于大多数模块都有示例性的程序,你可以从查看并修改它们起步。

就像许多开源项目一样,Bioperl也有着片段化和文档不均一的问题,这要归因于有大量的志愿者参与以及贡献者在地理上过于分散。但是,最近这个项目的工作使得0.7版本在2001年三月发布\footnote{译者注:最新版本是2014年发布的1.6.924。},这显著改善了该项目。尤其是,现在已经有足够的模块使用指南信息,让你可以充分利用这些代码。

有些困难仍然存在。大部分代码都是在Unix或者Linux系统上开发的。并不是所有的都可以在Macs或者Windows操作系统上工作,但大部分还是可以的。在Bioperl网站上还有一些文档讨论在非Unix计算机上使用Bioperl,但底线是你可能会发现有些事情并不能正常工作。

如果你打算尝试一下Bioperl(我也强烈推荐你去尝试一下),你需要确保已经安装了最新版本的Perl。你至少需要版本5.004,如果从Perl网站 \href{http://www.perl.com}{http://www.perl.com} 上下载安装最新的稳定版本会更好一些。

\subsection{示例模块}
为了让你对Bioperl可以让哪些任务变得更加简单有一个了解,\autoref{tab:table12.1}展示了一个最有用的模块中的代表性模块。

\begin{table}[!htbp]
  \begin{center}
  \caption{Bioperl模块}
  \label{tab:table12.1}
  \begin{tabu}{X[2,l]X[3,l]}
  \toprule
  模块 & 描述\\
  \midrule
  Bio::Seq & 有特征的序列对象\\
  Bio::SimpleAlign & 最为一个序列集的多序列比对\\
  Bio::Species & 通用物种对象\\
  Bio::DB::Ace & 针对ACeDB服务器的数据库对象接口\\
  Bio::DB::GDB & 针对GDB HTTP查询的数据库对象接口\\
  Bio::DB::GenBank & GenBank的数据库对象接口\\
  Bio::DB::GenPept & GenPept的数据库对象接口\\
  Bio::DB::NCBIHelper & 查询NCBI数据库有用的常规程序集合\\
  Bio::DB::SwissProt & 针对SWISS-PROT检索的数据库对象接口\\
  Bio::Index::Fasta & 索引FASTA文件的接口\\
  Bio::Index::GenBank & 索引GenBank序列文件,也就是GenBank格式的平面文件,的接口\\
  Bio::Location::Simple & 对序列进行简单定位的实现\\
  Bio::Location::Split & 对有多个位置的序列进行定位的实现\\
  Bio::SeqFeature::FeaturePair & 处理成对的特征信息,比如,BLAST击中\\
  Bio::SeqFeature::Generic & 通用的SeqFeature\\
  Bio::SeqFeature::Similarity & 基于相似性的序列特征\\
  Bio::SeqFeature::SimilarityPair & 基于两条序列的相似性的序列特征\\
  Bio::SeqFeature::Gene::Exon & 表征一个外显子的特征\\
  Bio::SeqFeature::Gene::GeneStructure & 表征一个基因任意复杂结构的特征\\
  Bio::SeqFeature::Gene::Transcript & 表征一个转录本的特征\\
  Bio::SeqFeature::Gene::TranscriptI & 表征一个有外显子、启动子、UTR和poly(A)位点的转录本的特征接口\\
  Bio::Tools::Blast & Bioperl的BLAST序列分析对象\\
  Bio::Tools::BPbl2seq & 使用BLAST算法进行双序列比对的轻量级BLAST解析器\\
  Bio::Tools::BPlite & 轻量级的BLAST解析器\\
  Bio::Tools::BPpsilite & 用于PSIBLAST报告的轻量级BLAST解析器\\
  Bio::Tools::CodonTable & Bioperl的密码子表对象\\
  Bio::Tools::Fasta & Bioperl的FASTA实用对象\\
  Bio::Tools::IUPAC & 从一个含糊的序列对象生成多个唯一的序列对象\\
  Bio::Tools::RestrictionEnzyme & 用于限制性核酸内切酶对象的Bioperl对象\\
  Bio::Tools::SeqPattern & 用于序列模式或基序的Bioperl对象\\
  Bio::Tools::SeqStats & 处理单个特定序列统计信息的对象\\
  Bio::Tools::SeqWords & 处理一条序列的n-mer统计信息的对象\\
  Bio::Tools::Blast::HSP & Bioperl的BLAST高分片段对对象\\
  Bio::Tools::Blast::HTML & 用于HTML格式的BLAST报告的Bioperl实用模块\\
  Bio::Tools::Blast::Sbjct & Bioperl的BLAST“击中”对象\\
  Bio::Tools::Blast::Run::LocalBlast & 本地运行BLAST分析的Bioperl模块\\
  Bio::Tools::Blast::Run::Webblast & 使用HTTP接口运行BLAST分析的Bioperl模块\\
  Bio::Tools::Prediction::Exon & 预测的外显子特征\\
  Bio::Tools::Prediction::Gene & 预测的基因结构特征\\
  Bio::Variation::AAChange & 用于多肽的序列改变集\\
  Bio::Variation::AAReverseMutate & 单氨基酸改变的点突变和密码子信息\\
  Bio::Variation::Allele & 等位基因特异的属性的序列对象\\
  Bio::Variation::DNAMutation & DNA水平的突变集\\
  Bio::Variation::IO & 序列变异I/O格式的处理器\\
  \bottomrule
  \end{tabu}
  \end{center}
\end{table}

\subsection{Bioperl指南脚本}
Bioperl有一个指南脚本,帮助你尝试这个包各个部分的功能。在本节中,我将展示如何入手并运行一些示例性的计算程序。

我已经提到过,你应该学习如何从CPAN上下载代码,来安装Bioperl这样的模块。Perl编程环境现在之所以非常有用,很大程度上是因为在CPAN上有各种各样的模块可以使用。这是一个设计性的决策:把精力集中放在核心Perl语言上,Perl的设计者可以集中力量让这门语言尽可能的好。然后Perl模块的开发者可以把精力集中在他们各自的模块上。通过各种手段,在CPAN网站上好好浏览一番,看看哪些模块对你比较有用,对此有一个大概的了解。

此处,我不想对如何安装Bioperl进行详细的讲解:已经说过,可以在Bioperl网站上找到它,或者你可以访问CPAN网站来寻找一些信息。

所以,让我们假设你已经安装上了Bioperl模块,并且已经浏览了Bioperl网站上的指南。现在,让我们看一下如何来尝试一些Bioperl程序。

进入你计算机上Bioperl软件构建安装的目录。比如,在我的Linux计算机上,我把下载的\textit{bioperl-0.7.0.tar.gz}文件放到了\textit{/usr/local/src}目录中,然后使用下面的命令将其解压缩:

\begin{lstlisting}[language=bash]
tar xvzf bioperl-0.7.0.tar.gz
\end{lstlisting}

它会创建\textit{/usr/local/src/bioperl-0.7.0}这个源目录。在安装上这个模块之后(请查阅文档),你就可以运行指南脚本了。

切换到源目录中,键入\verb|perl bptutorial.pl|。下面是运行结果(我也把给出作者和版权信息的指南的头信息显示出来了):

\begin{lstlisting}[language=]
% head bptutorial.pl 
# $Id: ch12,v 1.44 2001/10/10 20:37:42 troutman Exp mam $

=head1  BioPerl Tutorial

  Cared for by Peter Schattner <schattner@alum.mit.edu>

  Copyright Peter Schattner

   This tutorial includes "snippets" of code and text from various
   Bioperl documents including module documentation, example scripts
% perl bptutorial.pl 

The following numeric arguments can be passed to run the corresponding demo-script.
1 => access_remote_db ,
2 => index_local_db ,
3 => fetch_local_db ,               (# NOTE: needs to be run with demo 2)
4 => sequence_manipulations ,
5 => seqstats_and_seqwords ,
6 => restriction_and_sigcleave ,
7 => other_seq_utilities ,
8 => run_standaloneblast ,
9 => blast_parser ,
10 => bplite_parsing ,
11 => hmmer_parsing ,
12 => run_clustalw_tcoffee ,
13 => run_psw_bl2seq ,
14 => simplealign_univaln ,
15 => gene_prediction_parsing ,
16 => sequence_annotation ,
17 => largeseqs ,
18 => liveseqs ,
19 => demo_variations ,
20 => demo_xml ,

In addition the argument "100" followed by the name of a single
bioperl object will display a list of all the public methods
available from that object and from what object they are inherited.

Using the parameter "0" will run all tests.
Using any other argument (or no argument) will run this display.

So typical command lines might be:
To run all demo scripts:
 > perl -w  bptutorial.pl 0
or to just run the local indexing demos:
 > perl -w  bptutorial.pl 2 3
or to list all the methods available for object Bio::Tools::SeqStats -
 > perl -w  bptutorial.pl 100 Bio::Tools::SeqStats

%
\end{lstlisting}

现在,让我们来试一下选项9——BLAST解析器和选项1——\verb|access_remote_db|。所以下面先从BLAST解析器开始:

\begin{lstlisting}[language=]
% perl bptutorial.pl 9

Beginning blast.pm parser example... 

QUERY NAME     : gi|1401126
QUERY DESC     : UNKNOWN
LENGTH         : 504
FILE           : t/blast.report
DATE           : Thu, 16 Apr 1998 18:56:18 -0400
PROGRAM        : TBLASTN
VERSION        : 2.0.4 [Feb-24-1998]</b>
DB-NAME        : Non-redundant GenBank+EMBL+DDBJ+PDB sequences
DB-RELEASE     : Apr 16, 1998  9:38 AM
DB-LETTERS     : 677679054
DB-SEQUENCES   : 336723
GAPPED         : YES
TOTAL HITS     : 100
CHECKED ALL    : YES
FILT FUNC      : NO
SIGNIF HITS    : 4
SIGNIF CUTOFF  : 1.0e-05 (EXPECT-VALUE)
LOWEST EXPECT  : 0.0
HIGHEST EXPECT : 1e-05
HIGHEST EXPECT : 7.6 (OVERALL)
MATRIX         : BLOSUM62
FILTER         : NONE
EXPECT         : 10
LAMBDA, K, H   : 0.270, 0.0470, 0.230 (SHARED STATS)
WORD SIZE      : 13
S              : 42, 74 (SHARED STATS)
GAP CREATION   : 11
GAP EXTENSION  : 1

Number of hits is 4 
Fraction identical for hit 1 is 0.25 
Sequence identities for hsp of hit 1 are 66-68 70 73 76 79 80 87-89 114 117
119 131 144 146 149 150 152 156 162 165 168 170 171 176 178-182 184 187 190
191 205-207 211 214 217 222 226 241 244 245 249 256 266-268 270 278 284 291
296 304 306 309 311 316 319 324 
%
\end{lstlisting}

这是解析BLAST输出的一种有趣的方式!现在,让我们再看一下访问远程数据库:

\begin{lstlisting}[language=]
% perl bptutorial.pl 1
Beginning remote database access example... 
seq1 display id is MUSIGHBA1 
seq2 display id is AF303112 
Display id of first sequence in stream is AF041456
% 
\end{lstlisting}

好吧,它就像一个输出一样,只有很少的信息,但是看上去你可以推断远程数据库的访问成功了。(补充一句,如果你失败了,可能是因为你在防火墙之后,它阻止了访问——这在大学或者大型公司中并不罕见。)

文档建议在Perl调试器下运行\textit{bptutorial.pl}脚本,这样可以一步步观察到底发生了什么。我非常赞同它的建议,当不会在此处把它的输出也展示出来。自己去试一下吧!

既然上一个例子并不是那么有趣,就让我们再尝试一个吧。下面是序列操作的指南:

\begin{lstlisting}[language=]
% perl bptutorial.pl 4

Beginning sequence_manipulations and SeqIO example... 
First sequence in fasta format... 
>Test1
AGCTTTTCATTCTGACTGCAACGGGCAATATGTCTCTGTGTGGATTAAAAAAAGAGTGTC
TGATAGCAGCTTCTGAACTGGTTACCTGCCGTGAGTAAATTAAAATTTTATTGACTTAGG
TCACTAAATACTTTAACCAATATAGGCATAGCGCACAGACAGATAAAAATTACAGAGTAC
ACAACATCCATGAAACGCATTAGCACCACC
Seq object display id is Test1
Sequence is AGCTTTTCATTCTGACTGCAACGGGCAATATGTCTCTGTGTGGATTAAAAAAAGAGTGTCTGATAG
CAGCTTCTGAACTGGTTACCTGCCGTGAGTAAATTAAAATTTTATTGACTTAGGTCACTAAATACTTTAACCAATATA
GGCATAGCGCACAGACAGATAAAAATTACAGAGTACACAACATCCATGAAACGCATTAGCACCACC 
Sequence from 5 to 10 is TTTCAT 
Acc num is unknown 
Moltype is dna 
Primary id is Test1 
Truncated Seq object sequence is TTTCAT 
Reverse complemented sequence 5 to 10  is GTGCTA  
Translated sequence 6 to 15 is LQRAICLCVD 

Beginning 3-frame and alternate codon translation example... 
ctgagaaaataa translated using method defaults   : LRK*
ctgagaaaataa translated as a coding region (CDS): MRK

Translating in all six frames:
 frame: 0 forward: LRK*
 frame: 0 reverse-complement: LFSQ
 frame: 1 forward: *ENX
 frame: 1 reverse-complement: YFLX
 frame: 2 forward: EKI
 frame: 2 reverse-complement: IFS
Translating with all codon tables using method defaults:
1 : LRK*
2 : L*K*
3 : TRK*
4 : LRK*
5 : LSK*
6 : LRKQ
9 : LSN*
10 : LRK*
11 : LRK*
12 : SRK*
13 : LGK*
14 : LSNY
15 : LRK*
16 : LRK*
21 : LSN*
% 
\end{lstlisting}

这更加有趣了,因为Bioperl的这一部分做了许多我们在本书中已经做过的事情。

我希望对于Bioperl的这个简短的浏览能够勾起你的好奇心。去探索一下这些模块集绝对是一个不错的想法。有一个解析BLAST输出的Perl模块,叫做\textit{BPLite.pm},它可能也比较有趣:现在它还不是Bioperl项目的一部分。

\section{练习题}
\textcolor{red}{\textit{习题12.1}}
\begin{adjustwidth}{2em}{}
\textit{基本的字符串匹配}。编写一个程序,在目标字符串中查找一个查询字符串。比如,如果查询字符串是“gone”,它会在目标字符串“goof through the way-gone-osphere”的22位置找到一个匹配。不要使用正则表达式或者任何Perl内置的字符串匹配工具;相反,在字符串中检查每一个单独的位置,比较字符,发明你自己的算法。
\end{adjustwidth}

\textcolor{red}{\textit{习题12.2}}
\begin{adjustwidth}{2em}{}
探索 \href{http://www.ncbi.nlm.nih.gov/BLAST}{http://www.ncbi.nlm.nih.gov/BLAST} 上的NCBI BLAST网页。熟悉BLAST各个组成程序的目的和使用,阅读指南信息理解统计结果的含义。
\end{adjustwidth}

\textcolor{red}{\textit{习题12.3}}
\begin{adjustwidth}{2em}{}
探索 \href{http://www.bioperl.org}{http://www.bioperl.org} 上的Bioperl网页。下载代码,并在你的计算机上安装它。
\end{adjustwidth}

\textcolor{red}{\textit{习题12.4}}
\begin{adjustwidth}{2em}{}
在NCBI网站上进行BLAST检索。先针对DNA数据库检索DNA,然后针对蛋白质数据库检索同样的DNA,最后比较它们的输出。
\end{adjustwidth}

\textcolor{red}{\textit{习题12.5}}
\begin{adjustwidth}{2em}{}
对相关的序列进行两次BLAST检索。解析检索的BLAST结果,对于每一个检索都提取出头注释中的前10个击中。编写一个程序,报告两次检索的异同之处。
\end{adjustwidth}

\textcolor{red}{\textit{习题12.6}}
\begin{adjustwidth}{2em}{}
编写一个程序,使用Bioperl来在NCBI网站上进行BLAST检索,然后使用Bioperl来解析BLAST的输出。
\end{adjustwidth}

\textcolor{red}{\textit{习题12.7}}
\begin{adjustwidth}{2em}{}
使用Bioperl模块,混合你自己的代码,编写一个程序,在一个DNA序列集上运行BLAST,对每个BLAST的击中列表进行排序,把排序后的IDs保存进数组。允许用户查看每一个列表、多个列表共有的击中,以及多个列表中每个独有的击中。对于每一个击中,可以让用户获取整个的GenBank记录。
\end{adjustwidth}

\textcolor{red}{\textit{习题12.8}}
\begin{adjustwidth}{2em}{}
对子程序\textit{extract\_HSP\_information}中的代码编写解释说明。一定要参考作为代码输入的数据的格式。
\end{adjustwidth}


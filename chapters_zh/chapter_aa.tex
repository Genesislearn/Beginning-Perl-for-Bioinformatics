\chapter{附录A 资源}
\label{chap:chapteraa}
\minitoc

对于Perl和生物信息学编程来说,有大量的相关资源与材料。此处并不对其进行穷举,但是它包含一些在线的资源和一些印刷版的资源,我认为在你拓展Perl编程技能的时候,你会发现这些资源比较有趣且有用。

\section{Perl}
Perl的文档非常详尽。它包括FAQs(常见问题集,附带解答)列表,指南,以Unix风格的man手册页形式整理的精确定义,以及特定领域的讨论。有大量的网站,一个叫做CPAN的组织良好的有用的Perl程序仓库,具有可检索档案的新闻组,会议,和许多好的书籍。非常值得花一定的时间去寻找并结交当地的Perl社团。不要害怕去叨扰你的同事,随着你编程技能的提升,他们也会慢慢开始向你进行咨询。

我前面已经提到过,Perl是免费的。它是更加庞大的开源运动的一部分,它包括Linux、Apache网络服务器等的开发。既然Perl是免费的,它就依赖于同道中人组成的一个社区团体来开发代码并撰写文档。正因为如此,你可能注意到了有不少文档都有点破碎(对某些来说简直是支离破碎)。尽管如此,这些项目的支持水平绝不亚于最好的商业软件包的支持程度。

\subsection{网站}
\href{http://www.perl.com}{http://www.perl.com}

\begin{adjustwidth}{1cm}{}
这是Perl所有内容的起点。不管怎么样,去看看吧。在这里,你会发现更多关于Perl编程各方各面的站点。在这其中,你可能会发现\href{http://www.perl.org}{http://www.perl.org}尤其有用。
\end{adjustwidth}

\subsection{CPAN(Comprehensive Perl Archive Network):Perl综合典藏网}
\href{http://www.cpan.org/}{http://www.cpan.org/}

\begin{adjustwidth}{1cm}{}
CPAN是一个非常重要的资源,也是寻找Perl模块的地方。此外,它还是其他软件、文档和网页链接的仓库。在花时间编写自己的程序之前,先到这里看看是不是已经有写好的程序了。
\end{adjustwidth}

\subsection{FAQs(Frequently Asked Questions):常见问答集}
\href{http://www.perl.com/pub/v/faqs}{http://www.perl.com/pub/v/faqs}

\begin{adjustwidth}{1cm}{}
FAQs是一个新手最常询问的问题的摘要,同时附带解答,这些解答通常都非常有用。作为一个程序员菜鸟,要想尽快上手,花点时间去读一下FAQs绝对是一个不错的选择,必要时可以进行跳读。
\end{adjustwidth}

你至少应该花费足够的时间来阅读FAQs,对哪些问题在FAQs中有对应的存档要有一个大概的了解。在向当地专家寻求帮助或者在新闻组中提问之前,一定要先去检查一下FAQs。重复询问那些在FAQs中已经进行了详尽解答的问题,通常会让人生厌,尤其是在Perl的新闻组中。

你会发现Perl的FAQs分成了几个部分。当查阅FAQs时,看看它们上次更新的日期。这对于Perl来说不算是个大问题,当通常来说,你在网上会找到许多过时的信息。

\subsubsection{初学者}
在FAQs和文档中有许多专门针对初学者的资料。除了本书以外,还有许多其他适用于初学者的书籍,在该附录的其他地方提到了。在\href{http://learn.perl.org}{http://learn.perl.org}(当我撰写本书时这还是一个比较新的站点,但看起来非常有前途)上也有一些关于Perl的在线指南和初学者文章。此外,还有一些邮件列表,你可以去订阅,包括叫做\href{mailto:beginners@perl.org}{beginners@perl.org}的邮件列表,通过访问\href{http://lists.perl.org}{http://lists.perl.org}你可以订阅它。

\subsection{在线手册}
\href{http://www.perl.com/pub/v/documentation}{http://www.perl.com/pub/v/documentation}

\begin{adjustwidth}{1cm}{}
Perl的手册是在线的,位于前面提到的Perl网站上。同时它也应该安装在了你的计算机上。通过键入\verb|perldoc perl|你可以访问它。在Unix/Linux系统上,你也可以键入\verb|man perl|来得到初始的man手册。如其所述,手册被分割成了几部分。比如,要找到Perl内置函数的手册,需要键入\verb|man perlfunc|或者\verb|perldoc|。也有HTML版本的手册,可以把它们安装在你本地的计算机上。这是我最喜欢的获取文档的方法,它会给你链接使得导航更加容易,并且如果它被安装在了本地上,甚至在没有联网的情况下都可以使用它。
\end{adjustwidth}

\subsection{书籍}
有许多Perl的书籍。其中不少都非常出色,但有些也不好。下面是一个简短的Perl书籍列表,我发现在我的工作中它们是最有用的。

\textit{Programming Perl, Third Edition},Larry Wall、Tom Christiansen和Jon Orwant著,O'Reilly \& Associates出版\footnote{《Perl语言编程》(第四版):\href{http://item.jd.com/11544992.html}{http://item.jd.com/11544992.html}。}。这是Perl语言发明人撰写的关于Perl的标准书籍。尽管它滞后于最新版本的Perl,但它非常好得解释了一切。所以你安装的绝对权威还是在线的手册。\textit{Programming Perl}涵盖了大量的内部细节,所以它更适合作为参考、指南,当你需要深入细节的内容时,可以把它作为绝妙的故事来看。它展示了语言背后的一些哲学,所以可以通过理解一些计算机科学的思维方式。如果你正好有早期的版本,也是完全可以的;我个人尤其喜欢它的第一版。

\textit{Perl Cookbook},Tom Christiansen和Nathan Torkington著,O'Reilly \& Associates出版。它被宣称为\textit{Programming Perl}的姊妹篇,确实如此。在这里,你会发现使用Perl来完成不同任务的实例。在许多情况下它都非常有用,如果你要进行许多的Perl编程,花费至少几个小时去研读它是非常值得的。

\textit{Mastering Algorithms with Perl},Jon Orwant、Jarkho Hietaniemi和John Macdonald著,O'Reilly \& Associates出版。我已经提到过学习算法的重要性,而该书就用Perl演示了许多重要的算法。它解释概念并给出代码,但它并不教授分析和测试算法的数学知识。真正严谨的学习算法的学生可以在Corman、Leiserson和Rivest著的\textit{Introduction to Algorithms}中找到相应的信息。即使你是一个程序员新手,这也是一本很有价值的书,你会找到许多你可以使用的代码。

\textit{Mastering Regular Expressions},Jeffrey R. Friedl著,O'Reilly \& Associates出版\footnote{《精通正则表达式》(第3版):\href{http://item.jd.com/11070361.html}{http://item.jd.com/11070361.html}。}。一本关于重要主题的好书,很好地涵盖了Perl。 

\textit{Elements of Programming in Perl},Andrew L. Johnson著,Manning Publications出版。这是针对初学者的另一本书。这本书非常好,我推荐把它作为本书的补充。

\textit{Learning Perl, Third Edition},Randall L. Schwartz和Tom Christiansen著,O'Reilly \& Associates出版\footnote{《Perl语言入门》(第6版):\href{http://item.jd.com/10972653.html}{http://item.jd.com/10972653.html}。}。这是Perl的经典入门指南书籍。它的编写和组织都非常好。如果你从头到尾学习了\textit{Beginning Perl for Bioinformatics},那你阅读\textit{Learning Perl}应该没有什么困难。

\textit{Object-Oriented Perl},Damian Conway著,Manning Publications出版。一本很棒的书,其中涵盖的主题对于程序员菜鸟和老鸟都能受益匪浅。

\subsection{会议}
\textit{O'Reilly开源大会(O'Reilly Open Source Convention)}。该大会现在包括一年一度的Perl会议。这是一个机会,可以参加各种各样的讨论和报告,结识形形色色的Perl实践者。此外还有常规的YAPC(yet another Perl conference)会议;你可以在Perl的主站点上找到它的详细信息。

\subsection{新闻组}
Per新闻组是程序员的一个重要资源。如果你从未看到过它们,那是因为它们通过网络(以及其他方式)进行交流。它们让你可以给网络上的一大组人写一个信息,可以针对成百上千个特定主题中的一个。如果你遇到了一个问题,在Perl文档和FAQs中都找不到解答方法,在新闻组中搜索针对这个问题的主题往往能得到答案。如果找不到现成的答案,你也可以在新闻组中发表一个问题,但这通常并不是必需的。

我想强调一下这个资源真的非常有用。弊端就是这通常倾向于“低信噪比”:换言之,在新闻组中常常有大量的无信息材料。但它还是值得一看的,即使是负面的回复(没有给出问题的已知解决方法)也会节省你的时间和精力。

在comp.lang.perl层级中有许多和Perl相关的新闻组。搜索引擎deja.com(最近卖给了google.com,但是仍然可以访问)\footnote{请使用:\href{https://groups.google.com}{https://groups.google.com}。}让你可以搜索这些新闻组的档案。对于特定新闻组的更多信息可以在Perl的FAQs中找到。比如,许多特定的Perl模块都有它们自己的新闻组、邮件列表或者网站。CPAN网站是另一个可以找到可检索新闻组档案的地方。

\section{计算机科学}
尽管你是通过编程编写生物学应用,你还是会发现自己常常一不留神就进入了传统计算机科学的世界。这里是一些已经发表的资源,可以帮助你找到自己的方向。

\subsection{算法}
\textit{Mastering Algorithms with Perl},Jon Orwant、Jarkho Hietaniemi和John Macdonald著,O'Reilly \& Associates出版。对于使用Perl编程的非计算机专业的科学家来说,这是最好的书籍。

\textit{Introduction to Algorithms},Thomas H. Cormen、Charles E.  Leiserson和Ronald L. Rivest著,MIT Press and McGraw-Hill出版\footnote{《算法导论》(原书第3版):\href{http://item.jd.com/11144230.html}{http://item.jd.com/11144230.html}。}。这绝对是关于算法的一本好书——从许多方面来说,这是最好的一本书。不管对于研究生还是大学生,它都是标准的大学教材之一(按理来说就是标准的教材)。不管是作为教材书籍还是作为参考书籍,它都完全能够胜任。它的目标读者是计算机科学专业的学生,所以里面涉及相当数量的数学知识,但即使是非数学的程序员来说,也会发现这本书非常有用。

\textit{Fundamentals of Algorithmics},Gilles Brassard和Paul Bratley著,Prentice Hall出版。对于算法技术的浅显概述。

\textit{Algorithms on Strings, Trees, and Sequences: Computer Science and Computational Biology},Dan Gusfield著,Cambridge University Press出版。这本书专注于字符串相关的算法,包括序列比对等主题。它非常详尽,但即使是这样,也不是面面俱到,因为这是一个一场庞大的领域。这是专门针对字符串算法的最后的资源,有大量关于生物学序列相似性的信息。

下面的书籍共进阶学习使用。

\textit{The Design and Analysis of Computer Algorithms},Alfred V.  Aho、John E. Hopcroft和Jeffrey D. Ullman著,Addison-Wesley出版。这是关于算法科学的经典书籍。

\textit{Introduction to Parallel Algorithms and Architectures: Arrays, Trees, Hypercubes},Frank Thomson Leighton著,Morgan Kaufmann出版。一个全面且严谨的教材和参考。

\textit{Randomized Algorithms},Rajeev Motwani和Prabhakar Raghavan著,Cambridge University Press出版\footnote{《随机算法 》:\href{http://item.jd.com/10000060.html}{http://item.jd.com/10000060.html}。}。一本清晰而严谨的书。

\subsection{软件工程} \textit{Software Engineering, Second Edition},Ian Sommerville著,Addison-Wesley出版\footnote{《软件工程》(原书第9版):\href{http://item.jd.com/10645053.html}{http://item.jd.com/10645053.html}。}。一本好的通用的书,它涵盖了重要的主题,同时避开了对立的竞争性理论的相关讨论。

\subsection{计算机科学理论}
\textit{Introduction to Automata Theory, Languages, and Computation, Second Edition},John E. Hopcroft、Rajeev Motwani和Jeffrey D.  Ullman著,Addison-Wesley出版\footnote{《自动机理论、语言和计算导论》(原书第3版):\href{http://item.jd.com/10058560.html}{http://item.jd.com/10058560.html}。}。关于计算机科学理论的经典教材。

\textit{Computers and Intractability: A Guide to the Theory of Np-Completeness},Michael R. Garey和David S. Johnson著,W.H. Freeman \& Co出版。关于这个主题的经典、超赞的一本书。

\subsection{通用编程}
\textit{The Unix Programmers Manual},Steven V.  Earhart等著,Harcourt、Brace和Jovanovich School出版。关于Unix(不管是那个版本的Unix)的这个手册,是计算机科学中重点针对编程的速成课。交互式程序的设计,以及管道、重定向、进程的概念,等等都已经称为编程中巨大成功的范例之一。该手册对系统进行了概述:第一部分描述用户程序;第二部分和第三部分描述编程界面。可编程的shell,以及grep、awk和sed程序是Perl的主要灵感。

\textit{The C Programming Language},Brian W. Kernighan和Dennis M.  Ritchie著,Prentice Hall PTR出版\footnote{《C程序设计语言》(第2版):\href{http://item.jd.com/10057446.html}{http://item.jd.com/10057446.html}。}。C和C++是生物信息学中的重要编程语言,而这本经典的书籍教授的是C。如果你研读了全书,并且尝试了所有的编程练习,那么你已经有了良好的编程训练。

\textit{Structure and Interpretation of Computer Programs},Harold Abelson、Gerald Jay Sussman和Juke Sussman著,MIT Press出版\footnote{《计算机程序的构造和解释》(原书第2版):\href{http://item.jd.com/10057478.html}{http://item.jd.com/10057478.html}。}。一本真的非常有趣的书籍,在学习Lisp方言的过程中对编程进行了深入讲解。

\textit{The Unix Programming Environment},Brian W. Kernighan和Robert Pike著,Prentice Hall出版\footnote{《UNIX编程环境》:\href{http://item.jd.com/11423589.html}{http://item.jd.com/11423589.html}。}。这本书非常有趣,并且它讨论了好的软件设计。

\section{Linux}
如果你有一个Linux操作系统,你就有了整个系统的源代码(对于一些Unix系统来说也是这样)。(如果它没有被安装,你可以从发行版CD中得到它,从网站\href{http://www.linux.org}{http://www.linux.org}或者制造你使用的Linux版本的公司的网站上得到发行版CD。)这是一个巨大的资源。你可以看看任何一个程序,甚至是操作系统,是如何被编写的。现在你真的进入编程的世界了。

\section{生物信息学}
生物信息学是一个相对较新的学科,吸引了众多的目光,所以可用的资源也在飞速增长。下面是一些帮助你入门的书籍和其他资源。

\subsection{书籍}
\textit{Developing Bioinformatics Computer Skills},by Cynthia Gibas and Per Jambeck著,O'Reilly \& Associates出版\footnote{《生物信息学中的计算机技术》。}。对于初学者来说,这是相当好的一本书。它涵盖了Linux工作站的构建,以及许多优秀且廉价的生物信息学程序的安装与使用。它教授的是如何去使用生物信息学程序,而非如何去编程。它是现有的最实用的一本生物信息学书籍。

\textit{Introduction to Computational Biology: Maps, Sequences and Genomes},Michael S. Waterman著,CRC Press出版\footnote{《计算生物学导论:图谱、序列和基因组》:\href{http://item.jd.com/10005674.html}{http://item.jd.com/10005674.html}。}。这是一本经典的书籍,主要从统计学的角度进行讲解。

\textit{Bioinformatics: A Practical Guide to the Analysis of Genes and Proteins, Second Edition}Andreas D. Baxecvanis和B.F. Francis Ouellette编著,John Wiley \& Sons出版。包括由众多作者编写的多个章节,涉及比较广泛的主题。

\subsection{政府组织}
最基本的东西。下面这些网站是最重要的由政府资助的生物信息学组织。

\href{http://www.ncbi.nlm.nih.gov/}{http://www.ncbi.nlm.nih.gov/}:

\begin{adjustwidth}{1cm}{}
NCBI(National Center for Biotechnology Information):国家生物技术信息中心,美国政府中心。
\end{adjustwidth}

\href{http://www.embl.org/}{http://www.embl.org/}:

\begin{adjustwidth}{1cm}{}
EMBL(European Molecular Biology Laboratory):欧洲分子生物学实验室,欧洲联合实验室。
\end{adjustwidth}

\href{http://www.ebi.ac.uk/}{http://www.ebi.ac.uk/}:

\begin{adjustwidth}{1cm}{}
EMBL中的EBI(European Bioinformatics Institute):欧洲生物信息研究所。
\end{adjustwidth}

\subsection{会议}
生物信息学一直是各种生物学会议的一部分,比如,关于测序的冷泉港会议(Cold Spring Harbor conferences)。现在有许多涉及该方面的会议,通常都被冠以“基因组学”。下面是几个有趣的会议:

\begin{itemize}
  \item \textit{ISMB(Intelligent Systems for Molecular Biology):国际分子生物学智能系统会议},现在是它的第九个年头\footnote{译者注:指的是2001年。一年一届,2015年举办的是第23届。}
  \item \textit{生物信息学开放源代码会议(Bioinformatics Open Source Conference)},\href{http://www.bioinformatics.org/}{http://www.bioinformatics.org/}
  \item \textit{RECOMB(Conference on Computational Molecular Biology)}:计算分子生物学会议
\end{itemize}

\section{分子生物学}
\textit{Recombinant DNA},James Watson等著,W.H. Freeman \& Co出版。相对于这么一个快速发展的领域,该书显得有点陈旧了,但它还是一本引领程序员和计算机科学家进入生物信息学领域的佳作。许多标准的技术都用精彩的插图进行了清晰简洁的解释。去找一下它的第二版,看看能不能找到。

\textit{Molecular Biology of the Gene, Fourth Edition},James Watson等著,Addison-Wesley出版\footnote{《基因的分子生物学》(第七版):\href{http://item.jd.com/11672603.html}{http://item.jd.com/11672603.html}。}。分子生物学的经典书籍。它非常详尽,从知识面的覆盖来说,它确实有些过时了,但仍不失为经典之作。对于基础知识可以好好参考该书。

\textit{Molecular Cell Biology, Fourth Edition},Harvey Lodish等著,W.H.  Freeman \& Co出版。关于细胞生物学的优秀且宽泛的介绍性概述。

\textit{《Lewin基因X(中文版)》\footnote{译者补充。}:\href{http://item.jd.com/11159665.html}{http://item.jd.com/11159665.html}}。该书对分子生物学和分子遗传学进行了精彩的论述,内容涵盖了基因的结构、序列、组织和表达,是分子生物学和分子遗传学最经典的名著之一。

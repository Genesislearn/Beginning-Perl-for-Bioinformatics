\chapter{突变和随机化}
\label{chap:chapter7}
\minitoc

正如每一个生物学家所知道的那样,突变是生物学中的一个基本主题。在细胞中,DNA上的突变时时刻刻都在发生着。绝大多数突变都不影响蛋白质行使功能,是良性的。也有一部分突变确实会影响到蛋白质,导致肿瘤等疾病的发生。突变也会造成后代无法存活,它们在发育过程中就会死亡;有时,突变也能导致进化的改变。许多细胞都有很复杂的机制,来对突变进行修复。

DNA的突变可能来源于辐射、化学制剂、复制错误等原因。我们将使用Perl的随机数生成器,把突变看成随机化事件来对其进行建模。

随机化是一种计算机技术,它经常会在日常使用的密码等程序中突然出现,比如你想生成一个不容易被猜到的密码。但随机化也是算法中的一个重要分支:许多最快的算法都用到了随机化。

使用随机化,可以来模拟和研究DNA突变的机制,以及突变对相关蛋白质生物活性的影响。模拟是研究系统和预测结果的一个强有力的工具,随机化让你可以更好地模拟生物系统中的“有序混沌”。使用计算机程序来模拟突变,这将有助于进化、疾病以及分裂和DNA修复机制等基本细胞过程的研究。细胞发育和功能的计算机模型,现在还在它们的早起阶段,在未来的几年中它将会更加精确且有用,而突变就是这些模型将要囊括在内的一个基本的生物学机制。

从编程技术以及对进化、突变和疾病建模的立场来看,随机化是一个强大的编程技巧,而幸运的是,它非常容易使用。

在本章中,我们将要完成以下内容:

\begin{itemize}
  \item 在数组中随机选取一个索引,在字符串中随机选取一个位置:这些是在DNA(或其他数据)中选取随机位置的基本工具
  \item 使用随机数对突变进行建模,学习如何随机选取DNA中的一个核苷酸并把它突变成其他(随机)的核苷酸
  \item 使用随机数来生成DNA序列数据集,这可以用来研究实际基因组的随机化程度
  \item 重复突变DNA来研究在进化过程中突变随时间累积的影响
\end{itemize}

\section{随机数生成器}
\textit{随机数生成器}是你可以调用的一个子程序。对于大多数实践操作来说,你不需要知道它里面是什么。你从计算机中得到的随机数数值,和真实世界中测量到的随机事件有一定的差别,比如,检测的核衰变事件。有些计算机确实连接着盖革计数器等设备,这样就可以有一个真实随机事件的来源了。但我敢打赌,你的计算机上并没有这样的设备。你有的只是一个代替盖革计数器的算法,它就是随机数生成器。

随机数生成器输出的数字并不是真正随机的,因此它们被叫做\textit{伪随机数}。一个随机数生成器,作为一种算法,是可以被预测的。随机数生成器需要一个种子作为输入,通过改变种子,你可以得到一系列不同的(伪)随机数。

随机数生成器生成的结果给出的是数值的均匀分布,这是随机化最重要的特性之一,并且很大程度上也决定了要根据期望的随机范围的大小来调整算法的使用。

对于随机数生成器来说,另一个要牢记在心的就是你初始化使用的种子本身也应该是随机选择的。如果你每次都使用同样的数字作为种子,那么每次你都将得到同样的“随机数字”序列。(这就并不随机了!)试着选一个具有随机性的种子,比如某些随时间任意改变的计算机事件计算出来的数字。\footnote{即使这样,对于紧要的用途来说,你还是没有跳出如来佛的五指山。除非你小心地选择种子,否则黑客还是可以猜出你是如何选择种子,从而破解你的随机数和密码。本章中使用的生成种子的方法, \verb=time|$$=,是可以被黑客中的“有志青年”所破解的。一个更好的选择是 \verb=time() ^ ($$+<<15))=。如果程序安全非常重要的,你就应该好好查阅Perl的文档,以及CPAN中的\textit{Math::Random}和\textit{Math::TrulyRandom}模块。}

在接下来的例子中,我使用一个简单的方法来挑选种子,这对于大多数用途来说都是没有问题的。如果你使用随机数对存在紧要的隐私问题的数据(比如病人的记录)进行加密,你应该进一步参阅Perl中关于Perl提供给随机数生成器的几个高级选项的文档。在本书中,我使用的方法对于大多数情况的用途来说都已经足够了。

\section{使用随机化的一个程序}
\autoref{exam:example7.1}通过一个简单的程序介绍了随机化,它通过随机组合句子的片段来构造一个故事。这并不是一个生物信息学的程序,但是我发现这是学习随机化基础知识的一个有效的方法。你将学习如何从数组中随机选取一个元素,这会在后续突变DNA的程序实例中得到运用。

例子声明了几个包含句子片段的数组,然后把它们随机组合成完整的句子。这是一个微不足道的孩子的游戏,但它演示了一些编程要点。

%\textbf{例7-1:使用随机数的儿童游戏}
\lstinputlisting[label=exam:example7.1,caption={例7.1:使用随机数的儿童游戏}]{./scripts/example7-1.pl}

下面是\autoref{exam:example7.1}一些典型的输出:

\begin{lstlisting}[breaklines=true]
Joe and Moe jumped with Rebecca in New York City. Rebecca exploded Groucho in a dream. Mom ran to Harpo over the rainbow. TV giggled with Joe and Moe over the rainbow. Harpo exploded Joe and Moe at the beach. Robin Hood giggled with Harpo at the beach. 

Type "quit" to quit, or press Enter to continue: 

Harpo put hot sauce into the orange juice of TV before dinner. Dad ran to Groucho in a dream. Joe and Moe put hot sauce into the orange juice of TV in New York City. Joe and Moe giggled with Joe and Moe over the rainbow. TV put hot sauce into the orange juice of Mom just for the fun of it. Robin Hood ran to Robin Hood at the beach. 

Type "quit" to quit, or press Enter to continue: quit
\end{lstlisting}

例子的结构非常简单。使用以下语句强制对变量进行声明并开启警告模式:

\begin{lstlisting}
use strict;
use warnings;
\end{lstlisting}

之后,对变量进行声明,并使用值对数组进行初始化。

\subsection{为随机数生成器设置种子} 
接下来,通过调用内置函数 \verb|srand|,为随机数生成器设置种子。它需要一个参数,就是前面讨论的随机数生成器的种子。如前所述,为了得到一系列不同的随机数,你必须使用不同的种子。尝试把它改成像这样的语句:

\begin{lstlisting}
srand(100);
\end{lstlisting}

然后,多次运行该程序。每次,你都会得到完全相同的结果。\footnote{最新的随机数生成器会自动更改随机数序列,所以如果该实验不成功,很有可能你在使用一个非常新的随机数生成器。然而,有时你想重复一个随机数序列。注意,如果你像 \verb|srand;|这样调用 \verb|srand|,更新版本的Perl会自动给你一个好的种子。}你使用的种子:

\begin{lstlisting}
time|$$ 
\end{lstlisting}

每次都会计算返回不同的种子。

\textit{time}返回代表时间的一个数,\textit{\$\$}返回代表运行的Perl程序的ID(每次你运行程序它都会改变)的一个数,而|表示位元的或运算,它把两个数的位组合起来(更多细节请参看Perl文档)。还有选取种子的其他方法,但就让我们使用最流行的这种方法吧。

\subsection{控制流}
程序中的主循环是 \verb|do-until|循环。当你想在每次循环中采取任何行动(比如询问用户是否要继续)之前就做一些事情(比如打印出一个小故事)时,这种循环就非常方便了。 \verb|do-until|循环首先执行代码块中的语句,然后进行测试,决定它是否应该重复执行代码块中的语句。注意,这和你以前见过的其他类型的循环正好相反,它们是先进行测试后执行代码块。

既然总是向 \verb|$story|变量上附加内容,那就需要在每次循环的开始先把它清空。忘记需要在特定的地方把以某种形式递增的变量进行重置,这非常常见,所以在你编程时一定要留意这一点。线索就是不断增长的长字符串或大数字。

 \verb|for|循环包含着程序的主要工作。就像你前面看到的那样,这个循环初始化了一个计数器,执行测试,并在代码块的最后对计数器进行递增。

\subsection{造句}
在\autoref{exam:example7.1}中,注意造句用的语句蔓延了数行代码。这有一点点复杂,而这正是整个程序的真正内容,所以附加了一些注释帮助理解它。注意语句进行了精心的格式化,这样它就整洁德排列在了八行中。变量名也是进行选择的,这样这个过程就清晰了许多——你使用一个名词、一个动词、一个名词和一个介词短语进行造句。

然而,即使是这样,在中括号中也有多层嵌套的表达式,它用来指定数组的位置,要理解这些代码你需要进行一点仔细的分析。你会看到,你用以空格分隔的句子片段构造了一个字符串,并用一个句点和空格将其结束。字符串是通过多次使用点字符串连接操作符构建出来的,这些点字符串连接操作符被放在了每一行的开头,这样就使得整个语句的结构清晰了许多。

\subsection{随机选取数组的一个元素}
让我们仔细看看其中一个语句成分选择器吧:

\begin{lstlisting}
$verbs[int(rand(scalar @verbs))] 
\end{lstlisting}

对于这种多层嵌套的括号,要由内向外进行阅读和计算。所以包裹在括号最内层的表达式是:

\begin{lstlisting}
scalar @verbs
\end{lstlisting}

从语句前面的注释中你可以看到,内置函数\textit{scalar}返回数组元素的个数。例子中的数组 \verb|@verbs|有七个元素,所以这个表达式返回7。

所以现在你得到的是:

\begin{lstlisting}
$verbs[int(rand(7))]
\end{lstlisting}

而嵌套在最内层的表达式就成了:

\begin{lstlisting}
rand(7)
\end{lstlisting}

代码中语句前帮助性的注释提醒你,这个语句返回一个大于0、小于7的(伪)随机数。这个数是一个\textit{浮点数}(有一个分数的十进制数)。回想一下,一个有七个元素的数组的元素索引是从0到6。

所以现在你得到的类似于:

\begin{lstlisting}
$verbs[int(3.47429)] 
\end{lstlisting}

而你想要对这个表达式进行计算:

\begin{lstlisting}
int(3.47429) 
\end{lstlisting}

\textit{int}函数会丢掉浮点数的小数部分,仅返回它的整数部分,在这个例子中就是3。

所以你来到了最后的一步:

\begin{lstlisting}
$verbs[3]
\end{lstlisting}

这会给你 \verb|@verbs|数组的第四个元素,注释中已经给你了足够的提示。

\subsection{格式化}
为了随机选取一个动词,你调用了几个函数:

\begin{description}
  \item[\textcolor{red}{\textit{scalar}}] 确定数组的大小
  \item[\textcolor{red}{\textit{rand}}] 从数组大小决定的范围内选取一个随机数字
  \item[\textcolor{red}{\textit{int}}] 变换浮点数,\textit{rand}返回用于数组元素的整数值
\end{description}

使用嵌套的括号,这些函数调用被组合到了一行中。有时,这会生成难于阅读的代码,对于作者这种辛苦工作得来的成果,某些吹毛求疵的读者可能会直喊头疼,因为它们并不讨人喜欢。你可以使用一些额外的临时变量,试着重写这些行的代码。比如,你可以这样写:

\begin{lstlisting}
$verb_array_size = scalar @verbs;
$random_floating_point = rand ( $verb_array_size );
$random_integer = int $random_floating_point;
$verb = $verbs[$random_integer];
\end{lstlisting}

并且,对其他造句部分也进行类似的改写,最后通过这样的语句你就可以构造出句子了:

\begin{lstlisting}
$sentence = "$subject $verb $object $prepositional_phrase. ";
\end{lstlisting}

这是风格的问题。当你编程时,你总是会进行类似的抉择。\autoref{exam:example7.1}中的排版风格是基于这样的得失权衡:既要把整个任务表达清晰(得),又要避免难于阅读的高度嵌套的函数调用(失)。使用这种排版的另一个原因就是,在后面的程序中,你经常需要从数组中随机选取一个元素,所以你将对这种特殊的函数调用的嵌套习以为常。事实上,如果你将要多次重复同样的事情,你可能会对这样的调用编写一个小的子程序。

就像在大多数代码中一样,易读性是这里最重要的因素。你必须要能够阅读和理解代码,不管是你自己的还是别人的代码,这通常都要比实现其他的动人的目标重要,比如最快的速度、使用最少的内存以及最简练的程序。它并不总是很重要,但通常来说最好先把它写的易读一些,之后如果需要的话再返回去尝试提高其速度(或者其他)。你甚至可以把更加易读的代码作为注释写在那儿,这样阅读代码的人就能够对程序以及你是如何提高程序速度(或者其他)的有一个清晰的理解。

\subsection{计算随机位置的另一种方法}
Perl通常有不同的方法来完成同一个任务。下面就是编写这个随机数选择的另一种方法;它使用的同样的函数调用,但是没有使用小括号:

\begin{lstlisting}
$verbs[int rand scalar @verbs]
\end{lstlisting}

这种函数链在Perl中很常见,其中的每一个函数都需要一个参数。要计算表达式,Perl首先把 \verb|@verbs|作为\textit{scalar}的参数,这会返回数组的大小。然后它把得到的值作为 \verb|rand|的参数,这会返回一个大于 \verb|0|、小于数组大小的浮点数。然后,它把浮点数作为\textit{int}的参数,这会返回小于浮点数的最大整数。换句话说,它计算的数字和用于数组 \verb|@verbs|的下标是完全一样的。

为什么Perl允许这样呢?因为这样的计算非常频繁,并且,根据”让计算机干活“的精神,Perl的设计者拉里$\textbullet$沃尔决定让你(以及他自己)免于键入这些括号并使其配对的烦恼。

已经走了这么远了,拉里决定为了简单还要再进一步。你可以省略\textit{scalar}和\textit{int}函数的调用,直接使用:

\begin{lstlisting}
$verbs[rand @verbs]
\end{lstlisting}

这里发生了什么?既然\textit{rand}已经期望一个标量值,它就会把 \verb|@verbs|放在一个标量上下文中,也就是简单的返回数组的大小。拉里聪明的设计了数组的下标(当然,它总是整数值),这样当需要下标时,会自动提取浮点数值的整数部分,所以就不需要\textit{int}了。

\section{模拟DNA突变的程序}
\autoref{exam:example7.1}给你了突变DNA时需要的工具。在接下来的例子中,你将照常使用由字母A、C、G和T构成的字符串来表示DNA。你将在字符串中随机选取位置,然后使用\textit{substr}函数来改变DNA。

这次,让我们换一种思路,在给出整个程序之前,首先来编写一些将要使用到的子程序。

\subsection{伪代码设计}
从简单的伪代码开始,这是把DNA中一个随机位置突变成一个随机核苷酸的子程序的设计:

\begin{enumerate}
  \item 选取DNA字符串中一个随机的位置。
  \item 选择一个随机的核苷酸。
  \item 把DNA随机位置替换成随机的核苷酸。
\end{enumerate}

这看上去简洁且直指要害,所以你决定把前两句分别写成子程序。
\subsubsection{在字符串中选取一个随机位置}
怎样才能在一个字符串中随机选取一个位置呢?回忆一下,内置函数\textit{length}返回的就是字符串的长度,此外,字符串中的位置是从 \verb|0|到 \verb|length-1|进行编号的,就像数组中的位置一样。所以你可以使用和\autoref{exam:example7.1}一样的通用策略,编写成子程序:

\begin{lstlisting}
# randomposition
#
# A subroutine to randomly select a position in a string.
#
# WARNING: make sure you call srand to seed the
#  random number generator before you call this function.

sub randomposition {

  my($string) = @_;

  # This expression returns a random number between 0 and length-1,
  # which is how the positions in a string are numbered in Perl.

  return int(rand(length($string)));
}
\end{lstlisting}

如果不计算注释的话,\textit{randomposition}实际上是一个简短的函数。这和\autoref{exam:example7.1}中选取一个随机的数组元素的想法是一样的。

当然,如果你亲自编写这段代码,需要进行一点测试,来看看这个子程序能不能工作:

\begin{lstlisting}
#!/usr/bin/perl -w
# Test the randomposition subroutine

my $dna = 'AACCGTTAATGGGCATCGATGCTATGCGAGCT';

srand(time|$$);

for (my $i=0 ; $i < 20 ; ++$i ) {
  print randomposition($dna), " ";
}

print "\n";

exit;

sub randomposition {
  my($string) = @_;
  return int rand length $string;
}
\end{lstlisting}

下面是测试的一些典型输出(你的结果可能会有所不同):

\begin{lstlisting}
28 26 20 1 29 7 1 27 2 24 8 1 23 7 13 14 2 12 13 27 
\end{lstlisting}

注意 \verb|for|循环的新的写法:

\begin{lstlisting}
for (my $i=0 ; $i < 20 ; ++$i ) {
\end{lstlisting}

这里演示了你可以在 \verb|for|循环中使用 \verb|my|对计数器变量(在这个例子中就是 \verb|$i|)进行声明,从而使其进入循环。

\subsubsection{选择一个随机的核苷酸}
接下来,让我们写一个子程序,从四个核苷酸中随机选取一个:

\begin{lstlisting}
# randomnucleotide
#
# A subroutine to randomly select a nucleotide
#
# WARNING: make sure you call srand to seed the
#  random number generator before you call this function.

sub randomnucleotide {

  my(@nucs) = @_;

  # scalar returns the size of an array. 
  # The elements of the array are numbered 0 to size-1
  return $nucs[rand @nucs];
}
\end{lstlisting}

又一次,这个子程序简洁、悦目。(大多数有用的子程序都是这样的,尽管编写一个简短的子程序并不保证它是有用的。事实上,你将会看到还可以对这个子程序进行一点改进。)

让我们也对它进行以下测试:

\begin{lstlisting}
#!/usr/bin/perl -w
# Test the randomnucleotide subroutine

my @nucleotides = ('A', 'C', 'G', 'T');

srand(time|$$);

for (my $i=0 ; $i < 20 ; ++$i ) {
  print randomnucleotide(@nucleotides), " ";
}

print "\n";

exit;

sub randomnucleotide {
  my(@nucs) = @_;

  return $nucs[rand @nucs];
}
\end{lstlisting}

下面是一些典型的输出(它是随机的,所以理所当然,有很大的可能性你的输出会与此不同):

\begin{lstlisting}
C A A A A T T T T T A C A C T A A G G G 
\end{lstlisting}

\subsubsection{把随机的核苷酸放到随机的位置}
现在轮到第三个、也是最后一个子程序了,它要进行实际的突变。下面是代码:

\begin{lstlisting}
# mutate
#
# A subroutine to perform a mutation in a string of DNA
#

sub mutate {

  my($dna) = @_;
  my(@nucleotides) = ('A', 'C', 'G', 'T');

  # Pick a random position in the DNA
  my($position) = randomposition($dna);

  # Pick a random nucleotide
  my($newbase) = randomnucleotide(@nucleotides);

  # Insert the random nucleotide into the random position in the DNA.
  # The substr arguments mean the following:
  #  In the string $dna at position $position change 1 character to
  #  the string in $newbase
  substr($dna,$position,1,$newbase);

  return $dna;
}
\end{lstlisting}

这里还是一个简短的程序。当你查看它时,会发现它阅读、理解起来都相对比较容易。通过选取一个随机的位置、随机的核苷酸,并把字符串中那个位置的核苷酸替换成那个随机的核苷酸,你实现了突变。(如果你忘记了\textit{substr}是如何使用的,可以参看\autoref{chap:chapterab}或者其他的Perl文档。如果你像我一样,你可能不得不多次查阅文档,尤其是要确保以正确的顺序使用参数。)

这里使用的声明变量的风格有点不同。以前是在程序的开头声明变量,而这里则是在第一次使用变量的时候才对它们分别进行声明。每种编程风格都各有利弊。把所有变量放在程序的顶部是一种很好的组织形式,而且对阅读代码也有所帮助;而随用随声明在编写程序时看起来则是一种更加自然的方式。选择权在你手中。

此外,注意这个子程序的大部分是如何基于其他的子程序构建起来的,仅仅需要添加一点代码。这使得代码的易读性大大提高。此时,你可能会觉得你已经把任务进行了很好的分解,并且每一个小部分都比较容易完成,而最后它们也能很好的在一起协作。但真是这样的吗?

\subsection{改进设计}
你可能会对自己这么快就编写完程序而颇感自豪,但你注意到了一些事情。你总是要声明讨厌的 \verb|@nucleotides|数组变量,然后把它传递给\textit{randomnucleotide}子程序。但你使用这个数组的唯一地方只在\textit{randomnucleotide}子程序的内部。所以为什么不把设计改变一下呢?下面是一个新的尝试:

\begin{lstlisting}
# randomnucleotide
#
# A subroutine to randomly select a nucleotide
#
# WARNING: make sure you call srand to seed the
#  random number generator before you call this function.

sub randomnucleotide {
  my(@nucs) = ('A', 'C', 'G', 'T');

  # scalar returns the size of an array. 
  # The elements of the array are numbered 0 to size-1
  return $nucs[rand @nucs];
}
\end{lstlisting}

注意这个函数现在没有参数了,要像这样调用它:

\begin{lstlisting}
$randomnucleotide = randomnucleotide( );
\end{lstlisting}

它从一个特定的数据集中选取一个随机的元素。当然,你总是在思考,并且会说“要是有一个从任意数组中随机选取一个元素的子程序该会多方便呀。我可能现在并不需要它,但我敢打赌很快我就会需要这样的子程序!”所以,你定义了两个子程序,而不是一个:

\begin{lstlisting}
# randomnucleotide
#
# A subroutine to randomly select a nucleotide
#
# WARNING: make sure you call srand to seed the
#  random number generator before you call this function.

sub randomnucleotide {
  my(@nucleotides) = ('A', 'C', 'G', 'T');

  # scalar returns the size of an array. 
  # The elements of the array are numbered 0 to size-1
  return randomelement(@nucleotides);
}

# randomelement
#
# A subroutine to randomly select an element from an array
#
# WARNING: make sure you call srand to seed the
#  random number generator before you call this function.

sub randomelement {

  my(@array) = @_;

  return $array[rand @array];
}
\end{lstlisting}

回头看一下,你会注意到并不需要更改\textit{mutate}子程序,改变的只是\textit{randomnucleotide}的内部构造,而不是它的行为。

\subsection{组合子程序来模拟突变}
现在,所有的材料都到位了,所以你要编写如\autoref{exam:example7.2}一样的主程序,来看看新的子程序是否工作。

%\textbf{例7-2:突变DNA}
\lstinputlisting[label=exam:example7.2,caption={例7.2:突变DNA}]{./scripts/example7-2.pl}

下面是\autoref{exam:example7.2}的一些典型的输出:

\begin{lstlisting}
Mutate DNA

Here is the original DNA:

AAAAAAAAAAAAAAAAAAAAAAAAAAAAAA

Here is the mutant DNA:

AAAAAAAAAAAAAAAAAAAAGAAAAAAAAA

Here are 10 more successive mutations:

AAAAAAAAAAAAAAAAAAAAGACAAAAAAA
AAAAAAAAAAAAAAAAAAAAGACAAAAAAA
AAAAAAAAAAAAAAAAAAAAGACAAAAAAA
AAAAAAAAAAAAAACAAAAAGACAAAAAAA
AAAAAAAAAAAAAACAACAAGACAAAAAAA
AAAAAAAAAAAAAACAACAAGACAAAAAAA
AAAAAAAAAGAAAACAACAAGACAAAAAAA
AAAAAATAAGAAAACAACAAGACAAAAAAA
AAAAAATAAGAAAACAACAAGACAAAAAAA
AAAAAATTAGAAAACAACAAGACAAAAAAA
\end{lstlisting}

\autoref{exam:example7.2}在编程上有一定的挑战,但你最终看到(模拟的)DNA突变时还是会颇感欣慰。编写一个图形化的展示如何呢,这样每次碱基突变的时候,它都会有一个小的爆炸特效,而且颜色也高亮显示,这样你就可以实时观看突变的发生了。

在你对此进行嘲笑之前,你应该知道好的图形化展示对于大多数程序的成功是多么重要。这听起来可能有点像是一个雕虫小技的图形,但是如果你能够演示大多数常见的突变,比如用这种方式演示BRCA乳腺癌基因,它就会非常有用。

\subsection{程序中的一个Bug?}
言归正传,在查看\autoref{exam:example7.2}的输出时你可能已经注意到了某些事情。看一下"10 more successive mutations"部分的前两行吧,它们是完全一样的!在欣喜若狂、对自己能够完成如此漂亮的工作而颇感自豪之际,竟然发现了一个bug?

你该如何追踪它呢?就像在\autoref{chap:chapter6}学习的那样,你可能想用Perl调试器来一步步地运行程序。但是这一次,不要这么做,停下来先思考一下你的设计吧。你使用随机选取的碱基来替换随机位置上的碱基。啊!有的时候,你随机选取的某个位置上的碱基和你随机选取用来进行替换该位置碱基的碱基是完全一样的!偶尔,你用一个碱基替换了它本身!\footnote{这有多频繁呢?对于DNA中的每一个碱基,其出现的概率都为1/4。对于一个由四种碱基等概率出现构成的DNA来说。这种情况发生的频率就高达1/4!} 

我们假设,你觉得这种方式不是很有用,对于每一成功的突变,你都应该看到一个碱基的改变。你该如何修改代码来实现这一点呢?让我们先从\textit{mutate}子程序的伪代码开始吧:

\begin{lstlisting}
Select a random position in the string of DNA

Repeat:

  Choose a random nucleotide

Until: random nucleotide differs from the nucleotide in the random position

Substitute the random nucleotide into the random position in the DNA
\end{lstlisting}

这看起来应该管用,所以你修改了\textit{mutate}子程序,把它改名为\textit{mutate\_better}子程序:

\begin{lstlisting}
# mutate_better
#
# Subroutine to perform a mutation in a string of DNA--version 2, in which
#  it is guaranteed that one base will change on each call
#
# WARNING: make sure you call srand to seed the
#  random number generator before you call this function.

sub mutate_better {

  my($dna) = @_;
  my(@nucleotides) = ('A', 'C', 'G', 'T');

  # Pick a random position in the DNA
  my($position) = randomposition($dna);

  # Pick a random nucleotide
  my($newbase);

  do {
    $newbase = randomnucleotide(@nucleotides);

  # Make sure it's different than the nucleotide we're mutating
  }until ( $newbase ne substr($dna, $position,1));

  # Insert the random nucleotide into the random position in the DNA
  # The substr arguments mean the following:
  #  In the string $dna at position $position change 1 character to
  #  the string in $newbase
  substr($dna,$position,1,$newbase);

  return $dna;
}
\end{lstlisting}

当你用这个子程序替换掉\textit{mutate}后,运行代码,你会得到下面的输出:

\begin{lstlisting}
Mutate DNA

Here is the original DNA:

AAAAAAAAAAAAAAAAAAAAAAAAAAAAAA

Here is the mutant DNA:

AAAAAAAAAAAAATAAAAAAAAAAAAAAAA

Here are 10 more successive mutations:

AAAAAAAAAAAAATAAAAAAAACAAAAAAA
AAAAATAAAAAAATAAAAAAAACAAAAAAA
AAATATAAAAAAATAAAAAAAACAAAAAAA
AAATATAAAAAAATAAAAAAAACAACAAAA
AATTATAAAAAAATAAAAAAAACAACAAAA
AATTATTAAAAAATAAAAAAAACAACAAAA
AATTATTAAAAAATAAAAAAAACAACACAA
AATTATTAAAAAGTAAAAAAAACAACACAA
AATTATTAAAAAGTGAAAAAAACAACACAA
AATTATTAAAAAGTGATAAAAACAACACAA
\end{lstlisting}

看起来,在每次迭代的时候,确实都有一个碱基发生了改变。

对于声明变量还需要注意一点。在\textit{mutate\_better}的代码中,如果你在循环中声明了 \verb|$newbase|变量,因为循环被包裹在了代码块中,所以变量 \verb|$newbase|在循环外就是不可见的。具体来说,在突变过程中实际进行碱基替换的 \verb|substr|调用中,它没法使用。所以,在\textit{mutate\_better}中,你必须要在循环外声明这个变量。

对于那些喜欢随用随声明变量的程序员来说,这常常是一些混乱的源头,这也是一个强有力的证据,让你养成在程序顶部收集所有变量定义的习惯。

即使这样,有的时候你还是想把变量隐藏在代码块中,因为那里是你唯一会用到这个变量的地方。这时你就会想在代码块中对其进行声明。(如果代码块很长,可能就会在代码块的顶部进行声明?)

\section{生成随机DNA} 
有时处于测试的目的,生成随机的数据会非常有用。同样可以使用随机DNA来研究生物体中真实DNA的组织方式。在本小节中,我们就来编写一些生成随机DNA序列的程序。

这样的随机DNA序列在很多方面都非常有用。比如,流行的BLAST程序(参看\autoref{chap:chapter12}),就要依赖于随机DNA的属性来获得对序列相似性打分进行评估的分析和经验性结果,以及用于对“击中”进行排序的统计结果,这会被BLAST反馈给用户。

假设我们需要的是一系列长短不一的随机DNA片段。你的程序必须要设定一个最大和最小的长度,以及生成DNA片段的数目。

\subsection{自下而上 vs. 自上而下}
在\autoref{exam:example7.2}中,你编写了一些最基本的子程序,然后一个子程序调用这些基本的子程序,最后实现的是主程序。如果你忽略伪代码,这就是\textit{自下而上设计}的一个例子:从最基本的砖瓦开始,然后把它们组装成高楼大厦。

现在我们来看看另一种设计,它从主程序开始,然后是其中的子程序调用,当你发现需要子程序时你才编写它们。这就叫做\textit{自上而下设计}。

\subsection{生成一系列随机DNA的子程序}
考虑到我们生成随机DNA的目标,你需要的可能是一个直接生成数据的子程序:

\begin{lstlisting}
@random_DNA = make_random_DNA_set( $minimum_length, $maximum_length, $size_of_set );
\end{lstlisting}

这看起来没有问题,但还是要看如何来真正完成整个任务。(对你来说这就是自上而下的设计!)所以你需要一步步深入,编写\textit{make\_random\_DNA\_set}子程序的伪代码:

\begin{lstlisting}
repeat $size_of_set times:

  $length = random number between minimum and maximum length

  $dna = make_random_DNA ( $length );

  add $dna to @set
}

return @set
\end{lstlisting}

现在,继续自上而下的设计,你需要\textit{make\_random\_DNA}子程序的伪代码:

\begin{lstlisting}
from 1 to $size

  $base = randomnucleotide

  $dna .= $base
}

return $dna
\end{lstlisting}

不需要更加深入了,因为在\autoref{exam:example7.2}中你已经编写了\textit{randomnucleotide}子程序。

(你是否对伪代码中不配对的大括号感到厌烦?此处,你依赖于缩进,并用右大括号来表明代码块。既然是伪代码,只要它能工作,一切都是允许的。)

\subsection{把设计变成代码}
现在我们已经有了自上而下的设计,该如何编写代码呢?我们还是继续自上问下的设计,来看看它是如何工作的。

\autoref{exam:example7.3}按照伪代码中自上而下的设计顺序一步步前进,从主程序开始,之后才是子程序。

%\textbf{例7-3:生成随机DNA}
\lstinputlisting[label=exam:example7.3,caption={例7.3:生成随机DNA}]{./scripts/example7-3.pl}

下面是\autoref{exam:example7.3}的输出:

\begin{lstlisting}
Here is an array of 12 randomly generated DNA sequences
  with lengths between 15 and 30:

TACGCTTGTGTTTTCGGGGGAC
GGGGTGTGGTAAGGCTGTCTCAGATGTGC
TGAACGACAACCTCCTGGACTTTACT
ATCTATGCTTTGCCATGCTAGT
CCGCTCATTCCTCTTCCTCGGC
TGTACCCCTAATACACTTTAGCCGAATTTA
ATAGGTCGGGGCGACAGCGCCGG
GATTGACCTCTGTAA
AAAATCTCTAGGATCGAGC
GTATGTGCTTGGGTAAAT
ATGGAGTTGCGAGGAAGTAGCTGAGT
GGCCCATGACCAGCATCCAGACAGCA
\end{lstlisting}

\section{分析DNA}
在处理随机化的最后这个例子中,你将收集DNA的一些统计信息,来回答这个问题:平均来说,对于两个随机的DNA序列,它们的碱基相同的百分比是多少?尽管一些简单的数学计算就可以帮你回答这个问题,但这个程序的要点在于表明你现在已经有必需的编程能力来提问并回答关于DNA序列的问题了。(如果你在使用真实的DNA,比如收集到的在不同生物中略有不同的某个特定基因,这个问题就更加有趣了。稍后你可能会想尝试一下。)

让我们生成一个随机DNA的集合,所有的DNA长度都相等,然后对这个集合思考下面的问题:对于集合中成对的DNA序列来说,相同碱基位置的百分比平均是多少?

像平常一样,我们先尝试用伪代码勾画出程序的思路:

\begin{lstlisting}
Generate a set of random DNA sequences, all the same length

For each pair of DNA sequences

  How many positions in the two sequences are identical as a fraction?

}

Report the mean of the preceding calculations as a percentage
\end{lstlisting}

显而易见,要编写这个代码,你至少可以重用已经完成的一部分工作。你已经知道如何生成一系列随机的DNA序列。此外,虽然你还没有按照位置逐个比较两条序列碱基的子程序,但你知道如何查找DNA字符串中的位置。所以,这样的子程序并不难写。事实上,我们写一些伪代码,来把一个序列上的每一个核苷酸和另一个序列上相同位置的核苷酸进行比较:

\begin{lstlisting}
assuming DNA1 is the same length as DNA2,

for each position from 1 to length(DNA)

  if the character at that position is the same in DNA_1 and DNA_2

    ++$count
  }
}

return count/length
\end{lstlisting}

这个问题已经迎刃而解了。当然,你还需要编写一些代码,来挑选成对的序列,收集计算结果,最终获得结果的平均值并以百分比的形式将它报告出来。所有这些都在主程序中。\autoref{exam:example7.4}就是这样的尝试,所有的内容都在其中。

%\textbf{例7-4:计算成对随机DNA序列的平均一致性百分比}
\lstinputlisting[label=exam:example7.4,caption={例7.4:计算成对随机DNA序列的平均一致性百分比}]{./scripts/example7-4.pl}

\autoref{exam:example7.4}中的代码看起来和前面例子中的代码有些重复,确实是这样的。为了便于描述,我把子程序代码都放在了程序中。(在\autoref{chap:chapter8}中,你将开始使用模块,就会避免这种重复。)

下面是\autoref{exam:example7.4}的输出:

\begin{lstlisting}
In this run of the experiment, the average number of 
matching positions is 0.24%
\end{lstlisting}

好吧,这看起来还算合理。你可能会说,这是显而易见的:四分之一的位置相匹配,因为一共有四种碱基。但是,这点并不足以验证基本概率,它只是让你明白,你已经有足够的编程技能傍身,来编写一些针对DNA序列提问和回答问题的程序了。

\subsection{关于代码的一些注释}
注意,在主程序中,当调用:

\begin{lstlisting}
@random_DNA = make_random_DNA_set( 10, 10, 10 );
\end{lstlisting}

时,你并不需要声明并初始化像 \verb|$minimum_length|这样的变量。你只需要在调用子程序时填入真实的数值即可。(然而,把这样的东西存储在程序顶部声明的变量中,通常都是一个比较好的做法,因为这样比较容易寻找并修改它们。)这个例子中,你把最大和最小的长度都设置成了10,同时要求生成10条序列。

让我们重申一下刚刚解决的问题。你需要比较所有的DNA对,对每一对DNA,都要计算有相同核苷酸的位置的百分比。然后,你要得到这些百分比的平均值。

在\autoref{exam:example7.4}主程序中实现这一点的代码如下所示::

\begin{lstlisting}
# Iterate through all pairs of sequences
for (my $k = 0 ; $k < scalar @random_DNA - 1 ; ++$k) {
  for (my $i = ($k + 1) ; $i < scalar @random_DNA ; ++$i) {

    # Calculate and save the matching percentage
    $percent = matching_percentage($random_DNA[$k], $random_DNA[$i]);
    push(@percentages, $percent);
  }
}
\end{lstlisting}

为了比较每一对DNA,你使用了嵌套的循环。所谓\textit{嵌套循环}就是指在一个循环中还有另一个循环。这在编程中非常常见,但一定要小心处理它们。这看起来可能有一点复杂,花些时间来看看嵌套循环是如何工作的,因为当从一个集合中选取两个(或多个)元素的组合时这很常用。

这里的嵌套循环要查看$(n * (n-1)) / 2$对序列,这是数据集大小的平方函数。它会变得非常大!试着逐渐增加数据集的大小,然后重新运行程序,你会发现运算时间增加了,而且远比线性增加要大。

看看循环是如何工作的?首先,序列0(以 \verb|$K|进行索引)依次跟序列1、2、3、……9(以 \verb|$i|进行索引)进行配对。之后,序列1依次和2、3、……9进行配对,如此往复,最后,序列8和序列9进行配对。(回忆一下,数组元素的计数是从0开始的,所以有10个元素的数组的最后一个元素的索引是9。此外,回忆以下,标量 \verb|@random_DNA|返回的是数组中元素的数目。)

你可能发现这样做是比较值得的,就是把序列数目设置成一个小的数值,比如3或者4,然后思考(手中拿着纸笔)在程序运行过程中嵌套循环是如何工作的,变量 \verb|$k|和 \verb|$i|是如何一步步改变的。或者,你也是使用Perl调试器来看看到底发生了什么。

\section{练习题}
\textcolor{red}{\textit{习题7.1}}
\begin{adjustwidth}{4em}{}
编写一个程序,询问你选取一个氨基酸,然后(随机)猜测你选的是哪个氨基酸。
\end{adjustwidth}

\textcolor{red}{\textit{习题7.2}}
\begin{adjustwidth}{4em}{}
编写一个程序,选取四个核苷酸中的一个,然后一直进行提示,直到你正确猜出选取的是哪个核苷酸。
\end{adjustwidth}

\textcolor{red}{\textit{习题7.3}}
\begin{adjustwidth}{4em}{}
编写一个子程序,随机打乱数组的元素。子程序以一个数组作为参数,返回具有相同元素、但打乱成随机顺序的数组。原始数组中的每一个元素在输出的数组中都出现且仅出现一次,就像洗牌一样。
\end{adjustwidth}

\textcolor{red}{\textit{习题7.4}}
\begin{adjustwidth}{4em}{}
编写一个程序来突变蛋白质序列,就像\autoref{exam:example7.2}中突变DNA的代码一样。
\end{adjustwidth}

\textcolor{red}{\textit{习题7.5}}
\begin{adjustwidth}{4em}{}
编写一个子程序,给定一个密码子(一个长3bp的DNA片段),返回密码子中的一个随机突变。
\end{adjustwidth}

\textcolor{red}{\textit{习题7.6}}
\begin{adjustwidth}{4em}{}
有些版本的Perl为随机数生成器自动设置种子,这样在使用 \verb|rand|生成随机数之前就不用多余使用 \verb|srand|来设置种子了。实验一下,看看 \verb|rand|是否会自动调用 \verb|srand|,还是需要你自己明确地去调用 \verb|srand|,就像你在本章的代码中看到的那样。
\end{adjustwidth}

\textcolor{red}{\textit{习题7.7}}
\begin{adjustwidth}{4em}{}
有时,在随机选取的时候并不是所有的选项都会被选择到。编写一个子程序,随机返回一个核苷酸,而且是按照指定的每个核苷酸的几率。给这个子程序传递四个数字作为参数,代表每中核苷酸的几率。如果每种核苷酸的几率都是0.25,这个子程序在选取每种核苷酸时就是完全均等的。为了检查错误,子程序还要确保四个几率之和为1。

\textit{提示:}实现这一点的一种方法就是,把从0到1的范围分成四个区间,每个区间的长度和相应核苷酸的几率相对应。然后,简单的从0到1之间随机选取一个数字,看看它落在了哪个区间中,就返回相对应的核苷酸即可。
\end{adjustwidth}

\textcolor{red}{\textit{习题7.8}}
\begin{adjustwidth}{4em}{}
\textit{这是一个有一定难度的练习题。}Perl中的\textit{study}函数可能能提高DNA或蛋白质中基序的查找速度。查看关于这个函数的Perl文档。它的用法非常简单:给定存储在 \verb|$sequence|变量中的一些序列数据,在进行搜索之前键入:

\begin{lstlisting}
study $sequence;
\end{lstlisting}

基于你已经阅读的文档中的相关内容,你认为\textit{study}会提高DNA或蛋白质的搜索速度吗?
\end{adjustwidth}

\textit{可以得到大量更多的得分!}现在阅读标准模块Benchmark的Perl文档。(键入 \verb|perldoc Benchmark|,或者访问 \href{http://www.perl.com}{http://www.perl.com} 上的Perl主页。)编写一个程序,在使用和不使用\textit{study}的情况下,分别检测DNA和蛋白质基序的搜索速度,看看你的猜测是不是正确。

\documentclass[11pt,a4paper,twoside]{book}

\usepackage{fontspec}
\setmainfont{Times New Roman}
\setsansfont{Arial}
\setmonofont{Courier New}

\usepackage[BoldFont,SlantFont,CJKchecksingle,CJKnumber]{xeCJK}
\setCJKmainfont[BoldFont={Adobe Heiti Std},ItalicFont={Adobe Kaiti Std}]{Adobe Song Std}
\setCJKsansfont{Adobe Heiti Std}
\setCJKmonofont{Adobe Fangsong Std}
\punctstyle{hangmobanjiao}

\defaultfontfeatures{Mapping=tex-text}
\usepackage{xunicode}
\usepackage{xltxtra}

\XeTeXlinebreaklocale "zh"
\XeTeXlinebreakskip = 0pt plus 1pt minus 0.1pt

%\usepackage{indentfirst}
\makeatletter
\let\@afterindentfalse\@afterindenttrue
\@afterindenttrue
\makeatother
\setlength{\parindent}{2em}

\linespread{1.2}

\usepackage[a4paper,top=2.5cm,bottom=1.5cm,left=3cm,right=1.5cm,marginparwidth=4.7cm,marginparsep=0.3cm]{geometry}

%在每一章的开头列出section的列表
\usepackage{minitoc}
\setcounter{minitocdepth}{1}
\renewcommand{\mtctitle}{目录}

%整段缩进
\usepackage{changepage}

%图文混排
\usepackage{picins}

%在footnote中使用verb
\usepackage{fancyvrb}

%文本框
\usepackage{fancybox}

%在section中使用\verb
\usepackage{cprotect}

\usepackage{fancyhdr}
\pagestyle{fancy}
\renewcommand{\chaptermark}[1]{\markboth{\small 第 \thechapter 章\quad #1}{}}
\renewcommand{\sectionmark}[1]{\markright{\small \thesection \quad #1}{}}
\fancyhf{}
\fancyhead[ER]{\leftmark}
\fancyhead[OL]{\rightmark}
\fancyhead[EL,OR]{$\cdot$ \thepage \ $\cdot$}
\renewcommand{\headrulewidth}{0.5pt}

%单面模式
%\newpagestyle{main}{
%\sethead[$\cdot$~\thepage~$\cdot$][][\chaptername\quad\chaptertitle]{\small\S\,\thesection\quad\sectiontitle}{}{$\cdot$~\thepage~$\cdot$}
%\setfoot{}{}{}\headrule}
%\pagestyle{main}

\usepackage{titlesec}
\titleformat{\chapter}{\centering\LARGE\bfseries}{第 \thechapter 章}{1em}{}
%\titlespacing*{\section}{0pt}{0.2\baselineskip}{0.2\baselineskip}

\usepackage[dvipsnames,tables]{xcolor}
\usepackage{graphicx}
\graphicspath{{figures/}}
\usepackage[xetex,bookmarksnumbered=true,bookmarksopen=true,pdfborder=1,breaklinks,colorlinks,linkcolor=blue,urlcolor=blue,citecolor=blue]{hyperref}
\def\chapterautorefname~#1\null{第{#1}章\null}
\def\sectionautorefname~#1\null{第{#1}节\null}
\def\subsubsectionautorefname~#1\null{第{#1}小节\null}
\def\tableautorefname{表}
\def\lstlistingautorefname{例}
\newtheorem{example}{例}[chapter]

\renewcommand{\today}{\number\year 年 \number\month 月 \number\day 日}
\renewcommand{\contentsname}{目录}
\renewcommand{\listfigurename}{插图目录}
\renewcommand{\listtablename}{表格目录}
\renewcommand{\figurename}{图}
\renewcommand{\tablename}{表}
\renewcommand{\appendixname}{附录}
\renewcommand{\indexname}{索引}
%\renewcommand{\chaptername}{第 \CJKnumber{\thechapter} 章}
%\renewcommand{\chaptername}{第 \thechapter 章}
%\titleformat{\chapter}{\centering\normalfont\Huge\bfseries}{\chaptername}{20pt}{\Huge}
\renewcommand{\itemautorefname}{项}
\renewcommand{\appendixautorefname}{附录}
\renewcommand{\theoremautorefname}{定理}
\renewcommand{\figureautorefname}{图}
\renewcommand{\tableautorefname}{表}
\renewcommand{\footnoteautorefname}{脚注}

%使表格可以跨页
%\usepackage{booktabs,longtable}
\usepackage{booktabs,tabu,longtable}

%调整表格行高
\renewcommand{\arraystretch}{0.8}

%调整列表间及其上下的间距
%\usepackage{mdwlist}
\usepackage{enumitem}
\setlist{nosep}

%设置颜色的快捷命令
\newcommand{\red}{\textcolor{red}}
\newcommand{\gray}{\textcolor{gray}}
\newcommand{\black}{\textcolor{black}}

%罗马数字
\makeatletter
\newcommand{\rmnum}[1]{\romannumeral #1}
\newcommand{\Rmnum}[1]{\expandafter\@slowromancap\romannumeral #1@}
\makeatother

%插入源代码
\usepackage{listings}
%\lstset{
  %language=Perl,                  % 程序语言名称:TeX, Perl, R, sh, bash, Awk
  %basicstyle=\normalsize\tt,      %\tt指monospace字体族,程序源代码使用此族字体表示更加美观
  %%commentstyle=\color{red!50!green!50!blue!50}\tt,  %浅灰色的注释
  %commentstyle=\color{gray}\tt,
  %%keywordstyle=\color{blue!70}\bfseries\tt,         %代码关键字的颜色为蓝色,粗体
  %keywordstyle=\color{cyan}\bfseries\tt,
  %identifierstyle=\tt,
  %stringstyle=\color{black}\tt,                % 代码字符串的特殊格式
  %numbers=left,                   % 行号位置(左侧)
  %numberstyle=\small,             % 行号字体的字号
  %stepnumber=1,                   % 行号的显示步长
  %numbersep=5pt,                  % 行号与代码间距
  %backgroundcolor=\color{white},  % 背景色;需要 \usepackage{color}
  %%rulesepcolor=\color{red!20!green!20!blue!20},     %代码块边框为淡青色
  %rulesepcolor=\color{gray},
  %rulecolor=\color{black},
  %%frame=shadowbox,                % 把代码用带有阴影的框圈起来
  %frame=lrtb,
  %%frame=single,
  %%aboveskip=1em,                  %代码块边框
  %%xleftmargin=2em,
  %%xrightmargin=2em,
  %xleftmargin=\fboxsep,
  %xrightmargin=-\fboxsep,
  %captionpos=b,                   % 标题位置
  %keepspaces=true,
  %showspaces=false,               % 不显示空格
  %showstringspaces=false,         % 不显示代码字符串中的空格标记
  %showtabs=false,                 % 不显示 TAB
  %tabsize=4,
  %breaklines=true,                % 对过长的代码自动断行
  %breakatwhitespace=false,        % 断行只在空格处
  %breakautoindent=true,
  %%breakindent=22pt,
  %%breakindent=1em,
  %extendedchars=false,            % 解决代码跨页时,章节标题,页眉等汉字不显示的问题
  %flexiblecolumns=true,
  %%escapeinside={\%*}{*},         % 跳脱字符,添加注释,暂时离开 listings 
  %%escapeinside=``,
%}
\lstset{
  language=Perl,
  basicstyle=\normalsize\tt,
  frame=l,
  numbers=left,
  numberstyle=\footnotesize,
  showstringspaces=false,
  breaklines=true,
  breakatwhitespace=false,
}

%\lstset{ %
  %backgroundcolor=\color{white},   % choose the background color
  %basicstyle=\footnotesize,        % size of fonts used for the code
  %breaklines=true,                 % automatic line breaking only at whitespace
  %captionpos=b,                    % sets the caption-position to bottom
  %commentstyle=\color{OliveGreen},    % comment style
  %keywordstyle=\color{BlueViolet},       % keyword style
  %stringstyle=\color{black},     % string literal style
  %language=Perl,             % Set your language (you can change the language for each code-block optionally)
  %frame=lrtb, %
  %xleftmargin=\fboxsep, %
  %xrightmargin=-\fboxsep, %
  %%moretexcs={lstset,color,colorlet, cellcolor, newcolumntype, columncolor, rowcolor, multirow, xspace, LaTeX, TeX},
%}

\documentclass[11pt,a4paper,twoside]{book}

\usepackage{fontspec}
\setmainfont{Times New Roman}
\setsansfont{Arial}
\setmonofont{Courier New}

\linespread{1.2}
%取消每个段落的缩进
\setlength\parindent{0pt}
%调整段落之间的间距
%\setlength{\parskip}{1em}
\setlength{\parskip}{1.3ex plus 0.5ex minus 0.3ex}

\usepackage[a4paper,top=2.5cm,bottom=1.5cm,left=3cm,right=1.5cm,marginparwidth=4.7cm,marginparsep=0.3cm]{geometry}

%在每一章的开头列出section的列表
\usepackage{minitoc}
\setcounter{minitocdepth}{1}

%整段缩进
\usepackage{changepage}

%图文混排
\usepackage{picins}

%在footnote中使用verb
\usepackage{fancyvrb}

%文本框
\usepackage{fancybox}

%在section中使用\verb
\usepackage{cprotect}

\usepackage{fancyhdr}
\pagestyle{fancy}
\fancyhf{}
\fancyhead[ER]{\leftmark}
\fancyhead[OL]{\rightmark}
\fancyhead[EL,OR]{$\cdot$ \thepage \ $\cdot$}
\renewcommand{\headrulewidth}{0.5pt}

%单面模式
%\newpagestyle{main}{
%\sethead[$\cdot$~\thepage~$\cdot$][][\chaptername\quad\chaptertitle]{\small\S\,\thesection\quad\sectiontitle}{}{$\cdot$~\thepage~$\cdot$}
%\setfoot{}{}{}\headrule}
%\pagestyle{main}

\usepackage{xcolor}
\usepackage{graphicx}
\graphicspath{{figures/}}
\usepackage[xetex,bookmarksnumbered=true,bookmarksopen=true,pdfborder=1,breaklinks,colorlinks,linkcolor=blue,urlcolor=blue,citecolor=blue]{hyperref}
%\def\sectionautorefname{Section}
\def\chapterautorefname{Chapter}
\def\sectionautorefname{Section}
\def\subsectionautorefname{Subsection}
\def\subsubsectionautorefname{Subsubsection}
\def\tableautorefname{Table}
\def\lstlistingautorefname{Example}
%\def\chapterautorefname{Chap.}
%\def\sectionautorefname{Sec.}
%\def\subsectionautorefname{sub--Sec.}
%\def\figureautorefname{Fig.}
%\def\tableautorefname{Tab.}
%\def\equationautorefname{Eq.}

%使表格可以跨页
%\usepackage{booktabs,longtable}
\usepackage{booktabs,tabu}
%调整表头和表格之间的间距
\setlength\belowcaptionskip{5pt}
%\usepackage{caption} 
%\captionsetup[table]{skip=5pt}

%调整表格行高
\renewcommand{\arraystretch}{0.8}

%调整列表间及其上下的间距
%\usepackage{mdwlist}
\usepackage{enumitem}
\setlist{nosep}

%设置颜色的快捷命令
\newcommand{\red}{\textcolor{red}}
\newcommand{\gray}{\textcolor{gray}}
\newcommand{\black}{\textcolor{black}}

%罗马数字
\makeatletter
\newcommand{\rmnum}[1]{\romannumeral #1}
\newcommand{\Rmnum}[1]{\expandafter\@slowromancap\romannumeral #1@}
\makeatother

%插入源代码
\usepackage{listings}
\lstset{
  language=Perl,                  % 程序语言名称:TeX, Perl, R, sh, bash, Awk
  basicstyle=\normalsize\tt,      %\tt指monospace字体族,程序源代码使用此族字体表示更加美观
  numbers=left,                   % 行号位置(左侧)
  numberstyle=\small,             % 行号字体的字号
  stepnumber=1,                   % 行号的显示步长
  numbersep=5pt,                  % 行号与代码间距
  backgroundcolor=\color{white},  % 背景色;需要 \usepackage{color}
  showspaces=false,               % 不显示空格
  showstringspaces=false,         % 不显示代码字符串中的空格标记
  showtabs=false,                 % 不显示 TAB
  tabsize=4, 
  frame=shadowbox,                % 把代码用带有阴影的框圈起来
  captionpos=b,                   % 标题位置
  breaklines=true,                % 对过长的代码自动断行
  breakatwhitespace=false,        % 断行只在空格处
  extendedchars=false,            % 解决代码跨页时,章节标题,页眉等汉字不显示的问题
  %escapeinside={\%*}{*},         % 跳脱字符,添加注释,暂时离开 listings 
  %escapeinside=``,
  commentstyle=\color{red!50!green!50!blue!50}\tt,  %浅灰色的注释
  rulesepcolor=\color{red!20!green!20!blue!20},     %代码块边框为淡青色
  keywordstyle=\color{blue!70}\bfseries\tt,         %代码关键字的颜色为蓝色,粗体
  identifierstyle=\tt,
  stringstyle=\tt,                % 代码字符串的特殊格式
  keepspaces=true,
  breakindent=22pt,
  %breakindent=4em,
  breakautoindent=true,
  flexiblecolumns=true,
  aboveskip=1em,                  %代码块边框
  xleftmargin=2em,
  xrightmargin=2em
}
%\renewcommand\lstlistingname{Example}
%\renewcommand\lstlistlistingname{Examples}

%\usepackage{chngcntr}% http://ctan.org/pkg/chngcntr
%\AtBeginDocument{\counterwithin{lstlisting}{chapter}}

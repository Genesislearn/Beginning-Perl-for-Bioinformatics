\documentclass[11pt,a4paper,twoside]{book}

\usepackage{fontspec}
\setmainfont{Times New Roman}
\setsansfont{Arial}
\setmonofont{Courier New}

\linespread{1.2}
%取消每个段落的缩进
\setlength\parindent{0pt}
%调整段落之间的间距
%\setlength{\parskip}{1em}
\setlength{\parskip}{1.3ex plus 0.5ex minus 0.3ex}

\usepackage[a4paper,top=2.5cm,bottom=1.5cm,left=2cm,right=1.5cm,marginparwidth=4.7cm,marginparsep=0.3cm]{geometry}

%在每一章的开头列出section的列表
\usepackage{minitoc}
\setcounter{minitocdepth}{1}

%修改目录中的章节格式
\usepackage{titletoc}
\titlecontents{chapter}[0pt]{\vspace{0.3\baselineskip}\bfseries}{Chapter \thecontentslabel\quad}{}{\hfill\contentspage}
\titlecontents{section}[2em]{\vspace{0.05\baselineskip}}{\thecontentslabel\quad}{}{\hspace{.5em}\titlerule*{\ldots}\contentspage}
\titlecontents{subsection}[4em]{\vspace{0.02\baselineskip}}{\thecontentslabel\quad}{}{\hspace{.5em}\titlerule*{\ldots}\contentspage}
\titlecontents{figure}[0pt]{\vspace{0.05\baselineskip}}{\thecontentslabel\quad}{}{\hspace{.5em}\titlerule*[8pt]{\ldots}\contentspage}
\titlecontents{table}[0pt]{\vspace{0.05\baselineskip}}{\thecontentslabel\quad}{}{\hspace{.5em}\titlerule*[8pt]{\ldots}\contentspage}

%附录
%\usepackage[title,titletoc]{appendix}
\usepackage{appendix}

%整段缩进
\usepackage{changepage}

%图文混排
\usepackage{picins}

%在footnote中使用verb
\usepackage{fancyvrb}
%在section中使用\verb
\usepackage{cprotect}

%文本框
%\usepackage{fancybox}

\usepackage{fancyhdr}
\pagestyle{fancy}
\fancyhf{}
\fancyhead[ER]{\leftmark}
\fancyhead[OL]{\rightmark}
\fancyhead[EL,OR]{$\cdot$ \thepage \ $\cdot$}
\renewcommand{\headrulewidth}{0.5pt}

%单面模式
%\newpagestyle{main}{
%\sethead[$\cdot$~\thepage~$\cdot$][][\chaptername\quad\chaptertitle]{\small\S\,\thesection\quad\sectiontitle}{}{$\cdot$~\thepage~$\cdot$}
%\setfoot{}{}{}\headrule}
%\pagestyle{main}

\usepackage{titlesec}
\titleformat{\chapter}{\centering\LARGE\bfseries}{Chapter \thechapter}{1em}{}

\usepackage{xcolor}
\usepackage{graphicx}
\graphicspath{{figures/}}
%\usepackage[xetex,bookmarksnumbered=true,bookmarksopen=true,pdfborder=1,breaklinks,colorlinks,linkcolor=blue,urlcolor=blue,citecolor=blue]{hyperref}
\usepackage[xetex,bookmarksnumbered=true,bookmarksopen=true,pdfborder=1,breaklinks,colorlinks]{hyperref}

\def\chapterautorefname{Chapter}
\def\sectionautorefname{Section}
\def\subsectionautorefname{Subsection}
\def\subsubsectionautorefname{Subsubsection}
\def\figureautorefname{Figure}
\def\tableautorefname{Table}
\def\Appendixautorefname{Appendix}
\def\lstlistingautorefname{Example}
%\def\chapterautorefname{Chap.}
%\def\sectionautorefname{Sec.}
%\def\subsectionautorefname{sub--Sec.}
%\def\figureautorefname{Fig.}
%\def\tableautorefname{Tab.}

\renewcommand{\listfigurename}{Figures}
\renewcommand{\listtablename}{Tables}
\renewcommand{\listtablename}{Tables}

%使表格可以跨页
\usepackage{booktabs,tabu,longtable}
%调整表头和表格之间的间距
\setlength\belowcaptionskip{5pt}
%调整表格行高
\renewcommand{\arraystretch}{0.8}

%调整列表间及其上下的间距
\usepackage{enumitem}
\setlist{nosep}

%设置颜色的快捷命令
\newcommand{\red}{\textcolor{red}}
\newcommand{\gray}{\textcolor{gray}}
\newcommand{\black}{\textcolor{black}}

%罗马数字
\makeatletter
\newcommand{\rmnum}[1]{\romannumeral #1}
\newcommand{\Rmnum}[1]{\expandafter\@slowromancap\romannumeral #1@}
\makeatother

%插入源代码
\usepackage{listings}
\renewcommand{\lstlistlistingname}{Programs}
%\renewcommand{\lstlistingname}{Example}
\contentsuse{lstlisting}{lol}
\titlecontents{lstlising}[0pt]{\vspace{0.05\baselineskip}}{}{\thecontentslabel\quad}{\hspace{.5em}\titlerule*[8pt]{\ldots}\contentspage}
\lstset{
  language=Perl,
  basicstyle=\small\tt,
  frame=l,
  numbers=left,
  numberstyle=\footnotesize,
  showstringspaces=false,
  %breaklines=ture,
  breaklines=false,
  breakatwhitespace=false,
}
\usepackage{caption}
\captionsetup[lstlisting]{labelformat=empty,labelsep=none,format=plain,font=bf}
%\captionsetup[lstlising]{labelformat=empty,labelsep=none,textformat=empty,skip=-20pt}

\usepackage{textcomp}

\usepackage{emptypage}

